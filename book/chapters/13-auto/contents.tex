\chapter{Automatizarea sarcinilor}
\label{chapter:auto}

În utilizarea și administrarea sistemului de operare sau a unei rețele sau infrastructuri informatice mai ample, avem adesea acțiuni pe care le repetăm.
Fie că avem de compilat o aplicație, de pornit o mașină virtuală, de prelucrat datele dintr-o bază de date, de testat o aplicație desktop, mobilă sau web, de generat un raport de folosire a rețelei, este util să avem moduri prin care să
automatizăm aceste aceste acțiuni.

\begin{definition}{Automatizarea unei acțiuni}
Automatizarea unei acțiuni înseamnă executarea acesteia fără intermedierea unei persoane, fără ca aceasta să fie interactivă.
O acțiune interactivă solicită intervenția utilizatorului în vreme ce o acțiune neinteractivă nu are nevoie de intervenția utilizatorului.
Automatizarea înseamnă eliminarea sau înlocuirea părților interactive dintr-o acțiune sau un set de acțiuni.
\end{definition}

Un exemplu de automatizare este asigurat de utilitarul Make, prezentat în \labelindexref{Secțiunea}{sec:appdev:automation}.
Fără utilitarul Make, un dezvoltator ar trebui să compileze toate fișierele obiect, apoi să le link-editeze împreună cu biblioteci și, eventual, să le compună cu alte fișiere într-un pachet software.
Folosind Make și fișiere Makefile, dezvoltatorul automatizează procesul și permite ca toți pașii să fie executați la rularea comenzii \cmd{make}.
Mai mult, comanda \cmd{make} poate fi declanșată la un moment de timp sau la un alt eveniment; este cazul proceselor de tip \textit{nightly build} despre care vom discuta mai târziu.

Un alt exemplu de automatizare, de această dată la nivelul unui administrator de datacenter, este crearea de mașini virtuale.
Dacă un administrator are nevoie să creeze sute de mașini virtuale, nu le va crea pe rând, interactiv.
Ar dura prea mult și ar fi repetitiv și plictisitor.
Administratorul va automatiza procesul creării unei mașini virtuale și apoi va generaliza comanda pentru mai multe instanțe.
În final, prin rularea unei singure comenzi vor fi create toate mașinile virtuale.

Un utilizator tehnic, în special unul care are rol administrativ într-o organizație, are nevoie de cunoștințe și abilități de automatizare pentru a își eficientiza timpul de lucru.
Un utilizator care face mereu aceleași acțiuni în mod repetat va consuma timp care ar putea fi investit în alte acțiuni.
În momentul când automatizează acțiuni, acel timp consumat de repetiții va putea fi folosit în scopuri creative.
Activitățile creative nu pot fi automatizate, însă cele repetitive da, și un utilizator înțelept le automatizează pe cele din urmă ca să aibă cât mai multă energie și timp pentru celelalte.
Într-o notă mai amuzantă, după spusele unui prieten, un administrator de infrastructură informatică ar trebui plătit cu atât mai bine cu cât stă mai mult timp degeaba: înseamnă că acțiunile uzuale sunt automatizate iar cele critice apar rar și sunt bine tratate.

Termenul de automatizare este folosit în sens mai larg și la domenii diferite de domeniul IT\abbrev{IT}{Information Technology}: irigații automate, folosirea de roboți pentru automatizarea creării de componente, mașini autonome.
Oriunde este nenecesară intervenția unei persoane în execuția unei acțiuni, spunem că avem automatizare.
Automatizarea va necesita în continuare controlul din partea unei persoane, dar partea de execuție nu.

În domeniul IT, termenul de automatizare este asociat de obicei cu noțiunea de \textbf{scripting}.
Acțiunea se mai numește ,,a scripta'', întrucât acțiunile automatizate sunt formate din comenzi și instrucțiuni care se trec într-un fișier numit script.
Acest script este apoi interpretat și rulat la un moment convenabil, având loc astfel automatizarea acțiunii.

Un script este, de obicei, asociat interfeței în linie de comandă; vorbim de scripturi shell, scripturi Python, scripturi Perl, scripturi Lua.
Există însă și moduri de automatizare pentru interfețe grafice sau web.
Există aplicații\footnote{\url{https://en.wikipedia.org/wiki/Graphical\_user\_interface\_testing}} care testează interfețe grafice automatizând plasarea cursorului de mouse și evenimente grafice precum click, selectare, glisare.
Similară este tehnologia RPA (\textit{Robotic Process Automation}) care permite automatizarea unei activități pe care o execută cineva folosind interfața grafică, precum emiterea unei facturi care să cuprindă anumite date, sau citirea unor date dintr-un document și completarea acestora într-un formular web.
În cele ce urmează ne vom concentra pe automatizarea în interfața în linia de comandă, modul uzual de automatizare în sistemele de operare și infrastructurile IT.
Este de reținut că există tehnologii și posibilități de automatizare și în interfața grafică.

Așa cum am spus mai sus, automatizarea în linia de comandă folosește un script care conține comenzi și instrucțiuni pentru executarea unei acțiuni.
Un conținut similar are însă și un program și poate apărea întrebarea cu ce diferă un program de un script.
Nu există o linie clară de demarcație între un script și un program.
În general, un program se dorește a fi eficient în consumarea resurselor și execută o acțiune bine determinată.
Pe de altă parte, un script este ceva creat și folosit rapid (\textit{quick and dirty}) și folosește programe deja existente pentru executarea unei acțiuni combinate, care leagă mai multe componente.
De exemplu, un program va fi acela care generează un fișier PDF\abbrev{PDF}{Portable Document Format} dintr-un fișier de tip \texttt{.doc}; un script este acela care parcurge o ierarhie ce conține fișiere \texttt{.doc}, selectează fișierele care sunt mai noi de o anumită dată, le convertește în fișiere PDF și apoi creează o arhivă cu fișierele PDF pe care o stochează într-un director dat.
Păstrând măsura, putem folosi interschimbabil noțiunea de script și program, preferând termenul de script pentru situațiile în care urmărim o formă de automatizare.

Într-un sistem informatic multe aplicații au interfață grafică, care nu poate fi automatizată într-un script.
Pentru automatizare acțiunilor corespunzătoare, avem două opțiuni:
\begin{enumerate}
 \item să folosim aplicații alternative, în interfață de comandă
 \item să folosim forme de rulare neinteractivă a acestora
\end{enumerate}

Din prima categorie, de exemplu, folosim utilitarele \cmd{wget} sau \cmd{curl} în locul unui browser web grafic precum Firefox sau Chrome.
Sau folosim utilitarul \cmd{bc} în locul unei aplicații grafice de calculator.

Din a doua categorie fac parte aplicațiile care oferă o formă de rulare numită \textit{headless}.
Forma headless înseamnă o aplicație cu funcționalități similare dar fără interfață grafică.
În general, aplicații headless sunt acele aplicații care nu au interfața grafică.
De exemplu, clientul Bittorrent Transmission are varianta headless \cmd{transmission-cli}.
Soluția de virtualizare VirtualBox are varianta headless \cmd{VBoxManage}, pentru controlul mașinilor virtuale.
Folosind aplicații alternative sau forme headless ale unor aplicații putem automatiza acțiunile aferente acestora.
Putem spune că, în privința automatizării rulării aplicațiilor, \textit{where there is a will, there is a way}.

În cele ce urmează vom prezenta modul în care putem crea scripturi pentru automatizarea de acțiuni în linia de comandă, cu accentul pe cea mai simplă și accesibilă formă: \textbf{shell scripting}.

\section{Perspective pentru scripting}
\label{sec:auto:how}

Am precizat mai sus că forma de automatizare în linia de comandă este folosind scripturi ce conțin comenzi și instrucțiuni.
Atunci când automatizăm acțiuni în linia de comandă folosim scripturi în două cazuri mari de utilizare:
\begin{enumerate}
  \item automatizarea unor acțiuni repetitive
  \item prelucrarea datelor
\end{enumerate}

În prima categorie intră acțiuni precum compilarea unei suite de aplicații, crearea de mai multe mașini virtuale sau copierea unor fișiere pe o platformă la distanță pentru actualizarea conținutului.
În general, aceste scripturi conțin comenzi sau programe pe care utilizatorul le-ar rula pe rând în linia de comandă.
Scriptul are avantajul de a agrega acele comenzi, iar utilizatorul doar rulează scriptul.

În a doua categorie intră acțiuni precum crearea unui raport sau a unei diagrame din datele existente într-o bază de date, dintr-un fișier jurnal sau din date afișate de o aplicație.
În general aceste scripturi conțin comenzi specifice pentru prelucrarea datelor (text), adică filtre de text, așa cum le-am denumit în \labelindexref{Capitolul}{chapter:cli}.

Putem spune că ambele cazuri automatizează acțiuni și că acele acțiuni sunt repetitive.
Cea mai importantă diferență între cele două este însă tipul de comenzi și utilitare pe care scripturile le folosesc: comenzi obișnuite, dar acum folosite agregat, în cazul automatizării de acțiuni repetitive, și filtre de text în cazul prelucrării datelor.
Atunci când automatizăm o acțiune, ne putem da seama de tipul de comenzi și utilitare pe care le vom folosi în funcție de cazul de utilizare al acțiunii.

Atunci când dezvoltăm un script pentru automatizare, este recomandat să urmărim idei de bună proiectare, pe care le descriem mai jos.

Există mai multe moduri de a face o acțiune.
De exemplu, pentru a afișa informații despre un proces putem folosi utilitarul \cmd{ps} sau putem urmări intrările din \file{/proc}.
Pentru a scrie informație într-un fișier putem folosi comanda \cmd{echo} și redirectare sau putem folosi utilitarul \cmd{dd}.
Și în cazul general, și când automatizăm, \textbf{alegem soluția cea mai adecvată pentru cazul respectiv de utilizare}.
De exemplu utilitarul \cmd{sed} face tot ce face utilitarul \cmd{tr} și în plus; cu toate acestea vom folosi utilitarul \cmd{tr} pentru cazuri de utilizare simple.
Nu există soluții universale, iar autorul scriptului va alege cele mai adecvate soluții în funcție de cazul de utilizare.

Când creăm un script folosim cât mai mult utilitare și scripturi existente.
\textbf{Nu reinventăm roata}.
Dacă o comandă face deja ce ne dorim, nu facem un script.
Dacă o comandă cu anumiți parametri face deja ceea ce ne dorim, nu folosim două sau trei comenzi.
De exemplu, dacă dorim să aflăm dimensiunea unui fișier, avem cele două variante din \labelindexref{Listing}{lst:auto:ls-vs-stat}.

\begin{screen}[caption={Afișarea informațiilor despre un fișier (ls vs stat)},label={lst:auto:ls-vs-stat}]
student@uso:~$ ls -l vm-actions-log.txt  | tr -s ' ' | cut -d ' ' -f 5
4827
student@uso:~$ stat -c "%s" vm-actions-log.txt
4827
\end{screen}

Vom prefera a doua variantă care este mai simplă și mai clară și mai puțin expusă la posibile erori de formatare ale utilitarului \cmd{ls}.
Pentru aceasta, este nevoie de o cunoaștere temeinică a comenzilor, utilitarelor și facilităților existente pentru a nu ajunge să reinventăm roata.

Scripturile sunt folosite pentru acțiuni de tipul \textit{quick and dirty}.
Scripturile nu urmăresc să fie eficiente din punct de vedere al consumului de resurse (rapide, memorie redusă) ci urmăresc să fie eficace (să ducă treaba la bun sfârșit) și să eficientizeze munca utilizatorului, lăsându-i timp pentru alte activități.
De aceea nu vom ajunge ca în scripturi să facem calcule numerice complexe sau acțiuni intens computaționale.
Scripturile pot automatiza apelarea de programe scrise special pentru acțiuni intens computaționale, dar nu vor ajunge ele să facă astfel de acțiuni.

În lumea Unix, \textbf{modul de transfer de informație este textul}, conținut ușor de citit de om.
Acest lucru permite stocarea informației în fișiere text și, la nevoie, editarea lor de către utilizator.
De aceea, scripturile, mai ales cele de prelucrare de date, este recomandat să folosească informații text: ușor de citit de utilizator, ușor de modificat.
Cele mai multe utilitare shell lucrează cu informații text, de exemplu filtrele de text.

În linia de comandă avem utilitare care raportează informații de la nivelul sistemului.
De exemplu, utilitarul \cmd{ps} raportează informații despre procese, iar utilitarul \cmd{ls} afișează informații despre fișiere.
Un rol oarecum similar îl au, respectiv, utilitarele \cmd{pgrep} și \cmd{stat}.
Ne punem problema când folosim pe fiecare.
Pentru aceasta clasificăm utilitarele în două:
\begin{enumerate}
  \item utilitare pentru afișarea de informații, pe ecran, pentru a fi accesibile utilizatorului
  \item utilitare care afișează informația în format prelucrabil, util pentru legarea de alte comenzi, adică pentru automatizare
\end{enumerate}

Utilitarele \cmd{ps} și \cmd{ls} fac parte din prima categorie.
La va folosi utilizatorul atunci dorește să afișeze informații despre fișiere și procese.
Dacă însă utilizatorul va avea nevoie de PID-ul anumitor procese sau dimensiunea anumitor fișiere, va fi complicată folosirea utilitarelor \cmd{ps} și \cmd{ls}; atunci se recomandă folosirea utilitarelor \cmd{pgrep} și \cmd{stat}, ca în \labelindexref{Listing}{lst:auto:print-vs-process-cmd}.
Observăm că folosirea \cmd{pgrep} este mai curată, extrăgând exact informația dorită: PID-urile proceselor cu numele \texttt{VBoxClient}.
Folosirea \cmd{ps} este mai anevoioasă, necesitând operații suplimentare de prelucrare a rezultatului.

\begin{screen}[caption={Utiltare pentru afișare și pentru prelucrare},label={lst:auto:print-vs-process-cmd}]
student@uso:~$ ps -ef | grep VBoxClient | tr -s ' ' | cut -d ' ' -f 2
1300
1301
1310
1311
1321
1322
1328
1329
10819
student@uso:~$ pgrep VBoxClient
1300
1301
1310
1311
1321
1322
1328
1329
\end{screen}

În secțiunile ce urmează vom prezenta informații despre crearea de scripturi shell și vom furniza exemple în care vom ține cont de principiile enunțate mai sus.

\section{Dezvoltarea scripturilor}
\label{sec:auto:dev}

Un script este un fișier care conține comenzi, instrucțiuni și construcții care execută o anumită acțiune.
Din punctul de vedere al conținutului, un script este similar unui program.
Din punctul de vedere al execuției, un script este interpretat.
Așa cum multe limbaje de programare sunt compilate și executate, un script este interpretat de un interpretor.
În cazul unui script shell, interpretorul este chiar shellul.
Alte interpretoare pot fi utilitarul \cmd{sed} sau \cmd{awk} sau un interpretor de limbaj Python, Perl, Ruby, Lua etc.

\subsection{,,Hello, World!'' în scripting și shebang}
\label{sec:auto:dev:hello}

Cel mai cunoscut exemplu în orice limbaj este un program care afișează mesajul ,,Hello, World!''.
Vom exemplifica cele mai simple scripturi cu același rezultat: afișarea mesajul ,,Hello, World!''.

Dacă am creat fișierul \file{hello.sh} având conținutul din \labelindexref{Listing}{lst:auto:hello} (liniile 1-4), atunci vom putea apela shellul Bash pentru interpretarea sa (liniile 5-6).
În mod similar, dacă vom crea fișierul \file{hello.py} având conținutul din \labelindexref{Listing}{lst:auto:hello} (liniile 7-10), atunci vom putea apela Python pentru interpretarea sa (liniile 11-12).
La fel ar fi cazul și cu scripturi Perl sau Ruby sau alte limbaje de scripting.

\begin{screen}[caption={Hello, World! în script shell},label={lst:auto:hello}]
student@uso:~/.../code/13-auto/hello$ cat hello.sh
#!/bin/bash

echo "Hello, World!"
student@uso:~/.../code/13-auto/hello$ bash hello.sh
Hello, World!
student@uso:~/.../code/13-auto/hello$ cat hello.py
#!/usr/bin/env python3

print("Hello, World!")
student@uso:~/.../code/13-auto/hello$ python hello.py
Hello, World!
\end{screen}

Pentru a simplifica rularea unui script putem să oferim fișierului permisiuni de execuție și să îl rulăm ca pe un program.
Pentru ca acest lucru să funcționeze, shellul va trebui să știe ce interpretor să folosească.
Pentru a stabili interpretorul, folosim o construcție numită shebang (\texttt{\#!}).
Această construcție este folosită la începutul scriptului pentru a indica interpretorul folosit.
Este folosită la începutul scripturilor \file{hello.sh} și \file{hello.py}, așa cum se vede în \labelindexref{Listing}{lst:auto:hello}, liniile 2, respectiv 8.

Dacă este prezentă construcția shebang (\texttt{\#!}) pentru a indica interpretorul folosit (\texttt{/bin/bash}, respectiv \texttt{/usr/bin/env python3}), atunci scripturile pot fi executate ca un program obișnuit, după ce le vom acorda permisiuni de execuție, ca în \labelindexref{Listing}{lst:auto:exec-perm}.

\begin{screen}[caption={Rularea directă a scripturilor},label={lst:auto:exec-perm}]
student@uso:~/.../code/13-auto/hello$ chmod a+x hello.sh
student@uso:~/.../code/13-auto/hello$ chmod a+x hello.py
student@uso:~/.../code/13-auto/hello$ ./hello.sh
Hello, World!
student@uso:~/.../code/13-auto/hello$ ./hello.py
Hello, World!
\end{screen}

În scripturile pe care le vom construi în continuare vom folosi mereu shebang.

\subsection{Conținutul unui script shell}
\label{sec:auto:dev:content}

În acest capitol ne vom concentra pe automatizare în
linia de comandă, adică pe scripturi, și, dintre acestea, pe cele mai simple și
rapide: scripturile shell.

Scripturile shell sunt scripturi interpretate direct de shell.
În mod uzual, acestea vor începe cu simbolul shebang și apoi calea către executabilul aferent shellului.
Pentru Bash, shellul predominant pe Linux, calea este \file{/bin/bash}, așa cum am văzut în exemplul din \labelindexref{Listing}{lst:auto:hello}.
Astfel că prima linie într-un script shell / Bash va fi \texttt{\#!/bin/bash}.

Informațiile pe care le vom furniza și exemplele de scripturi vor fi specifice shellului Bash.
Cu toate acestea, multe dintre funcționalități sunt prezente și în alte shelluri.
În general, cu modificări minore un script Bash poate ajunge să ruleze și pe un shell Zsh sau Tcsh.
Nu vom intra în aceste detalii în această carte, ne vom concentra pe shellul Bash; lăsăm utilizatorului interesat să folosească resursele puse la dispoziție pe Internet legate de scripting pentru shellul Zsh sau Tcsh.
În continuare, când folosim shell, vom înțelege implicit Bash, iar când vom folosi script shell, vom înțelege script Bash.

Un script shell conține în cea mai mare parte comenzi pe care utilizatorul le folosește în linia de comandă, agregate pentru o nouă funcționalitate.
Pe lângă aceste comenzi, un script conține construcții specifice shellului așa cum au fost descrise în \labelindexref{Secțiunea}{sec:cli:shell-func}: operatori, expandare, escaping.
Cu ajutorul acestora, un script combină comenzi pentru a obține noi funcționalități.

Pe lângă comenzile din linia de comandă, un script shell conține și comenzi interne (despre care am discutat în \labelindexref{Secțiunea}{sec:cli:shell:command-types}) folosite pentru controlul fluxului, funcții și parametri în linia de comandă.

\subsection{Depanarea unui script shell}
\label{sec:auto:dev:debug}

Întrucât este conceput să fie \textit{quick and dirty}, un script shell este de obicei mic, iar depanarea se face, în general, folosind \textit{print debugging}: rularea sa de mai multe ori și afișarea de mesaje folosind comanda \cmd{echo}.
Afișăm fie locuri în care a ajuns scriptul în cadrul execuției comenzilor, fie valorile unor variabile.

La nevoie, dacă sunt foarte multe informații afișate pe ecran, scriem informațiile într-un fișier pentru a le consulta ulterior.

Întrucât un script este format din comenzi de utilitare deja existente, putem rula pas cu pas fiecare comandă și apoi să vedem efectul ei.
În fapt, este recomandată rularea comenzilor dorite înainte de adăugarea lor într-un script pentru a ne asigura de funcționarea corectă.

Pentru depanare mai avansată, putem folosi comanda \cmd{set -x} așa cum vom descrise în \labelindexref{Secțiunea}{sec:auto:script-advanced:other}.

\subsection{Exemple de scripturi shell}
\label{sec:auto:dev:ex}

În această secțiune prezentăm câteva exemple practice de scripturi shell și ce funcționalități sunt folosite în fiecare.

\labelindexref{Listing}{lst:auto:system-update} prezintă un script simplu folosit pentru actualizarea sistemului.
Scriptul este folosit pentru a rula o singură comandă pentru a actualiza pachetele din sistem.
Scriptul folosește:
\begin{itemize}
  \item \texttt{\#!} (\textit{shebang}) pentru precizarea interpretorului (\texttt{/bin/bash})
  \item utilitarul \cmd{apt} pentru operare cu pachete
  \item înlănțuirea comenzilor (operatorul \texttt{\&\&}); am discutat în \labelindexref{Secțiunea}{sec:cli:shell-func:operators}
\end{itemize}

\begin{screen}[caption={Script: Actualizarea systemului},label={lst:auto:system-update}]
student@uso:~/uso-book/code/13-auto$ cat system-update
#!/bin/bash

sudo apt update && sudo apt -y upgrade && sudo apt -y dist-upgrade
\end{screen}

\labelindexref{Listing}{lst:auto:wiki-cat} prezintă folosirea scriptului \file{wiki-cat}.
Scriptul este folosit în laboratoarele didactice pentru a agrega secțiuni de tip wiki din fișiere diferite într-un singur fișier.
Există directoare pentru fiecare laborator (\texttt{01-fs}, \texttt{02-process}, \texttt{03-user}, \texttt{04-appdev}) care conțin secțiuni de wiki în fișiere diferite.
Scriptul realizează fișiere agregate pentru fiecare laborator: \file{01-fs.wiki}, \file{02-process.wiki}, \file{03-user.wiki}, \file{04-appdev.wiki}.
Scriptul folosește:
\begin{itemize}
  \item \texttt{\#!} (\textit{shebang}) pentru precizarea interpretorului (\texttt{/bin/bash})
  \item \texttt{for} pentru parcurgerea elementelor din directorul curent; variabila \texttt{i} reține fiecare fișier sau director; detalii în \labelindexref{Secțiunea}{sec:auto:script-func:flow-control:if}
  \item \texttt{*} (\textit{star}), metacaracter care expandează la toate intrările din directorul curent; am discutat în \labelindexref{Secțiunea}{sec:cli:shell-func:expansion:globbing}
  \item \texttt{if} și \texttt{test} pentru a verifica elementele de tipul director (\texttt{-d "\$i"}); detalii în \labelindexref{Secțiunea}{sec:auto:script-func:flow-control:if}
  \item expandarea opțiunilor (cu acolade, construcția \verb|{opt1,opt2,opt3}|); am discutat în \labelindexref{Secțiunea}{sec:cli:shell-func:expansion}
  \item redirectarea ieșirii (operatorul \texttt{$>$}); am discutat în \labelindexref{Secțiunea}{sec:cli:shell-func:operators}
  \item expandarea comenzii \texttt{basename} (construcția \verb|$(...)|); am discutat în \labelindexref{Secțiunea}{sec:cli:shell-func:expansion:commands}
\end{itemize}

\begin{screen}[caption={Script: Agregarea secțiunilor de tip wiki},label={lst:auto:wiki-cat}]
student@uso:~/.../code/13-auto/wiki-cat$ ls
01-fs  02-process  03-user  04-appdev  wiki-cat

student@uso:~/.../code/13-auto/wiki-cat$ cat wiki-cat
#!/bin/bash

for i in *; do
    if test -d "$i"; then
        cat "$i"/{concepts,demo,recap,basics,need-to-know,nice-to-know,get-a-life,summary}.wiki > $(basename "$i").wiki
    fi
done

student@uso:~/.../code/13-auto/wiki-cat$ ls 01-fs 02-process 03-user 04-appdev
01-fs:
basics.wiki  concepts.wiki  demo.wiki  get-a-life.wiki  need-to-know.wiki  nice-to-know.wiki  recap.wiki  setup.wiki  summary.wiki

02-process:
basics.wiki  concepts.wiki  demo.wiki  get-a-life.wiki  need-to-know.wiki  nice-to-know.wiki  recap.wiki  setup.wiki  summary.wiki

03-user:
basics.wiki  concepts.wiki  demo.wiki  get-a-life.wiki  need-to-know.wiki  nice-to-know.wiki  recap.wiki

04-appdev:
basics.wiki  concepts.wiki  demo.wiki  get-a-life.wiki  need-to-know.wiki  nice-to-know.wiki  recap.wiki

student@uso:~/.../code/13-auto/wiki-cat$ ./wiki-cat
cat: 03-user/summary.wiki: No such file or directory
cat: 04-appdev/summary.wiki: No such file or directory

student@uso:~/.../code/13-auto/wiki-cat$ ls
01-fs  01-fs.wiki  02-process  02-process.wiki  03-user  03-user.wiki  04-appdev  04-appdev.wiki  wiki-cat
\end{screen}

\labelindexref{Listing}{lst:auto:gen-passwd} prezintă folosirea scriptului \file{gen-passwd}.
Scriptul generează o parolă și afișează parola și formatul ei hash MD5.
Parolele sunt reținute în format hash din rațiuni de securitate, așa cum am precizat în \labelindexref{Secțiunea}{sec:sec:auth:password} și în \labelindexref{Secțiunea}{sec:user:password-db}.
Scriptul folosește:
\begin{itemize}
  \item \texttt{\#!} (\textit{shebang}) pentru precizarea interpretorului (\texttt{/bin/bash})
  \item expandarea comenzilor \texttt{pwgen} și \texttt{openssl} pentru a reține rezultatul lor (construcția \verb|$(...)|); am discutat în \labelindexref{Secțiunea}{sec:cli:shell-func:expansion:commands}
  \item variabilele \texttt{salt}, \texttt{password} și \texttt{crypt} pentru a reține informații generate din expandarea comenzilor
\end{itemize}

\begin{screen}[caption={Script: Generator de parole și hash},label={lst:auto:gen-passwd}]
student@uso:~/uso-book/code/13-auto$ cat gen-passwd
#!/bin/bash

salt=$(pwgen -N 1 -c 8)
password=$(pwgen -N 1 -c 10)
crypt=$(openssl passwd -1 -salt "$salt" "$password")

echo "password: $password"
echo "crypt: $crypt"

student@uso:~/uso-book/code/13-auto$ ./gen-passwd
password: fie3kac6Ta
crypt: $1$Boongoh8$RPo3/ckltzviC4TXgxcAJ0

student@uso:~/uso-book/code/13-auto$ ./gen-passwd
password: Gei8ieno2i
crypt: $1$Pei4Beak$zHBDNjWMj3ozZW/cV.rKj/
\end{screen}

\section{Funcționalități pentru scripturi shell}
\label{sec:auto:script-func}

Scripturile shell sunt secvențe de comenzi existente combinate pentru o funcționalitate.
Shellul oferă și funcționalități similare unui limbaj de programare precum variabile, comenzi de control al fluxului, funcții.
Aceste funcționalități ajută la crearea de scripturi mai complexe și care să țină cont de diferitele condiții în care se rulează un script.

\subsection{Variabile}
\label{sec:auto:script-func:vars}

Variabilele au un rol similar altor limbaje de programare.
Au un nume și o valoare.
În shell (și în scripturi shell) variabilele nu au tip și interpretarea lor depinde de modul de utilizare.
O variabilă cu valoarea \texttt{12} poate fi număr sau doar un simplu string depinzând de cazul de folosire.
Exemple de inițializare și folosire a unei variabile sunt prezentate în \labelindexref{Listing}{lst:auto:use-vars}.
Variabila \texttt{f} reține numele fișierului \file{/etc/passwd}.
Pentru a folosi valoarea variabilei \texttt{f}, prefixăm cu simbolul \texttt{\$} (\textit{dollar}).
Este recomandat să folosim valorile variabilelor între ghilimele, pentru escaping, așa cum am prezentat în \labelindexref{Secțiunea}{sec:cli:shell-func:escaping}.
La fel, variabila \texttt{num} reține numărul de linii din fișierul reținut de variabila \texttt{f}, folosind expandarea comenzii \cmd{wc}.

\begin{screen}[caption={Folosirea variabilelor în shell},label={lst:auto:use-vars}]
student@uso:~/uso-book/code/13-auto$ f=/etc/passwd
student@uso:~/uso-book/code/13-auto$ wc -l < "$f"
46
student@uso:~/uso-book/code/13-auto$ num=$(wc -l < "$f")
student@uso:~/uso-book/code/13-auto$ echo "$num"
46
\end{screen}

Detalii despre variabile și folosirea lor în shell am prezentat în \labelindexref{Secțiunea}{sec:cli:shell-func:vars}.

Variabilele sunt utile în scripturi shell pentru a reține o valoare.
Acea valoare ar putea fi folosită și direct, dar e mai lizibil și mai ușor de înțeles scriptul când folosim un nume de variabilă.
Este cazul scriptului \file{gen-passwd} prezentat în \labelindexref{Listing}{lst:auto:gen-passwd}.
Altfel, variabilele sunt folosite implicit în comenzi / construcții shell, precum \cmd{for}, așa cum am folosit variabila \texttt{i} în scriptul \file{wiki-cat} din \labelindexref{Listing}{lst:auto:wiki-cat}.
Adesea, vom folosi variabile pentru menține o informație comună mai multor comenzi, similar unui limbaj de programare.
De exemplu, variabila \texttt{i} din scriptul \file{wiki-cat} din \labelindexref{Listing}{lst:auto:wiki-cat} este folosită împreună cu \cmd{for}, cu \cmd{if} / \cmd{test}, cu \cmd{cat} și cu \cmd{basename}.

Valoarea unei variabile poate fi inițializată și folosită prin operația simplă de atribuire, respectiv prin prefixarea ei cu simbolul \texttt{\$} (\textit{dolar}).
Pe lângă acestea, inițializarea și folosirea unei variabile poate fi realizată cu diferitele forme de expandare (aritmetică, a comenzilor, parametrică, cu opțiuni) pe care le-am prezentat în \labelindexref{Secțiunea}{sec:cli:shell-func:expansion} (cu exemple de folosire în \labelindexref{Listing}{lst:cli:expansion}).

Un tip special de variabile sunt parametrii unui script shell.
Un shell poate primi unul sau mai mulți parametri care pot afecta funcționarea shellului.
În \labelindexref{Listing}{lst:auto:wiki-cat-param} avem o actualizare a scriptului \file{wiki-cat} din \labelindexref{Listing}{lst:auto:wiki-cat}.
Scriptul \file{wiki-cat-param} primește un parametru: numele directorului de laborator ale cărui secțiuni wiki să fie unificate.
Dacă scriptul este rulat fără argumente, se afișează un mesaj de utilizare corectă.
La o rulare corectă, după transmiterea numelui unui director de laborator ca parametru (\texttt{01-fs/}), scriptul generează fișierul agregat corespunzător (\file{01-fs.wiki}).
Scriptul \file{wiki-cat-param} folosește construcțiile scriptului \file{wiki-cat} și parametri de script:
\begin{itemize}
  \item \texttt{\$\#} (împreună cu \cmd{if}) pentru numărul de parametri ai scriptului
  \item \texttt{\$0} pentru afișarea numelui scriptului atunci când nu se transmit numărul corect de parametri
  \item \texttt{\$1} pentru primul parametru al scriptului: director al cărui conținut va fi agregat
\end{itemize}
În script am fi putut să nu folosim variabila \texttt{lab} ci construcția \texttt{\$1} (primul parametru al scriptului).
Am preferat să definim o nouă variabilă pentru lizibilitate, fiind mai clar ce înseamnă construcția \texttt{\$lab} în loc de construcția \texttt{\$1}.

\begin{screen}[caption={Script: Agregarea secțiunilor de tip wiki (cu parametru)},label={lst:auto:wiki-cat-param}]
student@uso:~/.../code/13-auto/wiki-cat$ ls
01-fs  02-process  03-user  04-appdev  wiki-cat  wiki-cat-param

student@uso:~/.../code/13-auto/wiki-cat$ cat wiki-cat-param
#!/bin/bash

if test $# -ne 1; then
    echo "Usage: $0 <lab-folder>" 1>&2
    exit 1
fi

lab="$1"
if test -d "$lab"; then
    cat "$lab"/{concepts,demo,recap,basics,need-to-know,nice-to-know,get-a-life,summary}.wiki > $(basename "$lab").wiki
fi

student@uso:~/.../code/13-auto/wiki-cat$ ./wiki-cat-param
Usage: ./wiki-cat-param <lab-folder>

student@uso:~/.../code/13-auto/wiki-cat$ ./wiki-cat-param 01-fs/

student@uso:~/.../code/13-auto/wiki-cat$ ls
01-fs  01-fs.wiki  02-process  03-user  04-appdev  wiki-cat  wiki-cat-param
\end{screen}

Construcțiile aferente parametrilor unui script shell sunt prezentate în \labelindexref{Tabelul}{tab:auto:script-params}.
Construcțiile \texttt{\$*} și \texttt{\$@} au, în aparență același efect.
Diferență constă în escaparea celor două construcții prin ghilimele.
Astfel, construcția \texttt{\$*} se va expanda la o singură valoare care conține spații între parametrii scriptului, în vreme ce construcția \texttt{\$@} se va expanda la o listă de valori escapate fiecare prin ghilimele.

\begin{table}[!htb]
  \caption{Construcții pentru parametrii unui script shell}
  \begin{center}
    \begin{tabular}{ p{0.25\textwidth} p{0.65\textwidth} }
      \toprule
        \textbf{Nume construcție} &
        \textbf{Rol / Efect} \\
      \midrule
        \texttt{\$\#} &
        numărul de parametri ai scriptului (echivalentul \texttt{argc} în C) \\

        \texttt{\$0} &
        numele scriptului (echivalentul \texttt{argv$[$0$]$} în C) \\

        \texttt{\$1}, \texttt{\$2}, \ldots &
        primul, al doilea, \ldots{} parametru\\

        \texttt{\$*} &
        toți parametrii scriptului\\

        \texttt{\$@} &
        toți parametrii scriptului \\

      \bottomrule
    \end{tabular}
    \label{tab:auto:script-params}
  \end{center}
\end{table}

În lucrul cu parametrii, shellul oferă comanda \cmd{shift}.
Comanda \cmd{shift} este folosită pentru a deplasa parametri o poziție la stânga.
După o folosire a comenzii \cmd{shift}, construcția \texttt{\$1} va avea acum valoarea celui de-al doilea parametru, \texttt{\$2} valoarea celui de-al treilea parametru ș.a.m.d.

\subsection{Controlul fluxului}
\label{sec:auto:script-func:flow-control}

Așa cum am văzut și în exemplele de mai sus, shellul oferă construcții de
controlul fluxului: comenzi pentru decizii condiționale (\cmd{if}) și comenzi de
ciclare (\cmd{for}).

De avut în vedere că pot exista scripturi care nu folosesc
aceste comenzi; un script nu este un program în sine ci o înlănțuire de
funcționalități existente care pot să nu necesite comenzi de control al
fluxului, cum este cazul scripturilor \file{system-update} (din \labelindexref{Listing}{lst:auto:system-update}) sau \file{gen-passwd} (din \labelindexref{Listing}{lst:auto:gen-passwd}).

\subsubsection{if}
\label{sec:auto:script-func:flow-control:if}

Comanda \cmd{if} este folosită pentru controlul fluxului.
Comanda primește ca argument o altă comandă și interpretează codul de ieșire (\textit{exit code}) al acesteia, în funcție de care ia o decizie.

Sintaxa comenzii \cmd{if} este indicată în \labelindexref{Listing}{lst:auto:if-syntax}.
Secțiunile care cuprind \texttt{elif} (liniile \texttt{6-9}) și \texttt{else} (liniile \texttt{10-13}) pot lipsi.
Uzual, \texttt{then} apare pe aceeași linie cu \cmd{if}, separată cu operatorul de secvențiere (\texttt{;}), la modul \texttt{if command\_to\_check1; then}, așa cum apare și în \labelindexref{Listing}{lst:auto:wiki-cat} și \labelindexref{Listing}{lst:auto:wiki-cat-param}.
Comenzile \texttt{command\_to\_check1} și \texttt{command\_to\_check2} vor fi executate și va fi verificat codul de ieșire.
Dacă acest cod de ieșire înseamnă rulare cu succes, atunci se execută \texttt{command\_to\_execute1}, \texttt{command\_to\_execute2} și așa mai departe.

\begin{screen}[caption={Sintaxa comenzii if},label={lst:auto:if-syntax}]
if command_to_check1
then
    command_to_execute1
    command_to_execute2
    ...
elif command_to_check2
    command_to_execute3
    command_to_execute4
    ...
else
    command_to_execute5
    command_to_execute6
    ...
fi
\end{screen}

Cel mai adesea, comanda primită de comanda \cmd{if}, comandă al cărei cod de ieșire va fi verificat, este comanda \cmd{test}, ca în \labelindexref{Listing}{lst:auto:wiki-cat} și \labelindexref{Listing}{lst:auto:wiki-cat-param}.
Comanda \cmd{test} permite verificări de șiruri de caractere, numere, fișiere și întoarce succes / insucces dacă verificarea e adevărată sau nu.
De exemplu, în \labelindexref{Listing}{lst:auto:wiki-cat-param}, în linia 7 comanda \cmd{test} face o verificare a numărului de parametri ai scriptului, iar în linia 13 comanda \cmd{test} verifică dacă valoarea variabilei \texttt{lab} este un director.

Un cod de ieșire egal cu \texttt{0} înseamnă succes.
Un cod de ieșire diferit de \texttt{0} înseamnă insucces.
Codul de ieșire al unei comenzi este reținut de construcția \texttt{\$?}, așa cum am precizat în \labelindexref{Secțiunea}{sec:cli:shell-func:vars}.
Astfel că cele trei construcții din \labelindexref{Listing}{lst:auto:if-test-exit-code} sunt echivalente.

\begin{screen}[caption={Verificarea codului de ieșire al unei comenzi},label={lst:auto:if-test-exit-code}]
grep "ana" /etc/passwd > /dev/null && echo "found"

if grep "ana" /etc/passwd > /dev/null; then
    echo "found"
fi

grep "ana" /etc/passwd > /dev/null
if test $? -eq 0; then
    echo "found"
fi
\end{screen}

Folosim \cmd{if} în locul operatorilor de secvențiere condiționată (\texttt{\&\&}, \texttt{$\textbar\textbar$}) atunci când executăm mai multe comenzi sau pentru lizibilitate.
Folosim \cmd{if} cu \cmd{test} și verificarea codului de ieșire \texttt{\$?} din rațiuni de lizibilitate; dar e discutabil și probabil chestiune de gust care dintre variantele (liniile \texttt{3-5}, respectiv \texttt{7-10}) este preferabilă.

Comanda \cmd{test} primește ca argument o specificare de condiție care poate avea diferite forme.
Condiția poate fi comparație între numere, între șiruri de caractere, verificarea tipului unui fișier și altele.
\labelindexref{Tabelul}{tab:auto:test} conține o selecție a condițiilor folosite de comanda \cmd{test}.
O listă completă găsiți în pagina de manual a comenzii \cmd{test}.

\begin{table}[!htb]
  \caption{Condiții folosite de comanda test}
  \begin{center}
    \renewcommand*\arraystretch{1.3}  % Altfel liniile tabelului sunt imposibil de diferențiat.
    \begin{tabular}{ p{0.5\textwidth} p{0.45\textwidth} }
      \toprule
        \textbf{Exemplu condiție} &
        \textbf{Rol / Efect} \\
      \midrule
        \cmd{test -f /etc/passwd} &
        verifică dacă \file{/etc/passwd} este fișier \\

        \cmd{test -d "\$d"} &
        verifică dacă valoarea variabilei \texttt{d} este director \\

        \cmd{test -x my\_script} &
        verifică dacă \file{my\_script} are permisiuni de execuție \\

        \cmd{test "\$\#" -gt 2} &
        verifică dacă numărul de parametri ai scriptului este mai mare (strict) decât 2 \\

        \cmd{test "\$v" -eq 10} &
        verifică dacă valoarea variabilei \texttt{v} este egală cu 10 \\

        \cmd{test "\$v" -eq 10} &
        verifică dacă valoarea variabilei \texttt{v} este egală cu 10 \\

        \cmd{test "\$1" = "/etc/passwd"} &
        verifică dacă primul parametru al scriptului are ca valoare șirul \texttt{/etc/passwd} \\

        \cmd{test -z "\$store"} &
        verifică dacă variabila \texttt{store} nu conține nici o informație (este un șir vid) \\

        \cmd{test !\@ -z "\$store"} &  % \@ este necesar după !, altfel LaTeX consideră că se termină o frază și bugetează spațiu în plus.
        verifică opusul condiției de mai sus \\

      \bottomrule
    \end{tabular}
    \label{tab:auto:test}
  \end{center}
\end{table}

Comanda \cmd{test} are ca echivalent comanda \cmd{$[$} (paranteză dreaptă deschisă).
Cele două comenzi sunt echivalente: orice verificare se face cu \cmd{test}, se poate face și cu \cmd{$[$}, ca în \labelindexref{Listing}{lst:auto:test-bracket}.
Diferența este că, în cazul comenzii \cmd{$[$}, la sfârșitul condiției de verificare, trebuie adăugată construcția \texttt{$]$}.
Din cauză că această construcție este adesea omisă sau este ,,lipită'' de condiția de verificare, în loc să fie separată prin spațiu, recomandăm folosirea comenzii \cmd{test} în defavoarea comenzii \cmd{$[$}.

\begin{screen}[caption={Comanda test și comanda {]}},label={lst:auto:test-bracket}]
if test 3 -eq 3; then
    echo "equal"
fi

if [ 3 -eq 3 ]; then
    echo "equal"
fi

if test -f /etc/passwd; then
  echo "exists"
fi

if [ -f /etc/passwd ]; then
  echo "exists"
fi
\end{screen}

O altă comandă pentru verificarea condițiilor este comanda \cmd{$[[$}.
Comanda \cmd{$[[$} diferă în două moduri de comenzile \cmd{test} și \cmd{$[$}:
\begin{itemize}
  \item este o comandă internă shellului (cum am discutat în \labelindexref{Secțiunea}{sec:cli:shell:command-types}), în vreme ce \cmd{test} și \cmd{$[$} sunt comenzi externe
  \item este proprie shellului Bash; este posibil să se găsească și în alte shelluri, dar nu ne putem baza pe acest lucru
\end{itemize}

Comanda \cmd{$[[$} poate fi considerată o extensie a comenzilor \cmd{test} și \cmd{$[$}.
Aceleași verificări realizate cu \cmd{test} și \cmd{$[$} pot fi realizate și cu \cmd{$[[$}.
Pe lângă acestea, \cmd{$[[$} permite verificări suplimentare, ca cele prezentate în \labelindexref{Listing}{lst:auto:shell-builtin-bracket}:
\begin{itemize}
  \item liniile \texttt{1-2}: verificare de condiții compuse, folosind operatorul \texttt{\&\&} (ȘI logic - \textit{logical AND})
  \item liniile \texttt{4-5}: comparația unui șir cu o construcție de tip pattern / globbing (prezentate în \labelindexref{Secțiunea}{sec:cli:shell-func:expansion:globbing}), folosind operatorul \texttt{==}
  \item liniile \texttt{7-8}: comparația unui șir cu o construcție de tip expresie regulată (prezentate în \labelindexref{Secțiunea}{sec:cli:regex}), folosind operatorul \texttt{=$\sim$}
\end{itemize}

\begin{screen}[caption={Folosirea comenzii interne {[[}},label={lst:auto:shell-builtin-bracket}]
student@uso:~/.../code/13-auto$ if [[ 3 -eq 3 && 9 -eq 9 ]]; then echo "equal"; fi
equal

student@uso:~/.../code/13-auto/wiki-cat$ if [[ /etc/passwd == *c/p* ]]; then echo "match"; fi
match

student@uso:~/.../code/13-auto/wiki-cat$ if [[ /etc/passwd =~ /.*/.*d ]]; then echo "match"; fi
match
\end{screen}

În general recomandăm folosirea \cmd{test} pentru condiții uzuale și întotdeauna atunci când scriptul se dorește portabil între mai multe shelluri, așa cum vom discuta în \labelindexref{Secțiunea}{sec:auto:script-advanced:portable}.
Folosim \cmd{$[[$} pentru situații de potrivire cu expresii regulate sau patternuri.
Dacă avem nevoie de potrivire cu expresii regulate sau patternuri și este nevoie de portabilitatea scriptului, putem crea construcții echivalente folosind comenzi de lucru cu expresii regulate precum \cmd{grep}.

\subsubsection{case}
\label{sec:auto:script-func:flow-control:case}

Comanda \cmd{case} este echivalentul instrucțiunii \texttt{switch} din C; are rolul de a potrivi o variabilă la mai multe variante și de a executa comenzile echivalente.
Potrivirea se face folosind globbing / patternuri.
La fel ca instrucțiunea \texttt{switch} în C, comanda \cmd{case} este mai puțin întâlnită în scripturi.

Sintaxa comenzii \cmd{case} este indicată în \labelindexref{Listing}{lst:auto:case-syntax}.
Valoarea variabilei \texttt{var} este comparată cu diferite patternuri, precum \texttt{abc*} sau \texttt{*$[$de$]$fg*}.
În cazul fiecărei potriviri, se execută comenzile corespunzătoare.
De avut în vedere construcția \texttt{;;} pentru închiderea unei opțiuni \cmd{case}.
Construcția \texttt{*} definește echivalentul etichetei \texttt{default} din programarea în C: se potrivește cu orice șir.
În mod obișnuit, se plasează ca ultima opțiune pentru \cmd{case}, iar comenzile corespunzătoare sunt executate dacă nu s-au potrivit cu nici o altă opțiune.

\begin{screen}[caption={Sintaxa comenzii case},label={lst:auto:case-syntax}]
case var in
    abc*)
        command1
        command2
        ...
        ;;
    *[de]fg*)
        command3
        command4
        ...
        ;;
    *)
        command5
        command6
        ...
        ;;
esac
\end{screen}

Un exemplu de utilizare a comenzii \cmd{case} este în fișierul de startup \file{/etc/init.d/ssh}, prezentat în \labelindexref{Listing}{lst:auto:case-example}.
În cadrul fișierului, primul parametru al scriptului (\texttt{\$1}) este comanda de executat a scriptului cu valori precum \texttt{start}, \texttt{stop}, \texttt{reload}.
În cazul fiecărei valori a parametrului se execută anumite comenzi.
În cazul implicit (opțiunea \textit{default}: \texttt{*}) se afișează un mesaj legat de modul de utilizare a scriptului.

\begin{screen}[caption={Exemplu de folosire case: /etc/init.d/ssh},label={lst:auto:case-example}]
student@uso:~/uso-book/code/13-auto$ cat /etc/init.d/ssh
[...]
case "$1" in
  start)
	check_privsep_dir
	check_for_no_start
        [...]
	;;

  stop)
	log_daemon_msg "Stopping OpenBSD Secure Shell server" "sshd" || true
        [...]
	;;

  reload|force-reload)
	check_for_no_start
	check_config
        [...]
	;;

  [...]
  *)
	log_action_msg "Usage: /etc/init.d/ssh {start|stop|reload|force-reload|restart|try-restart|status}" || true
	exit 1
esac

exit 0
\end{screen}

\subsubsection{for}
\label{sec:auto:script-func:flow-control:for}

Atunci când dorim să trecem, pe rând, printr-o listă de valori posibile pentru o variabilă folosim comanda \cmd{for}.
O formă uzuală de folosire a comenzii for este pentru a trece printr-o listă de fișiere, cum este linia \texttt{7} din \labelindexref{Listing}{lst:auto:wiki-cat}.
Folosind construcția \texttt{for i in *}, variabila \texttt{i} primește, pe rând, ca valoare, numele fiecărei intrări din directorul curent.

Dacă dorim să folosim comanda \cmd{for} pentru a parcurge o listă de indecși, similar instrucțiunii \texttt{for} din C, folosim una dintre construcțiile din \labelindexref{Listing}{lst:auto:for-number}.
Prima construcție (linia \texttt{1}) este o formă de expandare aritmetică.
A doua construcție (linia \texttt{4}) folosește expandarea comenzii \cmd{seq} pentru a genera o listă de numere care va fi parcursă de variabila \texttt{i}.
A treia construcție (linia \texttt{7}) folosește o formă de expandare cu acolade (\textit{brace expansion}) pentru a genera lista de numere.
Folosirea \cmd{for} pentru a parcurge o listă de indecși este mai puțin frecventă în scripturi shell față de alte scenarii.
În general operațiile numerice / aritmetice, deși prezente în scripturi shell, sunt mai adecvate unor limbaje de programe complete, precum Python, Perl sau C.

\begin{screen}[caption={Parcurgere numerică folosind for},label={lst:auto:for-number}]
student@uso:~/uso-book/code/13-auto$ for ((i=1; i<=10; i++)); do echo "$i"; done
[...]

student@uso:~/uso-book/code/13-auto$ for i in $(seq 1 10); do echo "$i"; done
[...]

student@uso:~/uso-book/code/13-auto$ for i in {1..10}; do echo "$i"; done
[...]
\end{screen}

Un mod de folosire a comenzii \cmd{for} este trecerea prin parametrii unui script, ca în \labelindexref{Listing}{lst:auto:for-param}.
În linia \texttt{9} am folosit \cmd{for} pentru a parcurge parametrii scriptului, indicați de construcția \texttt{\$*}.

\begin{screen}[caption={Parcurgerea parametrilor unui script},label={lst:auto:for-param}]
student@uso:~/uso-book/code/13-auto$ cat show-ids
#!/bin/bash

if test $# -eq 0; then
    echo "Usage: $0 username1 [username2 ...]" 1>&2
    exit 1
fi

for i in $*; do
    echo -n "$i: "
    id -u "$i"
done

student@uso:~/uso-book/code/13-auto$ ./show-ids
Usage: ./show-ids username1 [username2 ...]

student@uso:~/uso-book/code/13-auto$ ./show-ids student root
student: 1000
root: 0
\end{screen}

În general, comanda \cmd{for} îmbracă una dintre formele din \labelindexref{Listing}{lst:auto:for-syntax}.
În prima formă, \texttt{expression} este cel mai adesea o expresie aritmetică, precum cea din linia \texttt{1} din \labelindexref{Listing}{lst:auto:for-number}.
În a doua formă, variabila \texttt{var} primește ca valoare, pe rând, elementele din lista \texttt{list}.
Lista \texttt{list} poate fi definită explicit, prin scrierea de elemente separate prin spațiu, sau implicit, prin expandări care conduc la obținerea unei liste, cum am prezentat în exemplele de mai sus.

\begin{screen}[caption={Sintaxa comenzii for},label={lst:auto:for-syntax}]
for ((expression)); do
    command_to_execute1
    command_to_execute2
    ...
done

for var in list; do
    command_to_execute1
    command_to_execute2
    ...
done
\end{screen}

De avut în vedere că folosirea \cmd{for} poate fi problematică sau ineficientă.
Folosirea \cmd{for} este problematică atunci când există spații în elementele listei.
În acel caz, în loc să fie preluate elementele efective, vor fi preluate părțile din fiecare element separate cu spații.
Vom discuta mijloace de rezolvare a spațiilor în listă în \labelindexref{Secțiunea}{sec:auto:script-advanced:robust}.

Folosirea \cmd{for} poate fi ineficientă când sunt alte forme de a folosi lista, fără a trece prin \cmd{for}, cum este cazul în \labelindexref{Listing}{lst:auto:for-inefficient}.
În primul exemplu, rulăm comanda \cmd{for} și creăm mai multe procese \cmd{stat}, câte unul pentru fiecare fișier.
În a doua variantă pur și simplu trimitem ca argument fișierele dorite comenzii \cmd{stat}.
Prima variantă rămâne, însă, utilă pentru cazul în care dorim să executăm mai multe comenzi care folosesc un element al unei liste (intrare din directorul curent).
În cazul nostru, am folosit doar comanda \cmd{stat} pentru fiecare element al listei.
Dacă am fi dorit să folosim mai multe comenzi, atunci nu am fi putut folosi a doua variantă.

\begin{screen}[caption={Alternativă la folosirea for},label={lst:auto:for-inefficient}]
student@uso:~/uso-book/code/13-auto$ for i in *; do stat -c "%n,%s" "$i"; done
gen-passwd,168
hello,4096
show-ids,158
system-update,80
wiki-cat,4096

student@uso:~/uso-book/code/13-auto$ stat -c "%n,%s" *
gen-passwd,168
hello,4096
show-ids,158
system-update,80
wiki-cat,4096
\end{screen}

Ca de obicei, în shell scripting, există mai multe moduri de a realiza un lucru.
Ideal, cumulat cu experiența, vom prefera cea mai bună opțiune pentru o situație dată: opțiunea cea mai robustă, în primul rând, și cea mai eficientă, în al doilea rând.
Vom discuta aceste aspecte în \labelindexref{Secțiunea}{sec:auto:script-advanced}.

\subsubsection{while}
\label{sec:auto:script-func:flow-control:while}

Similar comenzii \cmd{for} și a instrucțiunii \texttt{while} din C, comanda \cmd{while} poate fi folosită pentru a bucla în așteptarea unei condiții, ca în \labelindexref{Listing}{lst:auto:while-numeric}.
În a doua formă, am folosit expandare aritmetică pentru compararea valorii variabilei \texttt{i} cu \texttt{10}.

\begin{screen}[caption={Folosirea numerică a comenzii while},label={lst:auto:while-numeric}]
student@uso:~/uso-book/code/13-auto$ i=1; while test $i -le 10; do echo "$i"; ((i++)); done
[...]
student@uso:~/uso-book/code/13-auto$ i=1; while ((i <= 10)); do echo "$i"; ((i++)); done
[...]
\end{screen}

La fel ca în cazul comenzii \cmd{for}, forma de parcurgere numerică a comenzii \texttt{while} este mai puțin frecventă.
Astfel de operații numerice sunt preferate unor limbaje de programare precum Python, Perl sau C.
Frecvent, comanda \cmd{while} este folosită pentru a face parsing la fișiere text structurate tabelar (linii și coloane), folosind comanda \cmd{read}, ca în scriptul \file{extract-10} din \labelindexref{Listing}{lst:auto:extract-10}.
Scriptul afișează numele și grupa studenților care au nota finală 10.
În script am folosit comanda \cmd{while} pentru a parcurge fiecare linie din fișierul \file{students-surnames-only.txt}; fișierul este redirectat către comanda \cmd{while}.
Parsarea fiecărui element din linie este realizat de comanda \cmd{read}; fiecare element este reținut în câte o variabilă \texttt{name}, \texttt{group}, \texttt{final\_grade}, \ldots.

\begin{screen}[caption={Script: Extragerea studenților cu nota 10},label={lst:auto:extract-10}]
student@uso:~/.../code/13-auto/students$ ls
extract-10  extract-10-csv  extract-10-tab  students.csv  students-surnames-only.txt  students.txt

student@uso:~/.../code/13-auto/students$ cat extract-10
#!/bin/bash

while read name group final_grade test_grade practical_grade; do
    if test "$final_grade" -eq 10; then
        echo "$name,$group"
    fi
done < students-surnames-only.txt

student@uso:~/.../code/13-auto/students$ ./extract-10
GEORGIU,311CC
MUȘATESCU,311CC
RADU,311CC
GONDOȘ,312CC
NECULAI,313CC
MUHCINĂ,313CC
GRAMA,314CC
VĂSII,314CC
RĂDOI,314CC
NEACȘU,314CC
BARBU,315CC
MĂNOIU,315CC
ȘTIRBĂȚ,315CC
\end{screen}

Sintaxa uzuală pentru \cmd{while}, în combinație cu \cmd{read} este indicată în \labelindexref{Listing}{lst:auto:while-syntax}.
În general intrarea este dată de la un fișier (prin redirectare) de la o altă comandă (prin înlănțuire cu operatorul \textit{pipe} - \texttt{$\textbar$}).

\begin{screen}[caption={Sintaxa comenzii while (cu read)},label={lst:auto:while-syntax}]
while read field1 field2 ...; do
    command1
    command2
done < file_to_redirect

command_to_redirect | while read field1 field2 ...; do
    command1
    command2
done
\end{screen}

Pentru a face parsing, comanda \cmd{read} folosește variabila shell \texttt{IFS}\abbrev{IFS}{Input Field Separator} (\textit{Input Field Separator}).
În mod implicit, variabila \texttt{IFS} este inițializată la caracterele albe (spațiu, Tab, linie nouă).
Dacă dorim să folosim alt separator pentru parsare, vom actualiza variabila IFS ca în scripturile \file{extract-10-tab} și \file{extract-10-csv} din \labelindexref{Listing}{lst:auto:extract-10-tab}, respectiv \labelindexref{Listing}{lst:auto:extract-10-csv}.
\labelindexref{Listing}{lst:auto:extract-10-tab} folosește Tab pe post de separator și fișierul de intrare \texttt{students.txt}, ale cărui elemente sunt separate de Tab în fiecare linie.
\labelindexref{Listing}{lst:auto:extract-10-csv} folosește virgulă (\texttt{,}) pe post de separator și fișierul de intrare \texttt{students.csv} (de tip CSV - \textit{Comma Separated Values}), ale cărui elemente sunt separate de virgulă pe fiecare linie.

\begin{screen}[caption={Script: Extragerea studenților cu nota 10 (Tab ca separator)},label={lst:auto:extract-10-tab}]
student@uso:~/.../code/13-auto/students$ cat extract-10-tab
#!/bin/bash

IFS=$'\t'
while read name group final_grade test_grade practical_grade; do
    if test "$final_grade" -eq 10; then
        echo "$name,$group"
    fi
done < students.txt

student@uso:~/.../code/13-auto/students$ ./extract-10-tab
GEORGIU V. Alexandra-Maria,311CC
MUȘATESCU V. Alexandru-Petrișor,311CC
RADU L. Alina,311CC
GONDOȘ I. Gabriel,312CC
NECULAI M. Alexandru-George,313CC
MUHCINĂ V. Silviu-Mihai,313CC
GRAMA C. Serban-Alexandru,314CC
VĂSII I. Bogdan-Cristian,314CC
RĂDOI D. Ghennadi,314CC
NEACȘU C. Florin-Mărgărit,314CC
BARBU I. Ștefan,315CC
MĂNOIU I. Ioana-Veronica,315CC
ȘTIRBĂȚ A. Steliana,315CC
\end{screen}

\begin{screen}[caption={Script: Extragerea studenților cu nota 10 (CSV)},label={lst:auto:extract-10-csv}]
student@uso:~/.../code/13-auto/students$ cat extract-10-csv
#!/bin/bash

IFS=','
while read name group final_grade test_grade practical_grade; do
    if test "$final_grade" -eq 10; then
        echo "$name,$group"
    fi
done < students.csv

student@uso:~/.../code/13-auto/students$ ./extract-10-csv
GEORGIU V. Alexandra-Maria,311CC
MUȘATESCU V. Alexandru-Petrișor,311CC
RADU L. Alina,311CC
GONDOȘ I. Gabriel,312CC
NECULAI M. Alexandru-George,313CC
MUHCINĂ V. Silviu-Mihai,313CC
GRAMA C. Serban-Alexandru,314CC
VĂSII I. Bogdan-Cristian,314CC
RĂDOI D. Ghennadi,314CC
NEACȘU C. Florin-Mărgărit,314CC
BARBU I. Ștefan,315CC
MĂNOIU I. Ioana-Veronica,315CC
ȘTIRBĂȚ A. Steliana,315CC
\end{screen}

Prin inițializarea variabilei \texttt{IFS} la un separator dat (Tab sau virgulă) am putut să afișăm numele complet al studenților, nu doar numele de familie, cum era cazul în scriptul \file{extract-10}.
În scriptul \file{extract-10} am folosit o versiune modificată a fișierului de intrare, în care prima coloană a fișierului conține doar numele de familie.
Dacă ar fi conținut și inițiala tatălui și prenumele, atunci acestea ar fi fost câmpuri parsate de comanda \cmd{read} la folosirea implicită a separatorului \texttt{IFS} (care conține caractere albe: spațiu, Tab, linie nouă); caz în care valorile câmpurilor \texttt{group}, \texttt{final\_grade} și altele ar fi fost necorespunzătoare.

Așa cum am văzut în exemplele de mai sus, construcția \cmd{while read} este utilă atunci când dorim să prelucrăm date în format tabelar, putând rula comenzi pe câmpurile parsate din fiecare linie de la intrare.
Putem privi construcția \cmd{while read} ca o comandă \cmd{cut} extinsă.
\cmd{cut} doar permite extragerea uneia sau mai multor câmpuri / coloane dar fără a permite execuția de comenzi pentru fiecare element extras.
\cmd{while read} permite prelucrarea suplimentară (nu doar extragerea) câmpurilor parsate de pe fiecare linie.

La fel ca \cmd{for}, comanda \cmd{while} poate fi folosită necorespunzător într-un mod în care robustețea, lizibilitatea sau performanța sunt afectate.
În general, folosirea buclelor, în special pentru prelucrarea text poate fi sursă de probleme în scripturi shell\footnote{\url{https://unix.stackexchange.com/a/169765/227174}}.

\subsection{Funcții}
\label{sec:auto:script-func:func}

Atunci când avem secvențe de comenzi pe care vrem să le folosim modular, putem folosi funcții.
La fel ca într-un limbaj de programare, o funcție are un nume și are un conținut; o funcție rulează când este apelată.
O funcție este apelată folosind numele său, ca în \labelindexref{Listing}{lst:auto:functions}, care prezintă secvențe din scriptul \file{/etc/init.d/ssh}.
Sunt definite funcțiile \texttt{check\_dev\_null}, \texttt{check\_privsep\_dir} și \texttt{check\_config}, care sunt apoi apelate pe parcursul programului.

\begin{screen}[caption={Folosirea funcțiilor (/etc/init.d/ssh)},label={lst:auto:functions}]
student@uso:~/uso-book/code/13-auto$ cat /etc/init.d/ssh
[...]
check_dev_null() {
    if [ ! -c /dev/null ]; then
	if [ "$1" = log_end_msg ]; then
	    log_end_msg 1 || true
	fi
	if ! run_by_init; then
	    log_action_msg "/dev/null is not a character device!" || true
	fi
	exit 1
    fi
}

check_privsep_dir() {
    # Create the PrivSep empty dir if necessary
    if [ ! -d /run/sshd ]; then
	mkdir /run/sshd
	chmod 0755 /run/sshd
    fi
}

check_config() {
    if [ ! -e /etc/ssh/sshd_not_to_be_run ]; then
	/usr/sbin/sshd $SSHD_OPTS -t || exit 1
    fi
}

export PATH="${PATH:+$PATH:}/usr/sbin:/sbin"

case "$1" in
  start)
	check_privsep_dir
        [...]
	check_dev_null
        [...]
        ;;

  [...]
  reload|force-reload)
	check_for_no_start
	check_config
        [...]
	;;
[...]
\end{screen}

Din punctul de vedere al folosirii, o funcție sau o comandă sunt echivalente:
\begin{itemize}
  \item pentru apelarea unei funcții, se scrie numele funcției (fără paranteze, cum e cazul limbajelor de programare)
  \item dacă funcția primește parametri, aceștia sunt transmiși în continuarea numelui funcției, separați prin spații
  \item o funcție poate întoarce un cod de ieșire (folosind comanda \texttt{return}), cod ce poate fi investigat
  \item o funcție poate afișa mesaje ce pot fi reținute prin construcția de expandare a comenzilor (\texttt{\$(\ldots)})
\end{itemize}
Doar uitându-ne la un script care folosește comenzi sau funcții, nu am putea să ne diferențiem între cele două.

Așa cum am observat și în \labelindexref{Listing}{lst:auto:functions}, funcțiile ajung să fie folosite în scripturi shell mai complexe, mai puțin în scripturi uzuale.
Scripturi mai complexe sunt scripturile de inițializare din directorul \file{/etc/init.d/}, cum este și cazul scriptului \file{/etc/init.d/ssh} din \labelindexref{Listing}{lst:auto:functions}.

\section{Facilități avansate și bune practici în shell scripting}
\label{sec:auto:script-advanced}

Așa cum am precizat anterior, un script shell este o cale de mijloc între a rula comenzile în linia de comandă și a face un program într-un limbaj de programare.
Pe lângă acestea, în shell sunt comenzi similare și funcționalități diferite care pot conduce la același rezultat.
Fiind mai multe modurile de a ajunge la un rezultat în shell, cel mai bine este să creăm scripturi shell folosind cea mai bună abordare dintre cele posibile.Pentru aceasta, pentru fiecare script vom urmări:

\begin{itemize}
  \item \textbf{robustețea}: scriptul să ruleze indiferent de condiții, să nu aibă erori de rulare sau rezultate imprevizibile
  \item \textbf{lizibilitate}: scriptul să fie ușor de citit și de înțeles de altcineva; anumite construcții în shell sunt întortocheate și pot face dificilă înțelegerea scriptului
  \item \textbf{eficiența și performanța}: deși nu poate ajunge la nivelul de performanță al unui program, e indicat ca scriptul să ruleze cât mai repede și să nu consume resurse în mod exagerat
  \item \textbf{portabilitatea}: există mai multe tipuri de shell (Bash, Dash, ksh, tcsh, zsh) sau diferite versiuni de shell; dacă se știe că un script va rula pe shelluri și în medii diferite trebuie să fie portabil și să ruleze determinist
\end{itemize}

Mai jos prezentăm construcții specifice acestor deziderate și recomandări de folosire în scripturi shell.

\subsection{Robustețe}
\label{sec:auto:script-advanced:robust}

Robustețea unui script înseamnă că acesta funcționează în condiții cât mai diverse.
Acest lucru înseamnă că având intrări diferite, sau pe configurații diferite sau cu intrări în sistemul de fișiere diferite, scriptul va funcționa corespunzător.

Mai jos sunt indicate câteva surse de probleme care pot afecta robustețea scripturilor.
Vom detalia ulterior soluții la acestea:

\begin{itemize}
  \item \textbf{inconsecvența formatului datelor prelucrate}:
    Unele comenzi afișează informațiile în formate care pot varia și care pot produce probleme.
    Soluția este folosirea de comenzi sau fișiere cu informație consecventă sau a avea în vedere, cât de mult se poate, toate cazurile.
  \item \textbf{configurații diferite de sistem}: Un script presupune anumite configurații care nu sunt prezente pe toate sistemele.
    Sau necesită rularea în anumite condiții în absența cărora rularea duce la erori.
  \item \textbf{caractere speciale}: Intrările în sistemul de fișiere pot conține caractere speciale (precum spații sau chiar caractere newline) care nu sunt luate în considerare de script.
\end{itemize}

Așa cum am precizat în \labelindexref{Secțiunea}{sec:cli:basic-proc:gen} și în \labelindexref{Secțiunea}{sec:auto:how}, utilitarele generatoare de text sunt de două tipuri, în funcție de utilitatea rezultatului rulării:

\begin{enumerate}
  \item utilitare pentru afișarea de informații, pe ecran, pentru a fi accesibile utilizatorului
  \item utilitare care afișează informația în format prelucrabil, util pentru legarea de alte comenzi, adică pentru automatizare
\end{enumerate}

Pentru a nu fi afectate de inconsecvența datelor prelucrate, în scripturi e recomandată folosirea utilitarelor care au ca obiectiv prelucrarea, adică a doua categorie de mai sus.
Astfel, dacă dorim să obținem pentru prelucrare dimensiunea unui fișier vom folosi \cmd{stat} în loc de \cmd{ls}, ca în  \labelindexref{Listing}{lst:auto:ls-vs-stat}.
La fel, vom folosi \cmd{pidof} sau \cmd{pgrep} în loc de \cmd{ps}, ca în \labelindexref{Listing}{lst:auto:print-vs-process-cmd}.

Forme precum \cmd{ps $\textbar$ grep} rămân în continuare utile atunci când vrem să fie afișate utilizatorului doar anumite informații din rezultatul comenzii \cmd{ps}.
Dacă însă dorim prelucrarea acelor informații de alte comenzi, vom folosi \cmd{pidof} sau \cmd{pgrep}.

În partea de configurații diferite putem avea variabila de mediu \texttt{PATH} cu valori diferite.
Drept urmare, e posibil ca o comandă să poată fi rulată doar furnizând calea completă.
Adică să funcționeze \cmd{/bin/ls} dar nu \cmd{ls}.
Mai mult, anumite utilitare pot să nu fie instalate.
Dacă proiectăm scripturi de sistem (rulate cu permisiuni administrative) sau scripturi cu șanse de a a rula pe configurații diferite, vom specifica o comandă folosind calea completă.
În felul acesta, prevenim cazurile în care variabila de mediu \texttt{PATH} nu are conținutul așteptat.

Dacă un script se dorește rulat ca \texttt{root} trebuie verificat acest lucru printr-o construcție de genul celei din \labelindexref{Listing}{lst:auto:check-root}.
În caz contrar scriptul va fi rulat ca utilizator obișnuit (neprivilegiat) și vor apărea erori la rularea comenzilor privilegiate.

\begin{screen}[caption={Verificare root în script},label={lst:auto:check-root}]
if test $(id -u) -ne 0; then
    echo "You have to be root to run this script". 1>&2
    exit 1
fi
\end{screen}

Unele scripturi vor folosi \cmd{sudo} sau o conexiune SSH sau alte comenzi care pot cere (sau nu) introducerea unei parole.
Este posibil ca pe un sistem să avem configurată autentificarea SSH sau rularea \cmd{sudo} fără cererea parolei iar pe alte sistem să nu fie configurată.
Scriptul trebuie să aibă în vedere acest lucru și să configureze sistemul corespunzător sau să folosească un utilitar de automatizare a comenzilor neinteractive, așa cum vom preciza în \labelindexref{Secțiunea}{sec:auto:script-advanced:other}.

Când rulăm un script trebuie să avem în vedere din ce director îl rulăm.
În script este posibil să existe căi relative la directorul din care rulăm.
Din păcate, nu există nici o soluție care să funcționeze în toate cazurile\footnote{\url{http://mywiki.wooledge.org/BashFAQ/028}}.
Sunt recomandate două abordări:
\begin{enumerate}
  \item folosirea căilor absolute către fișierele folosite în script, precum plasarea fișierelor de configurare într-un director cu o cale fixă
  \item rularea scriptului doar din directorul în care se găsește scriptul
\end{enumerate}

Atunci când lucrăm cu fișiere trebuie să ținem cont de numele acestora.
Numele fișierelor pot conține caractere speciale precum spațiu (\textit{blank}), folosit de shell pentru separarea argumentelor.
De exemplu, în \labelindexref{Listing}{lst:auto:blanks-in-filenames}, avem o situație în care un director conține două subdirectoare (\file{movies/} și \file{books/}), fiecare subdirector conținând două fișiere al căror nume conțin spații; așa cum se vede în liniile \texttt{1-5}.
La o rulare a comenzii \cmd{for} în combinație cu \cmd{find}, nu vom obține o parcurgere corespunzătoare a fișierelor ce conțin spații; așa cum se vede în liniile \texttt{7-16}.
Soluția este fie folosirea \cmd{while} (liniile \texttt{18-22}), fie inițializarea separatorului din variabila \texttt{IFS} (\textit{Input Field Separator}) la caracterul linie nouă (\textit{newline}), prin construcția \texttt{\$'\textbackslash{}n'} (liniile \texttt{24-28}).

\begin{screen}[caption={Tratarea caracterelor speciale în numele fișierelor},label={lst:auto:blanks-in-filenames}]
student@uso:~/.../code/13-auto/filenames$ find -type f
./movies/The Godfather
./movies/Shawshank Redemption
./books/The Great Gatsby
./books/The Stand

student@uso:~/.../code/13-auto/filenames$ for f in $(find -type f); do stat -c "%n,%s" "$f"; done
stat: cannot stat './movies/The': No such file or directory
stat: cannot stat 'Godfather': No such file or directory
stat: cannot stat './movies/Shawshank': No such file or directory
stat: cannot stat 'Redemption': No such file or directory
stat: cannot stat './books/The': No such file or directory
stat: cannot stat 'Great': No such file or directory
stat: cannot stat 'Gatsby': No such file or directory
stat: cannot stat './books/The': No such file or directory
stat: cannot stat 'Stand': No such file or directory

student@uso:~/.../code/13-auto/filenames$ find -type f | while read f; do stat -c "%n,%s" "$f"; done
./movies/The Godfather,14
./movies/Shawshank Redemption,12
./books/The Great Gatsby,20
./books/The Stand,13

student@uso:~/.../code/13-auto/filenames$ IFS=$'\n'; for f in $(find -type f); do stat -c "%n,%s" "$f"; done
./movies/The Godfather,14
./movies/Shawshank Redemption,12
./books/The Great Gatsby,20
./books/The Stand,13
\end{screen}

Chiar și soluțiile care urmăresc citirea linie cu linie a elementelor (precum folosirea \cmd{while} sau inițializarea \texttt{IFS} la \textit{newline}) pot avea probleme dacă un fișier conține caracterul linie nouă (\textit{newline} - \texttt{\textbackslash{}n}) în numele său.
Deși ciudat, fișierele pot conține caracterul \textit{newline} în nume.
Nu este uzual să existe astfel de nume de fișiere, dar un atacator poate crea fișiere cu astfel de nume pentru a sabota funcționarea scriptului.
Trei dintre soluțiile posibile pentru cazul în care există \textit{newline} în numele unui fișier sunt prezentate în \labelindexref{Listing}{lst:auto:newline-in-names}:
\begin{itemize}
  \item Folosim globbing împreună cu \cmd{for}, ca în liniile \texttt{1-5}.
    Forma cu globbing este robustă la prezența \textit{newline} în numele fișierelor.
    Dar este problematică dacă avem nevoie de o parcurgere în adâncime (recursivă) a unor fișiere, caz în care preferăm să folosim \texttt{find}.
  \item Folosim \cmd{find} (de care am discutat în \labelindexref{Secțiunea}{sec:cli:advanced:find}) împreună cu opțiunea \texttt{-exec}, ca în liniile \texttt{7-11}.
    Această formă permite rularea unei singure comenzi pe intrare.
    Funcționează în cazul nostru, pentru că dorim rularea comenzii \cmd{stat} pentru fiecare intrare.
  \item Folosim \cmd{find} cu afișarea intrărilor separate de terminatorul de șir (\texttt{\textbackslash{}0}), ca în liniile \texttt{13-17}.
    Caz în care separatorul \texttt{IFS} este resetat, la fel și delimitatorul folosit de comanda \cmd{read}.
\end{itemize}

Aici\footnote{\url{https://www.dwheeler.com/essays/filenames-in-shell.html}} sunt descrise un spectru larg de soluții la situația caracterelor problematice în numele de fișiere.

\begin{screen}[caption={Tratarea prezenței newline în numele fișierelor},label={lst:auto:newline-in-names}]
student@uso:~/.../code/13-auto/filenames$ for f in ./*/*; do stat -c "%n,%s" "$f"; done
./books/The Great Gatsby,20
./books/The Stand,13
./movies/Shawshank Redemption,12
./movies/The Godfather,14

student@uso:~/.../code/13-auto/filenames$ find -type f -exec stat -c "%n,%s" {} \;
./movies/The Godfather,14
./movies/Shawshank Redemption,12
./books/The Great Gatsby,20
./books/The Stand,13

student@uso:~/.../code/13-auto/filenames$ find -type f -print0 | while IFS="" read -r -d "" f; do stat -c "%n,%s" "$f"; done
./movies/The Godfather,14
./movies/Shawshank Redemption,12
./books/The Great Gatsby,20
./books/The Stand,13
\end{screen}

\subsection{Lizibilitate}
\label{sec:auto:script-advanced:readable}

Scripturile sunt \textit{quick and dirty} și trebuie să fie lizibile, ușor de înțeles.
Pentru aceasta este de preferat să nu se folosească construcții în care se poate greși repede sau care sunt greu de înțeles.

De exemplu, pentru expandarea unei comenzi sunt posibile două forme, ca în \labelindexref{Listing}{lst:auto:command-expansion-options}:
\begin{itemize}
  \item Folosim caracterul \textit{backtick} (\texttt{`}) în sintaxa \texttt{`command`} ca în liniile \texttt{1-5}.
  \item Folosim construcția \texttt{\$(command)} ca în liniile \texttt{7-11}.
\end{itemize}
Între cele două forme, recomandăm a doua formă, care folosește construcția \texttt{\$(command)}.
Prima formă are probleme de lizibilitate: caracterul \textit{backtick} poate fi confundat cu apostrof; pe lângă aceasta, este mai dificil de localizat pe tastatură.
De partea cealaltă, construcția \texttt{\$(command)} este mai ușor de citit și tastat.

\begin{screen}[caption={Moduri de expandare a comenzilor},label={lst:auto:command-expansion-options}]
student@uso:~/.../code/13-auto/filenames$ IFS=$'\n'; for f in `find -type f`; do stat -c "%n,%s" "$f"; done
./movies/The Godfather,14
./movies/Shawshank Redemption,12
./books/The Great Gatsby,20
./books/The Stand,13

student@uso:~/.../code/13-auto/filenames$ IFS=$'\n'; for f in $(find -type f); do stat -c "%n,%s" "$f"; done
./movies/The Godfather,14
./movies/Shawshank Redemption,12
./books/The Great Gatsby,20
./books/The Stand,13
\end{screen}

Similar, între folosirea comenzii \cmd{test} și folosirea comenzii \cmd{$[$} preferăm folosirea comenzii \cmd{test}, așa cum am precizat și în \labelindexref{Secțiunea}{sec:auto:script-func:flow-control:if}.
Întrucât construcția \cmd{$[$} poate să fie văzută ca un operator, nu o comandă, și se pot omite spații, recomandăm folosirea comenzii \cmd{test}.

Atunci când avem de furnizat un mesaj unei comenzi putem folosi \textit{here documents} și \textit{here strings}, forme mai expresive de transmitere a informației.
\textit{Here documents} și \textit{here strings} sunt forme ,,inline'' de transmitere a informației așa cum sunt prezentate în \labelindexref{Listing}{lst:auto:here}.
Liniile \texttt{1-4} arată cum putem scrie 2 linii în fișierul \file{$\sim$/.inputrc} (\textit{here document}).
Linia \texttt{6} arată cum putem efectua calculul $2^{10}$, folosind utilitarul \cmd{bc} (\textit{here string}).
Ambele forme pot fi realizate și folosind redirectări / \textit{pipe}; dar, din rațiuni de lizibilitate, sau pentru a nu crea un proces nou, putem folosi \textit{here documents} și \textit{here strings}

\begin{screen}[caption={Here document și here string},label={lst:auto:here}]
cat > ~/.inputrc <<END
set show-all-if-ambiguous On
set completion-ignore-case On
END

bc <<< "2^\verb+^+^10"
\end{screen}
% I double dog dare you to come up with ^\verb+^+^ for making a ^ inside the screen environment without looking at the .sty.

\subsection{Eficiență}
\label{sec:auto:script-advanced:efficient}

Acolo unde avem mai multe opțiuni și avem robustețea asigurată, e de preferat să folosim soluția cea mai eficientă și performantă.
De exemplu, între variantele din \labelindexref{Listing}{lst:auto:newline-in-names}, am prefera ultima variantă, care folosește doar \cmd{find}, fără \cmd{while} sau \cmd{for}.
Este cea mai simplă și, deci, cea mai eficientă.
Pe lângă problemele inerente de robustețe posibile a procesului de parsing, o variantă precum cea cu \cmd{while} durează mai mult, trebuind să citească ieșirea comenzii \cmd{find} și apoi să o transfere către comandă.
Variantele folosind \cmd{while} și \cmd{for} rămân utile atunci când vrem să rulăm mai multe comenzi; comanda \cmd{find} permite, împreună cu \texttt{-exec} rularea unei singure comenzi\footnote{Se poate trece peste această limitare prin rularea unui shell nou sau a unui script, dar se pierde lizibilitate.}.

Altfel, atunci când folosim \cmd{find} pentru a executa o comandă, avem opțiunile din \labelindexref{Listing}{lst:auto:find-options}.
Prima variantă (liniile \texttt{1-5}) execută comanda \cmd{stat} pentru fiecare intrare; sunt create astfel 4 procese \cmd{stat}, câte una pentru fiecare fișier.
A doua (liniile \texttt{7-11}) și a treia (liniile \texttt{13-17}) variantă execută o singură comandă \cmd{stat} pentru toate intrările, fiind evident mai performantă.
A doua și a treia variantă sunt aproape echivalente.
Prima variantă creează câte un proces pentru fiecare fișier fiind mai costisitoare ca timp.
Lucrurile de acest fel încep să conteze când scripturile prelucrează cantități mari de informații: multe fișiere sau multe date de intrare.

\begin{screen}[caption={Forme de rulare de comenzi pe fișiere (cu find)},label={lst:auto:find-options}]
student@uso:~/.../code/13-auto/filenames$ find -type f -exec stat -c "%n,%s" {} \;
./movies/The Godfather,14
./movies/Shawshank Redemption,12
./books/The Great Gatsby,20
./books/The Stand,13

student@uso:~/.../code/13-auto/filenames$ find -type f -exec stat -c "%n,%s" {} +
./movies/The Godfather,14
./movies/Shawshank Redemption,12
./books/The Great Gatsby,20
./books/The Stand,13

student@uso:~/.../code/13-auto/filenames$ find -type f -print0 | xargs -0 stat -c "%n,%s"
./movies/The Godfather,14
./movies/Shawshank Redemption,12
./books/The Great Gatsby,20
./books/The Stand,13
\end{screen}

\begin{note}[Important]
Cu toate acestea, urmărind robustețea, există o situație în care prima forma a comenzii \cmd{find} (liniile \texttt{1-5} din \labelindexref{Listing}{lst:auto:find-options}) este preferată: atunci când sunt foarte multe intrări descoperite, celelalte forme pot eșua din cauza limitării numărului de argumente pe care le poate lua o comandă.
\end{note}

Ca un plus de performanță, dintre a doua și a treia variantă, varianta a treia (care folosește comanda \cmd{xargs}) poate fi utilă atunci când folosim opțiunea \texttt{-P}.
Această opțiune permite precizarea numărului de core-uri folosite pentru rularea comenzilor în paralel, ceea ce poate duce la creșterea performanței.

\subsection{Portabilitate}
\label{sec:auto:script-advanced:portable}

Anumite construcții din scripturi shell sunt specifice unui tip de shell (precum Bash).
Le mai numim \textit{bashisms}.
Dacă dorim să rulăm un script pe mai multe tipuri de shell\footnote{\url{https://stackoverflow.com/questions/19428418/what-is-the-use-of-portable-shell-scripts}} sau pe diferite versiuni de shell, trebuie să avem în vedere ce construcții sunt standard și portabile.
Pentru scripturi portabile recomandarea este folosirea de funcționalități care sunt recunoscute de standardul POSIX\abbrev{POSIX}{Portable Operating System Interface}.
Un shell compatibil POSIX\footnote{\url{http://pubs.opengroup.org/onlinepubs/009695399/utilities/xcu\_chap02.html}} va rula un script construit conform standardului.

De exemplu, pentru redirectarea simultană a ieșirii standard și ieșirii de eroare standard avem două moduri: \verb|&> file| sau \verb|> file 2>&1|.
Preferăm opțiunea a doua, care este portabilă.
Prima variantă nu funcționează, de exemplu, pe \texttt{sh} (Bourne Shell).

O altă formă de bashism este construcția \cmd{$[[$} pentru testarea condițiilor, despre care am discutat în \labelindexref{Secțiunea}{sec:auto:script-func:flow-control:if}, o construcție specifică Bash.
Construcția se găsește și pe alte shelluri, dar nu este garantată prezența ei pe toate shellurile.
Dacă dorim să avem scripturi portabile, vom evita folosirea \cmd{$[[$}.
Dacă avem nevoie de comparația valorii unei variabile cu o expresie regulată, vom apela la o soluție alternativ, folosind \cmd{grep}, ca în \labelindexref{Listing}{lst:auto:grep-vs-if-brackets}.
A doua variantă folosește \cmd{grep} pentru a realiza același lucru: verificarea că valoarea \texttt{ana} corespunde expresiei regulate \texttt{.*\$}.
Varianta cu \cmd{grep} este mai complicată și mai puțin lizibilă, dar este portabilă.

\begin{screen}[caption={Alternativă la folosirea {[[} (grep)},label={lst:auto:grep-vs-if-brackets}]
student@uso:~/uso-book/code/13-auto$ if [[ "ana" =~ .*a$ ]]; then echo "match"; fi
match

student@uso:~/uso-book/code/13-auto$ if grep '.*a$' <<< "ana" > /dev/null 2>&1; then echo "match"; fi
match
\end{screen}

Am văzut în \labelindexref{Listing}{lst:auto:newline-in-names} și în \labelindexref{Listing}{lst:auto:find-options} că, pentru a folosi robust prelucrarea fișierelor cu \cmd{find}, trebuie să folosim opțiunea \texttt{-print0}.
Opțiunea \texttt{-print0} nu este, însă, în standardul POSIX și este posibil să nu fie prezentă pe anumite sisteme.
Din păcate, folosirea \cmd{find} cu opțiunea \texttt{-print0} poate ajunge sa fie singura soluție pentru prelucrarea robustă a fișierelor dintr-o ierarhie; așa că aici trebuie optat între portabilitate 100\% și robustețe.

\subsection{Alte funcționalități}
\label{sec:auto:script-advanced:other}

Atunci când dorim să facem debugging în shell, soluția cea mai directă este afișarea unui mesaj sau a valorilor unei variabile folosind comenzile \cmd{printf} sau \cmd{echo}.
O altă variantă este să folosim în cadrul unui script shell construcțiile \texttt{set -x} și \texttt{set +x} care activează, respectiv dezactivează, afișarea comenzilor atunci când se execută.
Acest lucru permite să vedem dacă o comandă se execută sau nu și toate expandările realizate (variabile, expandare de comenzi, expandare aritmetică).
În scriptul \file{gen-passwd-dbg} din \labelindexref{Listing}{lst:auto:gen-passwd-dbg} activăm afișarea comenzilor pentru cele trei linii care folosesc expandarea comenzilor.
Observăm că se afișează și comenzile executate și atribuirile de variabile.
Dacă avem probleme într-un script, acest mod de depanare este util să depistăm rapid problemele.

\begin{screen}[caption={Permisiuni pe fișiere cu parole},label={lst:auto:gen-passwd-dbg}]
student@uso:~/uso-book/code/13-auto$ cat gen-passwd-dbg
#!/bin/bash

set -x
salt=$(pwgen -N 1 -c 8)
password=$(pwgen -N 1 -c 10)
crypt=$(openssl passwd -1 -salt "$salt" "$password")
set +x

echo "password: $password"
echo "crypt: $crypt"

student@uso:~/uso-book/code/13-auto$ ./gen-passwd-dbg
++ pwgen -N 1 -c 8
+ salt=bohJ0noj
++ pwgen -N 1 -c 10
+ password=uu2tiiGhoc
++ openssl passwd -1 -salt bohJ0noj uu2tiiGhoc
+ crypt='$1$bohJ0noj$30nn0fhwx2RTHDS5SxmK30'
+ set +x
password: uu2tiiGhoc
crypt: $1$bohJ0noj$30nn0fhwx2RTHDS5SxmK30
\end{screen}

Se întâmplă adesea să folosim comenzi care necesită intrare de la utilizator într-un script shell.
Aceste comenzi sunt interactive și nu pot fi folosite în această formă într-un script shell în care se dorește automatizare completă.
De exemplu comenzi precum \cmd{update-alternatives}, \cmd{passwd}, \cmd{ftp}, \cmd{curl}, \cmd{wget}, \cmd{ssh}, \cmd{sudo}.
Pentru unele există forme de rulare neinteractivă: în loc de \cmd{passwd} se poate folosi \cmd{chpasswd}; \cmd{curl} și \cmd{wget} au opțiuni de furnizare de nume de utilizator și parolă.
Atunci când nu există forme de rulare neinteractivă, se pot folosi utilitare din categoria \cmd{expect} care permit transmiterea de informații în mod automat către comenzi interactive.
Le vom detalia în \labelindexref{Secțiunea}{sec:auto:interactive}.

\subsection{Resurse suplimentare}
\label{sec:auto:script-advanced:resources}

Pentru sfaturi practice despre dezvoltarea shell scripturilor (și folosirea Bash) recomandăm ghidul BashFAQ: \url{https://mywiki.wooledge.org/BashFAQ}

\section{Automatizarea la nivelul sistemelor informatice}
\label{sec:auto:system}

\subsection{Gestiunea serviciilor}
\label{sec:auto:system:service}

În \labelindexref{Capitolul}{chapter:boot} am precizat că procesul \texttt{init} (de obicei implementat folosind \texttt{systemd} în distribuțiile Linux moderne) este responsabil pentru pornirea primelor procese ale sistemului, majoritatea fiind serviciile.
Serviciile sunt procese de tip daemon, care rulează în background și care gestionează sistemul sau oferă funcționalități pentru alte aplicații.
Serviciile sunt gestionate de \texttt{systemd} fie în modul compatibil System V (folosind intrările din directorul \file{/etc/init.d/}) fie în modul specific \texttt{systemd}, așa cum am precizat în \labelindexref{Secțiunea}{sec:boot:init-linux}.

Dacă dorim să adăugăm un serviciu nou în sistem putem face acest lucru folosind \texttt{systemd}.
Aceasta presupune crearea unui fișier de configurare pentru serviciu (numit \texttt{unit} în \texttt{systemd}) și apoi încărcarea sa în \texttt{systemd}\footnote{\url{https://www.devdungeon.com/content/creating-systemd-service-files}}.
Putem face acest lucru cu un serviciu la nivelul utilizatorului (nu la nivelul sistemului), folosind instanța de \cmd{systemd} specifică utilizatorului (\cmd{systemd -{}-user})\footnote{\url{https://www.brendanlong.com/systemd-user-services-are-amazing.html}}.

\labelindexref{Listing}{lst:auto:systemd-user} conține o secvență de comenzi pentru activarea unui serviciu \texttt{systemd} la nivelul utilizatorului.
Serviciul este definit în fișierul \texttt{dater.service}, fișier care va fi plasat în directorul de configurare \texttt{systemd} de la nivelul utilizatorului: \file{$\sim$/.config/systemd/user/}.
În fișierul serviciului, directiva de configurare \texttt{ExecStart} precizează care este comanda executată de serviciu, în cazul de față rularea scriptului \file{/home/student/uso-book/code/13-auto/systemd/\allowbreak{}dater}.  % \allowbreak to allow breaking the line and attempt to unfuck formatting.
Acest script este unul didactic, care afișează data curentă în fișierul \texttt{/home/student/date} la fiecare \texttt{60} de secunde.
După activarea și pornirea serviciului, verificăm fișierul \texttt{/home/student/date} (liniile \texttt{40-44}) și validând, astfel, funcționarea corectă a scriptului.
O dată instalat serviciul în directorul \file{$\sim$/.config/systemd/user/} rulăm comenzile de listare / verificare (linia \texttt{31}) și de activare, pornire, oprire, dezactivare.

\begin{screen}[caption={Creare serviciu systemd la nivelul utilizatorului},label={lst:auto:systemd-user}]
student@uso:~/.../code/13-auto/systemd$ ls
dater  dater.service

student@uso:~/.../code/13-auto/systemd$ cat dater.service
[Unit]
Description=My test dater service

[Service]
Type=simple
WorkingDirectory=/home/student
ExecStart=/bin/bash /home/student/uso-book/code/13-auto/systemd/dater
Restart=always
PrivateTmp=true
NoNewPrivileges=true

[Install]
WantedBy=default.target

student@uso:~/.../code/13-auto/systemd$ cat /home/student/uso-book/code/13-auto/systemd/dater
#!/bin/bash

while true; do
    date > /home/student/date
    sleep 60
done

student@uso:~/.../code/13-auto/systemd$ mkdir -p ~/.config/systemd/user/

student@uso:~/.../code/13-auto/systemd$ cp dater.service ~/.config/systemd/user/

student@uso:~/.../code/13-auto/systemd$ systemctl --user list-unit-files | grep dater
dater.service                         disabled

student@uso:~/.../code/13-auto/systemd$ systemctl --user enable dater
Created symlink /home/student/.config/systemd/user/default.target.wants/dater.service -> /home/student/.config/systemd/user/dater.service.

student@uso:~/.../code/13-auto/systemd$ systemctl --user start dater
dater.service                         enabled

student@uso:~/.../code/13-auto/systemd$ cat /home/student/date
joi 7 ianuarie 2021, 05:39:24 +0200

student@uso:~/.../code/13-auto/systemd$ cat /home/student/date
joi 7 ianuarie 2021, 05:42:24 +0200

student@uso:~/.../code/13-auto/systemd$ systemctl --user stop dater

student@uso:~/.../code/13-auto/systemd$ systemctl --user disable dater
Removed /home/student/.config/systemd/user/default.target.wants/dater.service.

student@uso:~/.../code/13-auto/systemd$ systemctl --user list-unit-files | grep dater
dater.service                         disabled
\end{screen}

\subsubsection{supervisor}
\label{sec:auto:system:service:supervisor}

Supervisor este un daemon care permite configurarea de servicii și monitorizarea acestora.
Are funcționalitate similară serviciilor configurate cu \texttt{systemd}.
În general, configurările \texttt{systemd} pot fi realizate și în supervisor și invers\footnote{\url{https://gunes.io/2017/08/25/systemd-vs-supervisor/}}.

Supevisor conține un daemon (\texttt{supervisord}), un utilitar de gestiune (\cmd{supervisorctl}) și fișiere de configurare.
Față de \texttt{systemd}, supervisor deține fișiere de configurare ceva mai simple de scris și poate fi configurat ușor să ruleze servicii ca alt utilizator.

\labelindexref{Listing}{lst:auto:supervisor} conține un exemplu configurare de serviciu supervisor pentru soluția vmchecker de verificare automată a submisiilor de teme\footnote{\url{https://vmchecker.cs.pub.ro/}}.
Fișierul de configurare \file{/etc/supervisor/conf.d/vmchecker.conf} conține mai multe servicii configurate.
Unul dintre aceste servicii este numit \texttt{vmchecker-ts} și rulează o comandă specifică pentru verificarea submisiilor pentru cursul de Teoria Sistemelor (TS).
Folosind \cmd{supervisorctl} putem afla starea acestui serviciu, îl putem opri sau porni.

\begin{screen}[caption={Configurare supervisor},label={lst:auto:supervisor}]
root@checker:~# cat /etc/supervisor/conf.d/vmchecker.conf
[...]
[program:vmchecker-ts]
command=/usr/local/bin/vmchecker-queue-manager --course_id=TS --stdin=/dev/null --stdout=/home/ts/tester/queue-manager.log --stderr=/home/ts/tester/queue-manager.log
directory=/srv
autostart=true
autorestart=true
startretries=3
stderr_logfile=/dev/null
stdout_logfile=/dev/null
user=ts

root@checker:~# supervisorctl status vmchecker-ts
vmchecker-ts                     RUNNING    pid 29463, uptime 59 days, 7:31:15
root@checker:~# supervisorctl stop vmchecker-ts
vmchecker-ts: stopped
root@checker:~# supervisorctl status vmchecker-ts
vmchecker-ts                     STOPPED    Jan 07 05:52 AM
root@checker:~# supervisorctl start vmchecker-ts
vmchecker-ts: started
root@checker:~# supervisorctl status vmchecker-ts
vmchecker-ts                     RUNNING    pid 27707, uptime 0:00:03
\end{screen}

\subsection{Planificarea sarcinilor}
\label{sec:auto:system:sched}

De multe ori dorim rularea de sarcini periodice sau la un moment dat.
Pentru aceasta în Linux avem servicii specifice: \cmd{cron} și \cmd{at}.
\cmd{cron} este folosit pentru rularea periodică a unei comenzi sau script.
\cmd{at} este folosit pentru rularea unei comenzi sau a unui script la un moment dat.

O formă apropiată ideii de rulare planificată este rularea după un anumit eveniment cum este cazul procesului de \textit{continuous integration} (CI\abbrev{CI}{Continuous Integration}).
În cazul continuous integration, orice actualizare a unui proiect software declanșează în spate rularea unor scripturi de verificare.
Exemple de utilitare de continuous integration sunt Travis CI, Jenkins, GitLabCI.

\subsubsection{cron}
\label{sec:auto:system:sched:cron}

\cmd{cron} este un serviciu care duce la rularea periodică a unei comenzi.
Un caz de utilizare pentru \cmd{cron} sunt scenariile de \textit{nightly builds} în care la momente bine determinate se verifică infrastructura, sau se compilează aplicații.
Sau situații în care dorim o reindexare a fișierelor de pe sistemul local sau un backup periodic care beneficiază de folosirea \cmd{cron}.

\cmd{cron} poate fi configurat la nivel global în fișiere de configurare sau la nivelul unui utilizator.

La nivel global, \cmd{cron} este configurat în directoare / fișiere de configurare precum cele din linia \texttt{2} din \labelindexref{Listing}{lst:auto:cron}.
O linie de configurare are formatul celor din liniile \texttt{10-13}.
Liniile de configurare precizează momentul de periodicitate, utilizatorul folosit și comanda ce va fi executată.
Momentul de periodicitate stabilește minutul, ora, ziua din lună, luna, ziua din săptămână.
În felul acesta putem configura periodicități diverse de la rulare la fiecare minut până la rulare o dată pe an.

\begin{screen}[caption={Configurarea cron},label={lst:auto:cron}]
student@uso:~/uso-book/code/13-auto$ ls -dF /etc/cron*
/etc/cron.d/  /etc/cron.daily/  /etc/cron.hourly/  /etc/cron.monthly/  /etc/crontab  /etc/cron.weekly/

student@uso:~/uso-book/code/13-auto$ cat /etc/crontab
[...]
SHELL=/bin/sh
PATH=/usr/local/sbin:/usr/local/bin:/sbin:/bin:/usr/sbin:/usr/bin

# m h dom mon dow user	command
17 *	* * *	root    cd / && run-parts --report /etc/cron.hourly
25 6	* * *	root	test -x /usr/sbin/anacron || ( cd / && run-parts --report /etc/cron.daily )
47 6	* * 7	root	test -x /usr/sbin/anacron || ( cd / && run-parts --report /etc/cron.weekly )
52 6	1 * *	root	test -x /usr/sbin/anacron || ( cd / && run-parts --report /etc/cron.monthly )
#
\end{screen}

La nivel unui utilizator, configurarea \cmd{cron} se realizează prin intermediul comenzii \cmd{crontab -e}.
Adică nu se editează direct un fișier, ci se apelează o comandă, similar configurării \cmd{sudo} folosind comanda \cmd{visudo}.
\labelindexref{Listing}{lst:auto:cron-user} conține un exemplu de configurare \cmd{cron} la nivelul utilizatorului \texttt{student}.
Folosind comanda \cmd{crontab -l} afișăm configurarea curentă a utilizatorului, inițial vidă.
Folosind comanda \cmd{crontab -e} ni se prezintă un ecran de editare în care adăugăm o linie de configurare precum cea de la linia \texttt{10}.
Linia de configurare va instrui \cmd{cron} să ruleze scriptul \file{/home/student/uso-book/code/13-auto/systemd/dater} la fiecare 10 secunde.
Dacă dorim să ștergem configurația \cmd{cron} a unui utilizator, folosim comanda \cmd{crontab -r}.

\begin{screen}[caption={Configurarea cron la nivelul utilizatorului},label={lst:auto:cron-user}]
student@uso:~/uso-book/code/13-auto$ crontab -l
no crontab for student

student@uso:~/uso-book/code/13-auto$ crontab -e
[...]

student@uso:~/uso-book/code/13-auto$ crontab -l
[...]
# m h  dom mon dow   command
*/10 * * * * /home/student/uso-book/code/13-auto/systemd/dater
\end{screen}

\subsubsection{at}
\label{sec:auto:system:sched:at}

Comanda \cmd{at} (care folosește în spate daemonul \texttt{atd}) permite rularea unei comenzi la un moment dat în viitor.
Comanda primește ca argument timpul când să aibă loc execuția și la intrarea standard comanda care să fie executată, ca în \labelindexref{Listing}{lst:auto:at} (liniile \texttt{3} și \texttt{7}).
Am instalat în prealabil pachetul \cmd{at}.
Similar opțiunilor comenzii \cmd{crontab}, folosind opțiunea \texttt{-l} putem lista planificările curente, iar folosind opțiunea \texttt{-r} putem șterge planificări.

\begin{screen}[caption={Planificarea unei acțiuni folosind at},label={lst:auto:at}]
student@uso:~/uso-book/code/13-auto$ sudo apt install at

student@uso:~/uso-book/code/13-auto$ echo /home/student/uso-book/code/13-auto/systemd/dater | at midnight
warning: commands will be executed using /bin/sh
job 1 at Fri Jan  8 00:00:00 2021

student@uso:~/uso-book/code/13-auto$ echo /home/student/uso-book/code/13-auto/systemd/dater | at 2021-01-20
warning: commands will be executed using /bin/sh
job 2 at Wed Jan 20 06:18:00 2021

student@uso:~/uso-book/code/13-auto$ at -l
1	Fri Jan  8 00:00:00 2021 a student
2	Wed Jan 20 06:18:00 2021 a student

student@uso:~/uso-book/code/13-auto$ at -r 1
student@uso:~/uso-book/code/13-auto$ at -r 2
student@uso:~/uso-book/code/13-auto$ at -l
student@uso:~/uso-book/code/13-auto$
\end{screen}

\section{Anexă: Utilitare pentru interacțiunea cu programe interactive}
\label{sec:auto:interactive}

Atunci când dorim să automatizăm și să folosim în scripturi programe interactive, lucrurile devin problematice.
În general, ne propunem ca un script să fie complet automatizat și neinteractiv.
Dacă putem, căutăm variante neinteractive ale acelor programe.
Altfel, unde se poate, folosim utilitarul \cmd{expect}.

Utilitarul \cmd{expect} permite transmiterea de informații în mod automat la intrarea standard a unor programe interactive, ca în scriptul \file{ftp.exp} din \labelindexref{Listing}{lst:auto:expect}.
Acest script deschide o conexiune FTP folosind utilitarul \cmd{nctftp} și transmite parola și comenzi FTP (\texttt{dir} și \texttt{quit}) în mod neinteractiv folosind \cmd{expect}.

\begin{screen}[caption={Folosire expect pentru programe interactive},label={lst:auto:expect}]
student@uso:~/uso-book/code/13-auto$ cat ftp.exp
#!/usr/bin/expect

set ftp_host repository.grid.pub.ro
set ftp_user clusterncituso
set ftp_password "XXXXXXXXXXXXXXXX"

spawn ncftp -u $ftp_user $ftp_host
expect "Password: "
send -- $ftp_password\r
expect ">"
send -- "dir\r"
expect ">"
send -- "quit\r"
expect eof

student@uso:~/uso-book/code/13-auto$ ./ftp.exp
spawn ncftp -u clusterncituso repository.grid.pub.ro
[...]
Password requested by 141.85.241.222 for user "clusterncituso".

    User clusterncituso OK. Password required

Password: *********

OK. Current restricted directory is /
Logged in to repository.grid.pub.ro.
ncftp / > dir
drwxr-xr-x                Aug 31  2018   2015-2016
drwxr-xr-x                Aug 31  2018   2016-2017
drwxr-xr-x                Oct 13  2017   2017-2018
drwxr-xr-x                Nov  4  2018   2018-2019
drwxr-xr-x                Sep 26 13:55   2019-2020
drwxr-xr-x                Feb 22  2020   experimental
drwxr-xr-x                Jan  4 22:45   old
-rw-r-----   5239211520   Jan  7  2019   USO-CTF.ova
-rw-r-----          124   Jan  7  2019   USO-CTF.ova.sums
-rw-r-----   1090779648   Nov 22  2015   USO Demo.ova
-rw-r-----   4818603008   Oct 27 00:04   USO.ova
-rw-rw-r--           74   Oct 26 23:36   USO.ova.sums
-rw-r-----   1607888384   Oct 29 21:14   USOTeme.ova
-rw-rw-r--           78   Oct 29 21:10   USOTeme.ova.sums
-rw-r-----   3370503680   Aug 31  2018   USO_tom_jerry.ova
-rw-r-----          136   Aug 31  2018   USO_tom_jerry.ova.sums
ncftp / > quit


	Thank you for using NcFTP Client.
	Ask your system administrator to try NcFTPd Server!
	http://www.ncftp.com/ncftpd/
\end{screen}

Comenzile \cmd{expect} creează un proces (\cmd{spawn}) căruia apoi îi pot controla intrarea standard și ieșirea standard, putând astfel realiza o comunicare neinteractivă.
Îi sunt transmise comenzi intrării standard a procesului interactiv folosind comanda \cmd{send}.
Așa trimitem parola și comenzile FTP (\texttt{dir} și \texttt{quit}).
Ieșirea procesului este capturată și analizată folosind comanda \cmd{expect}.
Cu ajutorul comenzii \cmd{expect} putem realiza acțiuni în funcție de ieșirea procesului; de exemplu, să știm ce trebuie să afișeze / transmită procesul pentru ca noi să-i transmitem o intrare specifică.

Pentru o variantă programatică a utilitarului \cmd{expect}, putem folosi \cmd{pexpect}\footnote{\url{https://pexpect.readthedocs.io/en/stable/}}.
\cmd{pexpect} oferă o interfață programatică Python în care pot fi operate funcționalitățile expect.

O variantă similară \cmd{pexpect}, folosită în comunitatea de securitate (hacking, competiții de tip CTF\abbrev{CTF}{Capture the Flag} - \textit{Capture the Flag}) este \cmd{pwntools}\footnote{\url{http://docs.pwntools.com/en/stable/}}.
Pe lângă modul de comunicare neinteractivă cu aplicații interactive, \cmd{pwntools} oferă funcționalități specifice pentru securitate și hacking: analiză de executabile, construire de payloaduri pentru atacuri de securitate, formatări de date, integrare cu depanatoare etc.

\section{Anexă: Gestiunea configurației}
\label{sec:auto:config-management}

Un administrator de sistem și infrastructură care gestionează mai multe sisteme urmărește un mod rapid, automatizat de configurare a acelor sisteme.
În general acest lucru înseamnă ca un set de configurații de pe un sistem central să ajungă pe sistemele administrate în mod rapid și scalabil.
De obicei, astfel de soluții oferă administratorului o interfață programatică, pentru a configura ce informații de configurare ajung pe sisteme (numite uzual noduri).

Soluții de gestiune a configurației (\textit{Configuration Management}) sunt Chef, Puppet, Ansible, SaltStack, Terraform.

Astfel de soluții sunt folosite de cei care ocupă poziții de \textit{DevOps} (\textit{Development and Operations}): persoane responsabile cu buna funcționare a infrastructurii IT a unei organizații (\textit{Operations}) care folosesc facilități programatice (\textit{Development}) furnizate de utilitare precum cele din categoria \textit{Configuration Management}.

\section{Anexă: Bune practici în automatizare și scripting}
\label{sec:auto:good-practice}

Atunci când folosim scripting sau, în general, când automatizăm, e util să avem în vedere bune practici și, mai ales, perspective de lucru.
Aceste bune practici asigură o rulare rapidă, automată și robustă de sarcini repetitive astfel încât să obținem mai mult pentru activități creative.

Pentru persoana care lucrează cu un sistem de operare, perspectiva trebuie să fie ,,Orice fac poate fi automatizat. Are sens să automatizez?''
Dacă o acțiune se întâmplă frecvent, acea acțiune este candidat la automatizare pentru a eficientiza realizarea acțiunii: timp mai scurt pentru acțiune repetitivă, timp mai mult pentru alte acțiuni.

Atunci când automatizăm o acțiune, avem în vedere că sunt mai multe moduri de realizare a acesteia.
Nu există un mod unic și recomandat de rezolvare.
Pentru cazuri simple folosim comenzi simple, pentru cazuri complexe folosim utilitare avansate.
Este nevoie de o cunoaștere bună a comenzilor, utilitarelor și facilităților oferite pentru a putea face alegerea optimă în situația respectivă.

Pentru o situație dată, când avem mai multe moduri de a realiza un lucru, urmărim criterii precum: robustețe, lizibilitate, eficiență, portabilitate.
Scriptul sau utilitarul folosit trebuie să funcționeze pe cât mai multe sisteme, în cât mai multe situații, cât mai performant și să fie cât mai ușor de înțeles.

În Linux, formatul uzual de lucru este formatul text.
Majoritatea utilitarelor citesc text și afișează text, ceea ce le face ușor de folosit de un utilizator uman; și ușor de combinat folosind operatorul \textit{pipe} (\texttt{\textbar}).
Atunci când folosim scripturi, ne folosim de filtre de text care prelucrează text.
Iar dacă dezvoltăm noi utilitare, urmărim ca intrarea, ieșirea și configurarea lor să fie făcute tot folosind format text.

Multe utilitare fac un tip de acțiune.
Acea acțiune înseamnă de multe ori prelucrarea unei intrări și afișarea unui rezultate.
Astfel de utilitare pot fi combinate prin operatorul pipe sau prin funcționalități precum expandarea comenzii.
Când automatizăm în scripturi shell, e recomandat să avem perspectiva combinării de comenzi care pot duce la rezultatul dorit.

Nu folosim shell scripting ca alternativă la un limbaj de programare.
Shell scripting este o formă de automatizare, de combinare rapidă de comenzi pentru a obține un rezultat cât mai rapid.
Pentru calcule de înaltă performanță, pentru aplicații complexe, pentru eficiență în folosirea resurselor, folosim un limbaj de programare.

\section{Sumar}
\label{sec:auto:summary}

Folosim automatizare pentru sarcini repetitive sau care necesită rulare neinteractivă la un moment dat, lăsând mai mult timp pentru sarcini creative.

Avem mai multe limbaje pentru scripting, cu scenariile avantajoase fiecăruia.

Preferăm folosirea shell scripting pentru acțiuni de tipul \textit{quick and dirty}.

Scripturile shell combină comenzi existente, comenzi interne shellului și alte funcționalități ale shellului: înlănțuire comenzi, variabile, expandări, globbing, controlul fluxului.

Când creăm scripturi shell avem în vedere că sunt mai multe opțiuni posibile și le alegem urmărim robustețe, lizibilitate, eficiență, portabilitate.

Pentru gestiunea unui sistem sau infrastructuri avem utilitare dedicate pentru automatizare.
