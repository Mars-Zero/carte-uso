% Authors: Alexandru Radovici, Răzvan Deaconescu
% Editors: Sergiu Weisz, Răzvan Deaconescu
% Reviewers: Silviu Popescu, Andrei Deftu, Emma Mirică, Vlad Dogaru, Lucian Cojocar, Andrei Buhaiu

\chapter{Pachete software}
\label{ch:package}

Spre deosebire de hardware, software-ul are marele avantaj al flexibilității: pe un sistem hardware putem rula diferite tipuri de software, putem adăuga și șterge software, putem modifica ușor.
Atunci când avem o nevoie, putem fie să dezvoltăm software nou sau putem instala software existent în sistem.
Vom prezenta partea de dezvoltare software, de aplicații, în \labelindexref{Capitolul}{ch:appdev}, și ne vom concentra în acest capitol pe partea de instalare și configurare de aplicații existente.

Atunci când un dezvoltator sau o firmă software produce o aplicație nouă, este interesul acestora să ajungă cât mai eficient la utilizator.
Această acțiune, de distribuire a software-ului, necesită ca aplicația și datele sale să fie agregate într-o formă care să fie instalabilă de utilizator.
Această formă este numită pachet software.
După instalarea unui pachet software utilizatorul va putea să ruleze aplicația pe sistemul său.

Pentru ca o aplicație să ruleze, aceasta are nevoie de un fișier executabil și fișiere de date și fișiere de configurare.
De exemplu, un browser web dispune de un fișier executabil, are fișier de date în care reține parole de acces la site-uri și bookmark-uri și are o zonă de configurare precum pagina de start, prezența unor intrări în meniu, directorul implicit în care se descarcă fișierele.
Aceste fișiere sunt împachetate de cei care se ocupă de dezvoltare.
Fișierul rezultat în urma împachetării, pachetul software, este apoi transmis utilizatorului pentru instalare.

Instalarea unui pachet înseamnă că fișierul executabil, fișierele de date și fișierele de configurare sunt dispuse în sistemul local de fișiere de unde vor fi folosite.
De exemplu, pe Windows, toate fișierele sunt de obicei dispuse într-un subdirector din directorul \file{C:\textbackslash{}Program Files\textbackslash{}}.
Pe macOS toate fișierele sunt dispuse într-un subdirectorul din directorul \file{/Applications/}.
Pe Linux fișierele sunt distribuite în funcție de tipul lor: în \file{/usr/bin/} fișierele executabile, în \file{/usr/share/} și \file{/var/} fișiere de date, în \file{/etc/} fișiere de configurare.

Cea mai simplă formă de împachetare a unei aplicații este o arhivă.
O arhivă este un fișier reprezentând o colecție de fișiere puse unul lângă altul.
Arhiva nu are informații dedicate despre aplicație, astfel încât instalarea aplicației presupune dezarhivarea.
Avantajul folosirii arhivelor este simplitatea.
Dezavantajul este gestiunea dificilă a pachetelor: dezinstalarea devine greoaie, pentru că trebuie șterse manual toate fișierele;
unele pachete pot fi incompatibile unele cu altele și trebuie instalate alte versiuni.

Forma modernă de livrare a aplicațiilor o reprezintă pachetele software.
Un \textbf{pachet}, spre deosebire de o arhivă simplă, conține informații despre aplicație, precum descrierea pachetului, versiune, dependențe de alte pachete, pachetele similare;
numim aceste informații \textbf{metadatele pachetului}.
Un pachet respectă un format;
acest format este cunoscut de o aplicație specifică de gestiune a pachetelor, numită \textbf{manager de pachete}.

În anumite situații dorim anumite versiuni de pachete.
Aceste versiuni riscă să afecteze celelalte pachete instalate, sau să necesite configurări mai complexe.
Sau să aibă nevoie de alte pachete și diferite versiuni care duce la o problemă numită ,,dependency hell'' Din acest motiv o formă de livrare a pachetelor este ,,toate-în-unul'' (\textit{all-in-one}) cu toate dependențele în aceeași arhivă.
Vom vorbi despre acestea în \labelindexref{Secțiunea}{sec:package:all-in-one}.

În cele ce urmează vom prezenta partea de pachete software în forma lor tradițională și apoi vom discuta despre pachete \textit{all-in-one}.

\section{Pachete software}
\label{sec:package:overview}

Un \textbf{pachet software} este o colecție de fișiere: executabile, de date, de configurare, de documentație.
Un pachet este distribuit și apoi instalat pe un sistem pentru a adăuga o nouă funcționalitate acelui sistem.
În urma instalării pachetului, fișierele din colecție ajung în sistemul de fișiere.
Fișierele executabile sunt cele folosite pentru rula aplicațiile în formă de procese, așa cum vom preciza în \labelindexref{Secțiunea}{sec:process:process-program}.
Fișierele de date sunt baze de date, imagini, sunete, filme folosite de aplicație pentru a rula.
Fișierele de configurare permit modificarea parametrilor de funcționare a aplicației.
Fișierele de documentație vor fi citite de utilizator pentru a ști cum să folosească și cum este proiectată aplicația.

Tipurile de fișiere care vor fi instalate reprezintă datele unui pachet.
Pe lângă date, un pachet conține și metadate, adică informații despre acele date.
\textbf{Metadatele unui pachet} sunt folosite pentru gestiunea versiunilor și dependențelor acelui pachet.
De exemplu un pachet Debian (\texttt{.deb}) conține două fișiere comprimate: \file{data.tar.xz/} și \file{control.tar.xz} ca în \labelindexref{Listing}{lst:package:deb-contents}.

\begin{screen}[caption={Conținutul unui pachet Debian (.deb)},label={lst:package:deb-contents}]
student@uso:~$ ar tf /var/cache/apt/archives/grep_3.1-2build1_amd64.deb
debian-binary
control.tar.xz
data.tar.xz
\end{screen}

Pentru a vedea metadatele pachetului, putem desface pachetul și vedea conținutul.
Mai simplu, putem folosi o comandă dedicată ca în \labelindexref{Listing}{lst:package:metadata}.

\begin{screen}[caption={Metadatele unui pachet Debian (.deb)},label={lst:package:metadata}]
student@uso:~$ dpkg -I /var/cache/apt/archives/grep_3.1-2build1_amd64.deb
 new Debian package, version 2.0.
 size 158964 bytes: control archive=1416 bytes.
    1219 bytes,    29 lines      control
     746 bytes,    13 lines      md5sums
 Package: grep
 Version: 3.1-2build1
 Architecture: amd64
 [...]
 Pre-Depends: libc6 (>= 2.14), libpcre3
 Depends: dpkg (>= 1.15.4) | install-info
 Suggests: libpcre3 (>= 7.7)
 [..]

student@uso:~$ apt show grep
Package: grep
Version: 3.1-2build1
[...]
Pre-Depends: libc6 (>= 2.14), libpcre3
Depends: dpkg (>= 1.15.4) | install-info
Suggests: libpcre3 (>= 7.7)
[...]
\end{screen}

Conținutul unui pachet, cuprinzând datele și metadatele acestuia, este descris în \labelindexref{Figura}{fig:package:content}.

\begin{figure}[htbp]
  \centering
  \def\svgwidth{0.8\columnwidth}
  \includesvg{chapters/03-package/img/package-content.svg}
  \caption{Conținutul unui pachet}
  \label{fig:package:content}
\end{figure}

Metadatele au rol în informarea utilizatorului și să asigure funcționarea corespuzătoare a aplicațiilor pe sistemelor.
Pentru o funcționare corespunzătoare pachetele trebuie să fie compatibile cu sistemul și lucreze bine împreună: un pachet are nevoie de o anumită versiune de sistem de operare și de pachete să fie instalate pentru a funcționa.
De exemplu, pachetul \texttt{grep}, analizat în \labelindexref{Listing}{lst:package:metadata}, analizat mai sus, are nevoie de pachetele \texttt{libc6} (versiune mai mare sau egală cu \texttt{2.14}) și \texttt{libpcre3}.

În plus, un pachet software evoluează și numărul său de versiune crește.
Acea versiune necesită o anumită versiune a altui pachet sau altă versiune de sistem de operare.
Prin analizarea versiunii, vom ști dacă acel pachet este instalabil pe un sistem dat și ce pachete (și versiuni) trebuie să fie instalate.

Spunem că un pachet are dependențe de alte pachete.
Fără ca acele pachete să fie instalate, aplicația din pachet nu va funcționa.
Este responsabilitatea unui \textbf{manager de pachete} să asigure instalarea și a pachetelor de care pachetul curent depinde;
numim aceste pachete \textbf{dependențe}.
Vom prezenta soluții de tip manager de pachete în secțiunile următoare.

Modul în care sunt organizate datele și metadatele într-un pachet software determină formatul acestuia și tipul de pachet.
Fiecare sistem de operare și fiecare distribuție are un format de pachete și aplicații specifice pentru gestiunea acelui tip de pachete.
Aplicațiile specifice pentru gestiunea pachetelor se ocupă de instalarea, dezinstalarea și investigarea pachetelor, acțiuni ce vor fi descrise în \labelindexref{Secțiunea}{sec:package:ops}.

\subsection{Tipuri de pachete}
\label{sec:package:types}

Așa cum e prezentat în \labelindexref{Figura}{fig:package:content}, în general un pachet este o arhivă de date și metadate, o colecție.
Indiferent de tipul de pachet, în mod uzual se poate investiga folosind o soluție de dezarhivat.

Un prim tip de pachete software sunt arhive simple care conțin datele aplicației.
Așa cum am precizat mai sus, această formă este simplă dar nu permite dezinstalarea facilă și rezolvarea dependențelor.
Din acest motiv nu este foarte întâlnită, preferându-se tipuri de pachete care conțin metadate care sunt gestionate de aplicațiile specifice.

Atunci când nu se dorește folosirea unei aplicații specifice de gestiunea pachetelor, se pot livra pachetele în format de arhivă auto-extractivă (\textit{self extracting archive}) sau pachet executabil auto-extractiv (\textit{self-extracting executable}).
Acest pachet este rulat și apoi are loc instalarea.

Aplicațiile de pe sistemele Windows nu sunt livrate în general, cu un format de instalare.
De multe ori aplicațiile au un fișier numit \file{setup.exe} care este folosit pentru instalarea și configurarea aplicației.
Alte aplicații pot fi instalate dintr-un pachet software de tip MSI \abbrev{MSI}{Microsoft Install} (\textit{Microsoft Install}), folosite de aplicația de instalare numită Windows Installer.

Pe macOS aplicațiile sunt livrate în pachete de tip imagine de disk, fișiere cu extensia \texttt{.dmg} sau fișiere de tip \texttt{.pkg}.
Fișierele \texttt{.pkg} au avantajul că pot să conțină scripturi instalare și de configurare;
fișiere \texttt{.dmg} sunt imagini de disc care conțin aplicația care se instalează de obicei prin copierea ei, fără o operație de instalare.

În Linux există cea mai mare variație de tipuri de pachete și de soluții pentru gestiunea acestora.
În vreme ce la macOS și la Windows există o singură organizație responsabilă de gestiunea sistemului de operare (Apple, respectiv Microsoft), lumea Linux este diversă și are multe surse de aplicații, multe distribuții care livrează acele aplicații în pachete și care întrețin soluțiile de gestiunea lor.
Cele mai răspândite formate de pachete sunt formatul \texttt{.deb} (în distribuțiile bazate pe Debian) și formatul \texttt{rpm} (în distribuțiile bazate pe RedHat).
Le vom prezenta detaliat, respectiv în \labelindexref{Secțiunea}{sec:package:deb} și \labelindexref{Secțiunea}{sec:package:rpm}.

Sistemele de operare mobile (precum Android și iOS) folosesc un depozit (\textit{repository}) central de pachete: Google Play pentru Android și AppStore pentru iOS.
Aplicația dedicată de pe telefonul mobil descarcă și instalează pachetul din depozit.
Formatul folosit pe iOS este formatul \texttt{.ipa} (\textit{iPhone Archive}) iar pe Android este formatul \texttt{.apk} (\textit{Android Package}).

Indiferent de sistem de operare, distribuție și aplicație, formatele de pachete au caracteristici comune: conțin date și metadate, sunt un fișier de tip arhivă, există o aplicație dedicată pentru gestiunea acestora: instalare, dezinstalare, configurare, investigare.
În continuare vom prezenta cum gestionăm aceste pachete în sistemele de operare moderne.

\section{Gestiunea pachetelor}
\label{sec:package:manage}

Pentru a putea fi instalat pe sistemele utilizatorilor finali, un pachet trebuie să fie distribuit către aceștia.
Utilizatorii pot accesa manual un site sau un depozit de pachete (\textit{package repository}) de unde descarcă pachetul și unde apoi îl instalează.
Sau pot folosi o aplicație specifică pentru localizarea, descărcarea și instalarea pachetelor.
Aceste aplicații poartă numele de manager de pachete (\textit{package manager}).

Un \textbf{depozit de pachete} (\textit{package repository}) este locul sunt stocate pachetele software.
În general un sistem de operare sau o distribuție de sistem de operare are un repository sau mai multe cu pachetele corespunzătoare.
De exemplu, pentru Ubuntu există repository-uri pentru fiecare versiune: un repository pentru Ubuntu 18.04 Bionic Beaver și un repository pentru Ubuntu 18.10 Cosmic Cuttlefish.
Pe sistemele de operare mobile putem considera Google Play sau Apple AppStore repository-urile de pachete.
Depozitele conțin pachetele în formatul specific distribuției sau sistemului de operare;
de exemplu pachete \texttt{.deb} pentru distribuțiile Debian/Ubuntu, pachete \texttt{.rpm} pentru distribuțiile RedHat, sau pachete \texttt{.ipa} în Apple AppStore.

Un repository este identificat printr-un URL \abbrev{URL}{Uniform Resource Locator} (\textit{Uniform Resource Locator}).
De exemplu URL-ul pentru un repository Ubuntu este \url{http://archive.ubuntu.com/ubuntu/}.
Acest URL este folosit de managerul de pachete pentru a afla lista de pachete și pentru a descărca pachetele.

Managerul de pachete este instalat într-un sistem de operare/distribuție de la bun început.
Pentru a fi folosit se configurează URL-ul repository-ului / repository-urilor folosite.
Apoi utilizatorul folosește managerul pentru a instala sau dezinstala pachete.
Managerul ține cont de URL-ul repository-ului și de dependențele pachetelor pentru instalare și dezinstalarea comandată de utilizator la fel ca în \labelindexref{Figura}{fig:package:manager}.

\begin{figure}[htbp]
  \centering
  \def\svgwidth{\columnwidth}
  \includesvg{chapters/03-package/img/package-manager.svg}
  \caption{Folosirea managerului de pachete}
  \label{fig:package:manager}
\end{figure}

În momentul instalării pachetelor managerul de pachete reține fișierele care au fost instalate pentru a putea să le șteargă la dezinstalare.
Tot în momentul instalării, managerul de pachete va investiga dependențele pachetului care se dorește instalat pentru a le instala și pe acestea.
Tipurile de operații efectuate de managerul de pachete, la comanda utilizatorului, sunt descrise în \labelindexref{Secțiunea}{sec:package:ops}.

Noțiunea de manager de pachete este răspândită îndeosebi în distribuții Linux/BSD.
Cu toate acestea există soluții de acest tip pe Windows, precum Chocolatey\footnote{\url{https://chocolatey.org/}}.
Pe macOS se poate folosi Homebrew\footnote{\url{https://brew.sh/}} sau MacPorts\footnote{\url{https://www.macports.org/}}.
Spre deosebire de distribuțiile Linux/BSD, soluțiile de tip manager de pachete de pe Windows și macOS nu sunt distribuite oficiale de companiile producătoare, Microsoft și Apple.

\labelindexref{Tabelul}{tab:package:types} conține cele mai cunoscute tipuri de managere de pachete cu tipurile de pachete și distribuțiile pe care le folosesc, împreună cu aplicațiile și comenzile specifice.

\begin{table}[!htb]
  \scriptsize
  \begin{center}
    \begin{tabular}{ p{0.2\textwidth} p{0.15\textwidth} p{0.27\textwidth} p{0.23\textwidth} }
      \toprule
        \textbf{Manager de pachete} &
        \textbf{Tip de pachet} &
        \textbf{Sistem de operare / distribuție} &
        \textbf{Comenzi specifice} \\
      \midrule
        APT &
        \texttt{.deb} &
        distribuții bazate pe Debian &
        \cmd{apt}, \cmd{aptitude}, \cmd{dpkg} \\
      \midrule
        YUM \& DNF &
        \texttt{.rpm} &
        distribuții bazate pe RedHat &
        \cmd{dnf}, \cmd{rpm} \\
      \midrule
        ZYpp &
        \texttt{.rpm} &
        distribuții bazate pe openSUSE &
        \cmd{zypper}, \cmd{rpm} \\
      \midrule
        Pacman &
        \texttt{.tar} &
        Arch &
        \cmd{pacman}, \cmd{pactree}, \cmd{paccache} \\
      \midrule
        Portage &
        \texttt{.tar.gz} &
        Gentoo &
        \cmd{emerge}, \cmd{equery}, \cmd{eix} \\
      \midrule
        Port &
        \texttt{.txz} &
        FreeBSD &
        \cmd{pkg} \\
      \midrule
        Homebrew &
        \texttt{.tar.gz} &
        macOS &
        \cmd{brew} \\
      \midrule
        MacPorts &
        \texttt{.tbz2} &
        macOS &
        \cmd{port} \\
      \midrule
        Chocolatey &
        \texttt{.nupkg} &
        Windows &
        \cmd{choco} \\
      \bottomrule
    \end{tabular}
  \end{center}
  \caption{Managere și tipuri de pachete}
  \label{tab:package:types}
\end{table}

\subsection{Operații cu pachete}
\label{sec:package:ops}

Un utilizator realizează operații cu pachete prin intermediul comenzilor oferite de un manager de pachete.
Operațiile tipice sunt de instalare, dezinstalare, actualizarea versiunii (\textit{update}), investigare de pachete.
Un utilizator nu trebuie să cunoască anatomia funcționării unui manager de pachete sau formatul pachetelor.
Utilizatorul trebuie să știe cum configurează repository-urile unui manager de pachete și apoi să instaleze sau să dezinstaleze pachetele.

Operațiile uzuale realizate de un utilizator sunt descrise în continuare.
Vom descrie în detaliul modul în care se realizează aceste operații pe distribuțiile bazate pe Debian și cele bazate pe RedHat în \labelindexref{Secțiunea}{sec:package:deb}, respectiv \labelindexref{Secțiunea}{sec:package:rpm}.
Pentru alte distribuții/sisteme de operare și, deci, alte managere de pachete, agregăm comenzile uzuale în tabele în \labelindexref{Secțiunea}{sec:package:other}.

Cele mai directe operații realizate cu pachete sunt instalarea și dezinstalarea pachetelor.
Instalarea presupune că un pachet este despachetat și conținutul său este distribuit în sistemul de fișiere, eventual împreună cu rularea unui scripturi de configurare.
Dezinstalarea înseamnă ștergerea din sistemul de fișiere a conținutului unui pachet împreună cu rularea unor scripturi de deconfigurare.
Prin intermediul unui manager de pachete, instalarea poate însemna descărcarea pachetului dorit dintr-un repository de pachete împreună cu pachete de care acesta depinde.
În cazul interacțiunii cu repository-ul de pachete putem opta pentru descărcarea pachetului fără instalare.

Pentru a menține sistemul la zi, atât cu cele mai noi caracteristici ale pachetelor cât și cu rezolvarea anumitor probleme (\textit{bug}-uri) de programare sau de securitate, se recomandă acțiunea de actualizare a pachetelor (\textit{update}).
Operația de actualizare înseamnă descărcarea din depozit a celor mai recente versiuni și înlocuirea versiunilor mai vechi în sistemul de fișiere.

De multe ori nu știm ce pachet să instalăm, dat fiind numărul mare al acestora și faptul că mai multe pachete pot face același lucru: mai multe editoare, mai multe navigatoare web, mai multe playere de muzică.
De aceea o operație frecventă este căutarea unui pachet după cuvinte cheie relevante.
Căutarea se face de regulă în metadatele pachetelor din depozitul de pachete.

Alte acțiuni, mai puțin frecvente, sunt:

\begin{itemize}
  \item descărcarea surselor acelui pachet pentru a le putea investiga ulterior
  \item listarea pachetele instalate
  \item căutarea unui pachet al cărui nume se potrivește unei expresii
  \item vizualizarea conținutului unui pachet instalat, adică ce fișiere din sistemul de fișiere corespund acelui pachet
  \item identificarea pachetului care conține un anumit fișier din sistemul de fișiere
\end{itemize}
Aceste acțiuni au comenzi și opțiuni specifice fiecărui manager de pachete.

\subsection{Pachete în sistemele de operare mobile}
\label{sec:package:mobile}

Operațiile de mai sus au corespondent și în cazul sistemelor de operare mobile.
Prin intermediul aplicațiilor GooglePlay\footnote{\url{https://play.google.com/store}} și AppStore\footnote{\url{https://www.apple.com/ios/app-store/}} utilizatorii interacționează cu depozitele de pachete oferite de Google și de Apple, numite magazine de aplicații (\textit{app store}).
Într-un magazin de aplicații, utilizatorii caută pachete sau instalează pachete.

În urma unei comenzi de instalare, transparent utilizatorului, aplicația GooglePlay sau AppStore descarcă din depozit pachetul în format respectiv (\texttt{.apk} sau \texttt{.ipa}).
După instalare, aplicația arele fișiere executabile, de date, de configurare, de documentație în sistemul de fișiere și este, în general, accesibilă prin intermediul unei icoane pe ecranul telefonului mobil.

Depozitele de pachete folosite de aplicațiile GooglePlay și AppStore nu pot fi configurate, sunt preconfigurate în aplicație.
Cu toate acestea pot fi folosite aplicații cu depozite neoficiale de aplicații, precum Cydia pe iOS\footnote{\url{https://cydia-app.com/}}.
De avut în vedere, însă, că folosirea unor depozite neaprobate de producătorul dispozitivului mobil reprezintă un risc de securitate;
instalarea acestor aplicații poate însemna instalarea de aplicații malițioase (\textit{malware}).

\section{Gestiunea pachetelor în Linux}
\label{sec:package:linux}

În Linux, pachetele sunt colectate în depozite (\textit{repository}-uri) pentru fiecare distribuție.
Fiecare distribuție are apoi sub-depozite pentru fiecare versiune a distribuției, sub-depozit ce conține versiunile de pachete corespunzătoare.
Sistemul de gestiunea a pachetelor, cuprinzând managerul de pachete și alte componente, este cel responsabil pentru interacțiunea cu depozitul de pachete, la comanda utilizatorului.

Repository-ul de pachete este identificat printr-un URL care poate fi local (de exemplu CD-ROM) sau poate fi la distanță (acesta este modul uzual).
Managerul de pachete are o zonă de configurare în care putem specifica URL-urile repository-urilor folosite.

Pentru interacțiunea cu sistemul de gestiune a pachetelor, există aplicații dedicate.
Acestea pot fi grafice sau în linia de comandă.
De exemplu, pe sistemele bazate pe Debiam, există aplicația grafică Synaptic și utilitarul în linia de comandă \cmd{apt}.
La fel, pe Fedora și alte distribuții, există aplicația grafică PackageKit și utilitarul în linia de comandă \cmd{dnf}.
Aceste aplicații interacționează cu sistemul de gestiune a pachetelor, cu repository-uri și instalează pachete.
Pentru operarea directă asupra pachetelor aceste aplicații folosesc o comandă dedicată care cunoaște formatul pachetelor: \cmd{dpkg} pe sistemele derivate din Debian și \cmd{rpm} pe sistemele derivate din RedHat.

În continuare vom detalia comenzile folosite pe sistemele derivate din Debian (cu pachete DEB) și pe sistemele derivate din RedHat (cu pachete RPM).

\subsection{Gestiunea pachetelor format DEB}
\label{sec:package:deb}

Sistemele derivate din Debian folosesc pachete în format DEB, cu extensia \texttt{.deb}.
Aceste pachete sunt arhive \texttt{.ar} care conțin datele și metadatele.

Sistemul de gestiune a pachetelor este APT \abbrev{APT}{Advanced Package Tool} (\textit{Advanced Package Tool}).
Configurația acestuia se găsește în directorul \file{/etc/apt/}.
În fișierul \file{/etc/apt/sources.list} sunt configurate URL-uri de depozite de pachete.

\begin{screen}[caption={Configurație APT (Debian)},label={lst:package:deb-config}]
student@uso:~$ cat /etc/apt/sources.list
[...]
deb http://ro.archive.ubuntu.com/ubuntu/ bionic main restricted
deb-src http://ro.archive.ubuntu.com/ubuntu/ bionic main restricted
deb http://ro.archive.ubuntu.com/ubuntu/ bionic-updates main restricted
deb-src http://ro.archive.ubuntu.com/ubuntu/ bionic-updates main restricted
deb http://ro.archive.ubuntu.com/ubuntu/ bionic universe
deb-src http://ro.archive.ubuntu.com/ubuntu/ bionic universe
deb http://ro.archive.ubuntu.com/ubuntu/ bionic-updates universe
deb-src http://ro.archive.ubuntu.com/ubuntu/ bionic-updates universe
[...]
\end{screen}

\labelindexref{Listing}{lst:package:deb-config} prezintă o selecție a fișierului de configurație \file{/etc/apt/sources.list} de pe o distribuție Ubuntu 18.04 Bionic Beaver.
Fișierul conține o listă de URL-uri de repository-uri.
URL-uri precizează dacă repository-ul este folosit pentru descărcarea pachetelor (liniile care încep cu \texttt{deb}) sau pentru descărcarea surselor (liniile care încep cu \texttt{deb-src}).
După URL urmează versiunea distribuției (\texttt{bionic}) și apoi sunt componente de repository unde sunt clasificate pachetele (\texttt{main}, \texttt{restricted}, \texttt{universe}).

Interacțiunea cu APT se face în mod uzual în linia de comandă prin intermediul utilitarului \cmd{apt}.
\labelindexref{Listing}{lst:package:apt} prezintă operații frecvente folosind comanda \texttt{apt}.
Fiecare comandă este prefixată de un comentariu care explică rolul acelei comenzi.

\begin{screen}[caption={Operații frecvente cu apt},label={lst:package:apt}]
# Cauta pachete care au in nume sau in descrire sirul "svg".
student@uso:~$ apt search svg

# Instaleaza pachetul inkscape.
# Sau actualizeaza pachetul daca este deja instalat si exista o versiune mai noua.
student@uso:~$ sudo apt install inkscape

# Afiseaza informatii despre pachetul inkscape (instalat sau nu).
student@uso:~$ apt show inkscape

# Dezinstaleaza pachetul inkscape.
student@uso:~$ sudo apt remove inkscape

# Actualizeaza informatiile locale despre pachete.
student@uso:~$ sudo apt update

# Actualizeaza pachetele sistemului.
student@uso:~$ sudo apt upgrade

# Descarca pachetul inkscape fara sa fie instalat.
student@uso:~$ sudo apt -d remove inkscape

# Descarca sursele pachetului inkscape.
student@uso:~$ apt source inkscape
\end{screen}

Majoritatea acțiunilor realizate folosind utilitarul \cmd{apt} sunt privilegiate;
de aceea sunt prefixate de comanda \cmd{sudo}.
Excepție fac, în \labelindexref{Listing}{lst:package:apt}, acțiune de căutare a unui pachet și cea de descărcare a surselor pachetului.
Acestea sunt exceptate pentru că nu modifică fișiere de sistem.

Pentru operațiile din spate, de lucru cu pachete, utilitarul \cmd{apt} apelează uilitarul \cmd{dpkg}.
Utilitarul \cmd{dpkg} are informații despre pachete și despre instalările din sistem.
\labelindexref{Listing}{lst:package:dpkg} prezintă operații frecvente folosind utilitarul \cmd{dpkg}.
Fiecare comandă este prefixată de un comentariu care explică rolul acelei comenzi.

\begin{screen}[caption={Operații frecvente cu dpkg},escapechar=,label={lst:package:dpkg}]
# Afiseaza pachetele instalate local.
student@uso:~$ dpkg -l | grep '^ii'

# Cauta pachetele instalate local dupa expresia book.
student@uso:~$ dpkg -l *book*

# Identifica pachetul care contine fisierul /usr/bin/ssh.
student@uso:~$ dpkg -S /usr/bin/ssh

# Afiseaza continutul pachetului inkscape, instalat local.
student@uso:~$ dpkg -L inkscape
\end{screen}

\subsection{Gestiunea pachetelor format RPM}
\label{sec:package:rpm}

Similar distribuțiilor derivate din Debian, distribuțiile derivate din RedHat folosesc pachete RPM \abbrev{RPM}{RPM Package Manager} (\textit{RPM Package Manager}) cu extensia \texttt{.rpm}.
Aceste pachete sunt arhive \texttt{cpio} care conțin datele și metadatele.

Sistemul de gestiune a pachetelor pe Fedora este DNF \abbrev{DNF}{Dignified YUM} (\textit{Dignified YUM}), care este o îmbunătățire a vechiului sistem care numea YUM \abbrev{YUM}{Yellowdog UPdater Modified} (\textit{Yellowdog Updater Modified}) care la rândul său era o îmbunătățire de la YUP \abbrev{YUP}{Yellowdog Updater} (\textit{Yellowdog Updater});
Yellowdog a fost o distribuție Linux care a creat sistemul YUP.
Versiunile mai vechi (și mai stabile) de Red Hat Enterprise Linux (RHEL) sau CentOS (derivat din RHEL) folosesc în continuare YUM;
probabil la un moment dat vor tranzita spre DNF.
Configurația DNF se găsește în directorul \file{/etc/dnf/}.
În directorul \file{/etc/yum/yum.repos.d/} sunt configurate URL-uri repository-uri depozite de pachete.
\labelindexref{Listing}{lst:package:rpm-config} prezintă o configurație uzuală.

\begin{screen}[caption={Configurație de repository YUM/DNF (RedHat)},label={lst:package:rpm-config}]
[student@uso:~]$ cat /etc/yum.repos.d/fedora.repo
[fedora]
name=Fedora $releasever - $basearch
failovermethod=priority
#baseurl=http://download.fedoraproject.org/pub/fedora/linux/releases/$releasever/Everything/$basearch/os/
metalink=https://mirrors.fedoraproject.org/metalink?repo=fedora-$releasever&arch=$basearch
enabled=1
[...]
\end{screen}

Interacțiunea cu DNF se face în mod uzual în linia de comandă prin intermediul utilitarului \cmd{dnf}.
La fel ca în cazul Debian, \labelindexref{Listing}{lst:package:dnf} prezintă operații frecvente folosind utilitarul \cmd{dnf}.
Fiecare comandă este prefixată de un comentariu care explică rolul acelei comenzi.

\begin{screen}[caption={Operații frecvente cu dnf},label={lst:package:dnf}]
# Afiseaza pachetele instalate local.
[student@uso:~]$ dnf list installed

# Afiseaza lista de repository-uri folosite.
[student@uso:~]$ dnf repolist all

# Cauta pachete care au in nume sau in descrire sirul "svg".
[student@uso:~]$ dnf search svg

# Instaleaza pachetul inkscape.
[student@uso:~]$ sudo dnf install inkscape

# Afiseaza informatii despre pachetul inkscape (instalat sau nu).
[student@uso:~]$ dnf info inkscape

# Dezinstaleaza pachetul inkscape.
[student@uso:~]$ sudo dnf remove inkscape

# Actualizeaza pachetul inkscape.
[student@uso:~]$ sudo dnf update inkscape

# Actualizeaza pachetele sistemului.
[student@uso:~]$ sudo dnf upgrade

# Descarca pachetul inkscape fara sa fie instalat.
[student@uso:~]$ sudo dnf download inkscape

# Descarca sursele pachetului inkscape.
[student@uso:~]$ dnf download --source inkscape
\end{screen}

Pentru operațiile din spate, de lucru cu pachete, utilitarul \cmd{dnf} folosește utilitarul \cmd{rpm}.
Utilitarul \cmd{rpm} are informații despre pachet și despre instalările din sistem.
\labelindexref{Listing}{lst:package:rpm} prezintă operații frecvente folosind utilitarul \cmd{rpm}.
Fiecare comandă este prefixată de un comentariu care explică rolul acelei comenzi.

\begin{screen}[caption={Operații frecvente cu rpm},label={lst:package:rpm}]
# Afiseaza pachetele instalate local.
[student@uso:~]$ rpm -qa

# Afiseaza informatii despre pachetul inkscape.
[student@uso:~]$ rpm -qi inkscape

# Identifica pachetul care contine fisierul /usr/bin/ssh.
[student@uso:~]$ rpm -qf /usr/bin/ssh

# Afiseaza continutul pachetului inkscape, instalat local.
[student@uso:~]$ rpm -ql inkscape
\end{screen}

\subsection{Gestiunea pachetelor în alte formate}
\label{sec:package:other}

În continuare prezentăm în format tabelar comenzi comparative între diferite sisteme de gestiune a pachetelor descrise în \labelindexref{Tabelul}{tab:package:types}.
Pentru completitudine vom include și APT și DNF.
O coloană va descrie tipul de sistem de gestiune a pachetelor iar alte coloane vor prezenta comenzile corespunzătoare fiecărui tip de operație.
Din limitări de spațiu sunt mai multe tabele cu aceeași primă coloană.
În tabele vom folosi numele de pachet generic \texttt{hello};
când apare cuvântul \texttt{hello} într-o comandă presupunem că e vorba de numele unui pachet.

Sunt prezente următoarele tabele:
\begin{itemize}
  \item \labelindexref{Tabelul}{tab:package:config-repository} prezintă fișierul de configurare pentru sistemul de gestiune a pachetelor și comenzile pentru configurarea acestuia.
  \item \labelindexref{Tabelul}{tab:package:config-update} prezintă comenzile pentru actualizarea informațiilor despre pachete și actualizarea sistemului (\textit{update}).
  \item \labelindexref{Tabelul}{tab:package:config-install} prezintă comenzile pentru instalarea și dezinstalarea pachetelor.
  \item \labelindexref{Tabelul}{tab:package:config-download} prezintă comenzile pentru descărcarea pachetelor (fără instalare) și descărcarea codului sursă al pachetelor.
  \item \labelindexref{Tabelul}{tab:package:config-list} prezintă comenzile pentru afișarea de informații și conținut al pachetelor.
\end{itemize}

\begin{table}[!htb]
  \scriptsize
  \begin{center}
    \begin{tabular}{ p{0.15\textwidth} p{0.4\textwidth} p{0.25\textwidth} }
      \toprule
        \textbf{Sistem de pachete} &
        \textbf{Fișier / director de configurare} &
        \textbf{Utilitar(e) de configurare} \\
      \midrule
        APT &
        \file{/etc/apt/sources.list} &
        \cmd{apt}, \cmd{dpkg} \\
      \midrule
        DNF / YUM &
        \file{/etc/dnf/dnf.conf}, \file{/etc/yum.repos.d/} &
        \cmd{dnf}, \cmd{rpm} \\
      \midrule
        Pacman &
        \file{/usr/local/etc/pacman.conf} &
        \cmd{pacman} \\
      \midrule
        Portage &
        \file{/etc/portage} &
        \cmd{emerge} \\
      \midrule
        ZYpp &
        \file{/etc/zypp/zypp.conf} &
        \cmd{zypper}, \cmd{rpm} \\
      \midrule
        Port &
        \file{/etc/portsnap.conf} &
        \cmd{portsnap}, \cmd{pkg} \\
      \midrule
        Homebrew &
        N/A &
        \cmd{brew} \\
      \midrule
        MacPorts &
        \file{/etc/macaports/} &
        \cmd{port} \\
      \midrule
        Chocolatey &
        se folosește \cmd{choco config} &
        \cmd{choco} \\
      \bottomrule
    \end{tabular}
  \end{center}
  \caption{Configurare repository}
  \label{tab:package:config-repository}
\end{table}

\begin{table}[!htb]
  \scriptsize
  \begin{center}
    \begin{tabular}{ p{0.15\textwidth} p{0.3\textwidth} p{0.33\textwidth} }
      \toprule
        \textbf{Sistem de pachete} &
        \textbf{Actualizare cache / versiuni} &
        \textbf{Upgrade sistem / pachete} \\
      \midrule
        APT &
        \cmd{apt update} &
        \cmd{apt upgrade} \\
      \midrule
        DNF / YUM &
        \cmd{dnf makecache} &
        \cmd{dnf update} \\
      \midrule
        Pacman &
        \cmd{pacman -Fy} &
        \cmd{pacman -Syu} \\
      \midrule
        Portage &
        \cmd{emerge -{}-sync} &
        \cmd{emerge -{}-update -{}-deep world} \\
      \midrule
        ZYpp &
        N/A &
        \cmd{zypper update} \\
      \midrule
        Port &
        \cmd{portsnap fetch update} &
        \cmd{pkg upgrade} \\
      \midrule
        Homebrew &
        \cmd{brew update} &
        \cmd{brew upgrade} \\
      \midrule
        MacPorts &
        \cmd{port selfupdate}, \cmd{port sync} &
        \cmd{port upgrade outdated} \\
      \midrule
        Chocolatey &
        N/A &
        \cmd{choco upgrade all} \\
      \bottomrule
    \end{tabular}
  \end{center}
  \caption{Actualizare / Upgrade}
  \label{tab:package:config-update}
\end{table}

\begin{table}[!htb]
  \scriptsize
  \begin{center}
    \begin{tabular}{ p{0.15\textwidth} p{0.25\textwidth} p{0.2\textwidth} p{0.2\textwidth} }
      \toprule
        \textbf{Sistem de pachete} &
        \textbf{Căutare} &
        \textbf{Instalare} &
        \textbf{Dezinstalare} \\
      \midrule
        APT &
        \cmd{apt search hello} &
        \cmd{apt install hello} &
        \cmd{apt remove hello} \\
      \midrule
        DNF / YUM &
        \cmd{dnf search hello} &
        \cmd{dnf install hello} &
        \cmd{dnf remove hello} \\
      \midrule
        Pacman &
        \cmd{pacman -Ss hello} &
        \cmd{pacman -S hello} &
        \cmd{pacman -R hello} \\
      \midrule
        Portage &
        \cmd{emerge -{}-search hello} &
        \cmd{emerge hello} &
        \cmd{emerge -C hello} \\
      \midrule
        ZYpp &
        \cmd{zypper search hello} &
        \cmd{zypper install hello} &
        \cmd{zypper remove hello} \\
      \midrule
        Port &
        \cmd{pkg search hello} &
        \cmd{pkg install hello} &
        \cmd{pkg remove hello} \\
      \midrule
        Homebrew &
        \cmd{brew search hello} &
        \cmd{brew install hello} &
        \cmd{brew uninstall hello} \\
      \midrule
        MacPorts &
        \cmd{port search hello} &
        \cmd{port install hello} &
        \cmd{port uninstall hello} \\
      \midrule
        Chocolatey &
        \cmd{choco search hello} &
        \cmd{choco install hello} &
        \cmd{choco uninstall hello} \\
      \bottomrule
    \end{tabular}
  \end{center}
  \caption{Instalare / Dezinstalare}
  \label{tab:package:config-install}
\end{table}

\begin{table}[!htb]
  \scriptsize
  \begin{center}
    \begin{tabular}{ p{0.15\textwidth} p{0.4\textwidth} p{0.35\textwidth} }
      \toprule
        \textbf{Sistem de pachete} &
        \textbf{Descărcare (fără instalare)} &
        \textbf{Descărcare sursă pachet} \\
      \midrule
        APT &
        \cmd{apt install -d hello} &
        \cmd{apt source hello} \\
      \midrule
        DNF / YUM &
        \cmd{dnf download hello} &
        \cmd{dnf download -{}-sourcehello} \\
      \midrule
        Pacman &
        \cmd{pacman -Sw hello} &
        \cmd{asp export hello} \\
      \midrule
        Portage &
        \cmd{emerge -{}-fetchnly hello} &
        N/A \\
      \midrule
        ZYpp &
        \cmd{zypper install -{}-download-only hello} &
        \cmd{zypper source-install hello} \\
      \midrule
        Port &
        \cmd{pkg fetch hello} &
        \cmd{pkg fetch hello} \\
      \midrule
        Homebrew &
        N/A &
        \cmd{brew fetch hello} \\
      \midrule
        MacPorts &
        N/A &
        \cmd{port fetch hello} \\
      \midrule
        Chocolatey &
        \cmd{choco download hello} &
        N/A \\
      \bottomrule
    \end{tabular}
  \end{center}
  \caption{Descărcare pachet / sursă}
  \label{tab:package:config-download}
\end{table}

\begin{table}[!htb]
  \scriptsize
  \begin{center}
    \begin{tabular}{ p{0.15\textwidth} p{0.2\textwidth} p{0.2\textwidth} p{0.2\textwidth} }
      \toprule
        \textbf{Sistem de pachete} &
        \textbf{Căutare pachet local} &
        \textbf{Conținut pachet instalat} &
        \textbf{Pachet al unui fișier existent} \\
      \midrule
        APT &
        \cmd{dpkg -l hello} &
        \cmd{dpkg -L hello} &
        \cmd{dpkg -S /usr/bin/hello} \\
      \midrule
        DNF / YUM &
        \cmd{rpm -qi hello} &
        \cmd{rpm -ql hello} &
        \cmd{rpm -qf /usr/bin/hello} \\
      \midrule
        Pacman &
        \cmd{packman -Qs hello} &
        \cmd{packman -Ql hello} &
        \cmd{packman -Qo hello} \\
      \midrule
        Portage &
        N/A &
        \cmd{emerge files hello} &
        \cmd{emerge belongs /usr/bin/hello} \\
      \midrule
        ZYpp &
        \cmd{rpm -qi hello} &
        \cmd{rpm -ql hello} &
        \cmd{rpm -qf /usr/bin/hello} \\
      \midrule
        Port &
        \cmd{pkg info hello} &
        \cmd{pkg info -l hello} &
        \cmd{pkg provides /usr/bin/hello} \\
      \midrule
        Homebrew &
        \cmd{brew ls | grep -i hello} &
        \cmd{brew ls hello} &
        N/A \\
      \midrule
        MacPorts &
        \cmd{port list} &
        \cmd{port contents hello} &
        \cmd{port provides /usr/bin/hello} \\
      \midrule
        Chocolatey &
        \cmd{choco search -{}-local-only hello} &
        N/A &
        N/A \\
      \bottomrule
    \end{tabular}
  \end{center}
  \caption{Căutare / Listare}
  \label{tab:package:config-list}
\end{table}

\section{Sisteme de distribuție cu tot cu sistem}
\label{sec:package:all-in-one}

Atunci când instalăm un pachet pe un sistem trebuie să avem grijă să fie compatibil cu celelalte pachete, adică să fie versiuni de pachete care pot lucra împreună.
În unele situații avem nevoie de o anumită versiune, lucru imposibil de realizat câtă vreme versiunile altor pachete sunt incompatibile.
Ne interesează să avem un mediu relativ izolat în care să avem instalat pachetul: o ierarhie în care pachetul să fie instalat împreună cu dependențele sale fără legătură cu ce este deja instalat la nivelul sistemului.

Această abordare de instalare numită \textit{all-in-one} folosește un alt sistem de gestiune decât cel al sistemului.
Sistemul de gestiune a pachetelor sistemului este la nivelul global al sistemului;
soluțiile all-in-one lucrează independent de acesta.
Soluțiile all-in-one permit actualizarea (\textit{update}) mai facilă a pachetelor prin actualizarea directorului în care sunt instalate.
Suplimentar permit configurări de securitate care pot fi realizate granular, doar la nivelul pachetului în cauză, nu la nivelul sistemului.

Un caz de utilizare pentru aceste sisteme de gestiune a pachetelor sunt soluțiile IoT (Internet of Things) despre care vom discuta în \labelindexref{Capitolul}{ch:embed}.
Aceasta pentru că sistemele IoT au de obicei roluri foarte bine definite și dorim să instalăm ușor o aplicație software cu toate dependențele sale.

În aceste sisteme există un pachet (o arhivă) care conține toate informațiile necesare rulării aplicație: aplicația, depdendențe, metadate, scripturi de instalare și configurare.

Sistemele Ubuntu folosesc Snapcraft\footnote{\url{https://snapcraft.io/}}, o soluție care înglobează aplicațiile în pachete;
un pachet se numește \textbf{snap}.
Un serviciu local numit \cmd{snapd} este responsabil de instalarea unei aplicații folosind utilitarul \cmd{snap}.
\labelindexref{Listing}{lst:package:snap-list} prezintă aplicațiile instalate implicit în format \textbf{snap} pe Ubuntu 18.04.

\begin{screen}[caption={Pachete instalate în format snap pe Ubuntu 18.04},label={lst:package:snap-list}]
student@uso:~$ snap list
Name                  Version                     Rev   Tracking         Publisher   Notes
core                  16-2.45.3.1                 9804  latest/stable    canonical   core
core18                20200724                    1885  latest/stable    canonical   base
gnome-3-26-1604       3.26.0.20200529             100   latest/stable/   canonical   -
gnome-3-28-1804       3.28.0-17-gde3d74c.de3d74c  128   latest/stable    canonical   -
gnome-3-34-1804       0+git.3009fc7               36    latest/stable    canonical   -
gnome-calculator      3.36.0+git9.96b95fd2        748   latest/stable/   canonical   -
gnome-characters      v3.34.0+git8.a46106b        550   latest/stable/   canonical   -
gnome-logs            3.34.0                      100   latest/stable/   canonical   -
gnome-system-monitor  3.36.0-12-g35f88a56d7       148   latest/stable/   canonical   -
gtk-common-themes     0.1-36-gc75f853             1506  latest/stable/   canonical   -
\end{screen}

Pe un sistem pot coexista pachete instalate folosind managerul de pachete al sistemului și pachete snap.

\labelindexref{Listing}{lst:package:snap} prezintă operații uzuale folosind utilitarul \cmd{snap}.

\begin{screen}[caption={Operații uzule cu snap},label={lst:package:snap}]
# Cauta pachete care au in nume sau in descrire sirul "svg".
student@uso:~$ snap find svg

# Instaleaza pachetul inkscape.
student@uso:~$ sudo snap install inkscape

# Afiseaza informatii despre pachetul inkscape (instalat sau nu).
student@uso:~$ snap info inkscape

# Dezinstaleaza pachetul inkscape.
student@uso:~$ sudo snap remove inkscape

# Actualizeaza pachetul inkscape.
student@uso:~$ sudo snap refresh inkscape

# Actualizeaza pachetele sistemului.
student@uso:~$ sudo snap refresh

# Descarca pachetul inkscape fara sa fie instalat.
student@uso:~$ snap download inkscape
\end{screen}

Alte soluții de tip all-in-one sunt Flatpak\footnote{\url{https://www.flatpak.org/}} și AppImage\footnote{\url{https://appimage.org/}}.
Flatpak creează un mediu izolat folosind o formă de virtualizare de aplicație, similar unui container, descris în \labelindexref{Secțiunea}{sec:vm:concepts:containers}.
În acest mediu izolat rulează aplicații.
AppImage este o soluție care permite instalarea unei aplicații fără a fi nevoie de cont privilegiat;
există aplicații care sunt descărcabile în format de pachet AppImage și instalabile astfel de un utilizator neprivilegiat.

Un avantaj al folosirii soluțiilor all-in-one este că folosesc versiuni mai recente ale aplicațiilor.
În vreme ce aplicațiile care folosesc managerul de pachete al sistemului vor folosi versiunea disponibilă în repository, care ține de versiunea curentă a distribuției.

\subsection{Medii specifice pentru limbaje de programare}
\label{sec:package:specific}

Limbaje de programare precum Python sau Ruby permit dezvoltarea de aplicații în aceste limbaje și distribuirea acestora ca pachete.
Pentru aceasta cele două limbaje au sisteme de gestiune ale pachetelor: \cmd{pip} pentru Python și \cmd{gem} pentru Ruby.

Sistemele de gestiune a pachetelor pentru limbaje de programare sunt independente de sistemele de gestiune a pachetelor la nivel de sistem.
Folosesc în mod similar sistemul de fișiere, dar se ocupă de gestiunea acestor noi pachete.

Pachetele specifice Python sau Ruby se pot găsi însă și în repository-ul sistemului de gestiune a pachetelor la nivelul sistemului.
Astfel că, dacă se instalează pachete în ambele sisteme (cel al sistemului și \cmd{pip} sau \cmd{gem}), pot apărea conflicte.

În mod ideal, vom folosi doar pachetele de la nivelul sistemului.
Dar sunt situații în care ne dorim să folosim \cmd{pip} sau \cmd{gem}:

\begin{itemize}
  \item dorim să instalăm pachetele într-o zonă locală (un mediu virtual) pentru o aplicație, fără a fi afectate de pachetele din sistem
  \item pachetele respective nu se găsesc în repsitory-ul sistemului de gestiune a pachetelor
  \item pachetele de la nivelul sistemului au versiuni nepotrivite (prea vechi)
  \item dorim să instalăm pachetele doar la nivelul utilizatorului curent, nu la nivelul sistemului
\end{itemize}

Pentru a instala \cmd{pip} și \cmd{gem} pe un sistem Ubuntu folosim comenzile din \labelindexref{Listing}{lst:package:install-pip-gem}.

\begin{screen}[caption={Instalarea Python pip și Ruby gem pe Ubuntu},label={lst:package:install-pip-gem}]
student@uso:~$ sudo apt install python3-pip python-dev

student@uso:~$ sudo apt install rubygems
\end{screen}

\labelindexref{Listing}{lst:package:pip} conține exemple de folosire a utilitarului \cmd{pip} iar \labelindexref{Listing}{lst:package:gem} conține exemple de folosire a utilitarului \cmd{gem} pentru căutarea, instalarea, listarea și dezinstalarea de pachete.

\begin{screen}[caption={Gestiunea pachetelor folosind pip},label={lst:package:pip}]
# Afiseaza pachetele instalate local.
student@uso:~$ pip3 list

# Cauta pachete care au in nume sau in descrire sirul "pwn".
student@uso:~$ pip3 search pwn

# Instaleaza pachetul pwntools.
student@uso:~$ sudo pip3 install pwntools

# Afiseaza informatii despre pachetul pwntools (instalat sau nu).
student@uso:~$ pip3 show pwntools

# Dezinstaleaza pachetul pwntools.
student@uso:~$ sudo pip3 uninstall pwntools

# Actualizeaza pachetul pwntools.
student@uso:~$ sudo pip3 install -U pwntools

# Actualizeaza pachetele sistemului.
student@uso:~$ sudo pip3 install -U

# Descarca pachetul pwntools fara sa fie instalat.
student@uso:~$ pip3 install -d pwntools

# Instaleaza pachetul pwntools doar pentru utilizatorul curent.
student@uso:~$ pip3 install --user pwntools
\end{screen}

\begin{screen}[caption={Gestiunea pachetelor folosind gem},label={lst:package:gem}]
# Afiseaza pachetele instalate local.
student@uso:~$ gem list

# Cauta pachete care au in nume sau in descrire sirul "rake".
student@uso:~$ gem search pwn

# Instaleaza pachetul rake.
student@uso:~$ sudo gem install rake

# Afiseaza informatii despre pachetul rake (instalat sau nu).
student@uso:~$ gem list -d rake

# Dezinstaleaza pachetul rake.
student@uso:~$ sudo gem uninstall rake

# Actualizeaza pachetul rake.
student@uso:~$ sudo gem update rake

# Actualizeaza pachetele sistemului.
student@uso:~$ sudo gem update

# Descarca pachetul rake fara sa fie instalat.
student@uso:~$ gem fetch rake

# Instaleaza pachetul rake doar pentru utilizatorul curent.
student@uso:~$ gem install --user-install rake
\end{screen}

Pentru a preveni conflicte, sau pentru a lucra la un proiect specific în Python sau Ruby, se pot crea medii specifice unde se instalează pachete folosind \cmd{pip} sau \cmd{gem}.
Aceste medii se numesc medii virtuale și sunt similare celor create de sisteme precum Snapcraft sau Flatpack;
aceste medii se mai numesc și \textit{sandbox}-uri.
Aceste medii sunt create cu ajutorul utilitarelor \cmd{virtualenv} pentru Python și \cmd{bundler}.
Mediile create folosind \cmd{virtualenv} și \cmd{bundler} sunt îndeosebi utile dezvoltatorilor în Python și Ruby.
Aceștia vor crea un astfel de mediu izolat de lucru, în care vor instala versiunile de pachete necesare pentru dezvoltarea aplicației lor.

Pentru a fi disponibile aceste utilitare, trebuie să instalăm utilitarele respective.
Putem opta pentru a le instala chiar folosind \cmd{pip} și \cmd{gem} ca în \labelindexref{Listing}{lst:package:install-venv-bundler}.

\begin{screen}[caption={Instalarea utilitarelor virtualenv și rvm},label={lst:package:install-venv-bundler}]
student@uso:~$ sudo pip3 install virtualenv

student@uso:~$ sudo gem install bundler
\end{screen}

\labelindexref{Listing}{lst:package:python-env-create} conține secvențe de comenzi pentru crearea unui mediu virtual Python și instalarea unui pachet.
Aceste pachete sunt instalate doar în mediul virtual din directorul \file{myenv/} fără a fi afectate de pachetele instalate în sistem.

\begin{screen}[caption={Crearea unui mediu virtual (Python și Ruby)},label={lst:package:python-env-create}]
student@uso:~$ virtualenv myenv

student@uso:~$ cd myenv/
student@uso:~/myenv$ ls -F
bin/  lib/  pyvenv.cfg

student@uso:~/myenv$ source bin/activate

(myenv) student@uso:~/myenv$ pip3 install grip
Collecting grip
  Downloading grip-4.5.2.tar.gz (145 kB)
[...]

(myenv) student@uso:~/myenv$ ls bin/grip
bin/grip

(myenv) student@uso:~/myenv$ grip
[...]

(myenv) student@uso:~/myenv$ deactivate
student@uso:~/myenv$
\end{screen}

O soluție similară pentru JavaScript (Node.js\footnote{\url{https://nodejs.org/en/}}) folosește utilitarul \cmd{npm}\footnote{\url{https://www.npmjs.com/}} ca sistem de gestiune a pachetelor și \cmd{nvm}\footnote{\url{https://github.com/nvm-sh/nvm}} pentru crearea unui mediu izolat.

\section{Anexă: Instalarea unui pachet din surse}
\label{sec:package:source}

Modul uzual de a obține o aplicație este prin intermediul unui pachet software și a unui sistem de gestiune a pachetelor ca mai sus.
Aceste pachete conțin de obicei executabile și cod compilat pentru a face cât mai rapid procesul de compilare.

Cu toate acestea, există situații în care suntem interesați de obținerea codului sursă al unei aplicații și compilarea acesteia și instalarea manuală din codul compilat.
În mod evident, doar pentru aplicațiile open source, la care avem acces la codul sursă.
În trecut, această formă era frecventă pentru că distribuțiile erau puține și nu aveau toate pachetele;
singurul mod de a instala pachetul era obținând sursele și compilându-l.
În zilele noastre, nu mai este cazul;
majoritatea distribuțiilor au foarte multe pachete în depozitele lor de pachete.
Rămân însă câteva motive pentru care ne-am dori instalarea din surse:

\begin{itemize}
  \item aplicația este suficient de obscură încât să nu se găsească în depozitul de pachete al distribuției;
  \item dorim să edităm codul sursă al aplicației pentru a adăuga anumite funcționalități punctuale nevoilor noastre;
    acest scenariu e mai degrabă util unui dezvoltator, nu unui utilizator obișnuit;
  \item aplicația este în dezvoltare, nu este încă lansată și nu există pachete;
    acest scenariu este de asemenea util unui dezvoltator;
  \item vrem să compilăm aplicația într-un anumit mod, folosind funcționalități de compilare care țin cont de particularități ale sistemului (de exemplu arhitectura procesorului sau componente hardware specifice).
\end{itemize}

Instalarea din codul sursă presupune obținerea unei arhive care are codul sursă al aplicației sau clonarea repository-ului sursă al aplicației.
În cadrul codului sursă astfel obținut se regăsesc în mod uzual un fișier \file{README} cu indicații de instalare și scripturi sau componente de compilare.

În general pentru compilare se folosesc sisteme de build care automatizează procesul.
Acestea pot fi Make, Cmake, Scons, Maven, Ant, Gradle depinzând de limbajul de programare folosit și de preferințele dezvoltatorului.
Mai multe informații despre sistemele de build se găsesc în \labelindexref{Capitolul}{ch:appdev}.

În cazul aplicațiilor scrise în limbajele C/C++ cel mai frecvent sistem de build este cel bazat pe Make, împreună cu utilitarele complementare.
Acest sistem se numește Autotool, sau sistemul de build GNU.

Ca exaemplu, \labelindexref{Listing}{lst:package:src-install} conține secvența de comenzi care duce la instalarea OpenSC\footnote{\url{https://github.com/OpenSC/OpenSC/wiki}}, un set de aplicații și biblioteci pentru a lucra cu smart carduri.

\begin{screen}[caption={Instalarea OpenSC din surse},label={lst:package:src-install}]
student@uso:~$ cd Downloads/
student@uso:~/Downloads$ wget https://github.com/OpenSC/OpenSC/releases/download/0.20.0/opensc-0.20.0.tar.gz
[...]
Saving to: 'opensc-0.20.0.tar.gz'

student@uso:~/Downloads$ tar xzf opensc-0.20.0.tar.gz
student@uso:~/Downloads$ cd opensc-0.20.0/
student@uso:~/Downloads/opensc-0.20.0$ ls
aclocal.m4  bootstrap.ci  config.guess  config.sub  configure.ac  depcomp  etc         ltmain.sh  MacOSX       Makefile.in   missing  packaging  src          tests          win32
bootstrap   compile       config.h.in   configure   COPYING       doc      install-sh  m4         Makefile.am  Makefile.mak  NEWS     README     test-driver  version.m4.ci

student@uso:~/Downloads/opensc-0.20.0$ ./configure
[...]
configure: error: winscard.h is required for pcsc

student@uso:~/Downloads/opensc-0.20.0$ apt-file search winscard.h
libpcsclite-dev: /usr/include/PCSC/winscard.h
libwine-dev: /usr/include/wine/windows/winscard.h
libwine-development-dev: /usr/include/wine-development/wine/windows/winscard.h
mingw-w64-common: /usr/share/mingw-w64/include/winscard.h
mingw-w64-i686-dev: /usr/i686-w64-mingw32/include/winscard.h
mingw-w64-x86-64-dev: /usr/x86_64-w64-mingw32/include/winscard.h

student@uso:~/Downloads/opensc-0.20.0$ sudo apt install libpcsclite-dev
[...]

student@uso:~/Downloads/opensc-0.20.0$ ./configure
[..]
configure: creating ./config.status
config.status: creating Makefile
[...]

student@uso:~/Downloads/opensc-0.20.0$ ls
aclocal.m4    compile       config.h.in    config.sub    COPYING  etc         ltmain.sh  Makefile     Makefile.mak  packaging  stamp-h1     version.m4.ci
bootstrap     config.guess  config.log     configure     depcomp  install-sh  m4         Makefile.am  missing       README     test-driver  win32
bootstrap.ci  config.h      config.status  configure.ac  doc      libtool     MacOSX     Makefile.in  NEWS          src        tests

student@uso:~/Downloads/opensc-0.20.0$ make
make  all-recursive
make[1]: Entering directory '/home/student/Downloads/opensc-0.20.0'
Making all in etc
  [...]
  CC       compat_getopt_main.o
  CC       compat_dummy.lo
  CC       compat_strlcat.lo
  [...]

student@uso:~/Downloads/opensc-0.20.0$ ls src/tools/
cardos-tool                  goid-tool.c                    openpgp-tool.c                         opensc-notify.plist.in                     pkcs15-crypt
[...]
student@uso:~/Downloads/opensc-0.20.0$ sudo make install
[...]
student@uso:~/Downloads/opensc-0.20.0$ sudo ldconfig
student@uso:~/Downloads/opensc-0.20.0$ cardos-tool
No smart card readers found.
student@uso:~/Downloads/opensc-0.20.0$ pkcs15-tool
Usage: pkcs15-tool [OPTIONS]
Options:
      --version                 Print OpenSC package version
      --list-info               List card information
\end{screen}

În secvența de comenzi pentru instalarea din surse de mai sus, am urmat pașii:
\begin{enumerate}
  \item liniile \texttt{1-X}: am descărcat arhiva ce conține sursele OpenSC
  \item liniile \texttt{X-Y}: am despachetat arhiva
  \item liniile \texttt{X-Y}: am văzut că nu am avem fișier \file{Makefile} dar există fișierul \file{configure}
  \item liniile \texttt{X-Y}: am rulat scriptul \file{configure};
    am observat că nu există în sistemul fișierul header \file{winscard.h}
  \item liniile \texttt{X-Y}: am căutat fișierul header \file{winscard.h} în pachete folosind \cmd{apt-file};
    am descoperit pachetul \texttt{libpcslite-dev} pe care l-am instalat
  \item liniile \texttt{X-Y}: am rulat din nou scriptul \file{configure};
    acum a rulat cu succes
  \item liniile \texttt{X-Y}: am observat prezența fișierului \file{Makefile} generat din rularea cu succes a scriptului \file{configure}
  \item liniile \texttt{X-Y}: am rulat comanda \cmd{make} pentru a compila sursele și a obține fișierele executabile și bibliotecile dorite
  \item liniile \texttt{X-Y}: am observat existența fișierelor executabile dorite în directorul \file{src/tools/}, precum \file{cardos-tool} sau \file{pkcs15-crypt}
  \item liniile \texttt{X-Y}: am instalat pachetele în sistem folosind comanda \cmd{make install} și actualizând cacheul de biblioteci (\cmd{ldconfig}), astfel că executabilele generate (\cmd{cardos-tool}, \cmd{pkcs15-tool} sunt acum folosibile)
\end{enumerate}

După cum se observă, compilarea din surse necesită cunoștințe tehnice mai avansate legate de sisteme de build, despre care vom discuta în \labelindexref{Capitolul}{ch:appdev}.
De aceea sunt de evitat pentru majoritatea utilizatorilor.
Pentru aproape toate scenariile acestora, sistemele de gestiune ale pachetelor prezentate mai sus vor fi suficiente.

\section{Sumar}
\label{sec:package:summary}

Un dezvoltator sau firmă de dezvoltare are nevoie de o formă de distribuție a aplicației create pentru a ajunge la utilizatori.
Uzual este în formă de pachete software.

Pachetele conțin date și metadate care sunt prelucrate de un sistem de gestiune a pachetelor pentru instalare, dezinstalare.

Sistemele de gestiune sunt aplicații care cunosc formatul pachetelor.
Exemplu sunt APT sau YUM pe Linux.

Pachetele se găsesc într-un depozit (\textit{repository}) sau magazin de apicații de unde aplicația de gestiune a pachetelor le obține.
Pachetul este obținut dintr-un repository și instalat.

Gestiunea pachetelor poate fi dificilă, pot fi versiuni incompatibile.
O soluție este folosirea sistemelor \textit{all-in-one}.
