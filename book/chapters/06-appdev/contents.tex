\chapter{Dezvoltarea aplicațiilor}
\label{ch:appdev}

\section{Introducere}
\label{sec:appdev:intro}

În momentul de față avem o varietate mare de dispozitive electronice cu care interacționăm: de la sisteme complexe, cum ar fi calculatoarele personale sau laptop-urile, la telefoanele mobile sau chiar ceasurile inteligente pe care le folosim intens zilnic.
Deși foarte diferite, toate aceste sisteme electronice au un lucru în comun: toate rulează software.

Având în vedere marea varietate a sistemelor fizice pe care rulează aplicațiile pe care le dezvoltăm, este foarte important să înțelegem toate aspectele dezvoltării unui program și toate opțiunile pe care le avem la dispoziție în acest proces.
Astfel, putem face alegerile corecte pentru a dezvolta eficient o aplicație adaptată dispozitivului fizic pe care va rula.

De exemplu, dacă dezvoltăm o aplicație ce va rula pe un ceas inteligent, trebuie să ținem cont că avem resurse limitate comparativ cu o aplicație ce va rula pe un sistem desktop.
În cazul unui ceas inteligent, avem mai puțină memorie și putere de procesare și, mai mult, trebuie să ținem cont de consumul de energie pe care îl are dispozitivul.

Alegerea unui limbaj potrivit pentru un program este esențială pentru succesul acestuia.
În funcție de tipul aplicației, trebuie ales limbajul adecvat având în vedere proprietăți precum: viteza de execuție, ușurința de scriere a codului, numărul de biblioteci existente, utilitarele existente, portabilitatea și comunitatea din jurul limbajului.

\section{Limbaje de programare}
\label{sec:appdev:lang}

Un \textbf{limbaj de programare} definește un set de reguli folosite pentru a formula instrucțiuni pe care calculatorul să le execute.
Regulile limbajului constau în cuvinte cheie, instrucțiuni care pot fi folosite și gramatica acestora, care se mai numește și sintaxă.

Folosind un limbaj de programare, putem defini operații pe care dorim să le executăm într-o anumită ordine.
Acesta este practic programul pe care îl scriem.
Operațiile sunt rulate pe procesorul sistemului pe care dorim să îl programăm, dar înainte de a fi executate, ele trec printr-un proces de transformare.
Fiecare procesor cunoaște un anumit set de instrucțiuni pe care le poate executa.
Codul scris folosind aceste instrucțiuni se numește \textbf{cod mașină} și este singurul limbaj pe care un procesor îl înțelege și îl execută.
Prin urmare, programul scris de noi într-un limbaj de programare este transformat în cod mașină pentru a putea fi înțeles și executat.

La început, calculatoarele erau niște mașini cu capacitate de calcul mult mai redusă, care puteau executa operații simple, în general stocate pe benzi magnetice sau cartele perforate, ca în \labelindexref{Figura}{fig:appdev:card}\footnote{\url{https://en.wikipedia.org/wiki/Computer_programming_in_the_punched_card_era\#/media/File:FortranCardPROJ039.agr.jpg} (CC BY-SA 2.5)}.
Odată cu evoluția procesoarelor și a mediilor de stocare, a apărut limbajul de asamblare, folosit pentru a scrie programe mai complexe.
\textbf{Limbajul de asamblare} este un limbaj specific fiecărui procesor și are în general o corespondență de aproximativ unu-la-unu între instrucțiunile de asamblare și cele are procesorului.
Practic, putem considera că fiecare instrucțiune a limbajului de asamblare este tradusă printr-o altă instrucțiune a procesorului.
Codul sursă scris în limbaj de asamblare este transformat în cod mașină de un asamblor.

\begin{figure}[!htbp]
  \centering
  \includegraphics[width=15cm]{chapters/06-appdev/img/fortran-card-img.png}
  \caption{Program scris pe cartelă perforată}
  \label{fig:appdev:card}
\end{figure}

În general, producătorul unui procesor definește și limbajul de asamblare specific acestuia, pentru a oferi o variantă lizibilă a instrucțiunilor procesorului.
Scopul limbajului de asamblare a fost de a permite scrierea de aplicații într-o manieră ușoară, la început acesta fiind singura modalitate de a dezvolta programe.
Deși mai ușor de folosit decât codul binar, și limbajul de asamblare este dificil de urmărit mai ales pentru programe lungi.
De aceea, în timp, au apărut limbajele mai complexe din punct de vedere al logicii, limbaje ce folosesc comenzi apropiate ca structură de limbajul natural, și care sunt mai ușor de urmărit.
Acestea permit programatorilor să se concentreze pe logica aplicației, nu pe limbajul efectiv și implementarea acestuia.
Ele se numesc \textbf{limbaje de nivel înalt}.

În prezent, majoritatea aplicațiilor sunt dezvoltate folosind limbaje de programare de nivel înalt, care sunt mult mai ușor de “citit” de către programatori.
Totuși, pentru dezvoltarea de aplicații care controlează dispozitive hardware și care trebuie să fie foarte eficiente în folosirea resurselor (ex. drivere) se folosește uneori limbajul de asamblare, deși cazurile acestea sunt destul de rare.

În mare parte, limbajele de programare folosite în zilele noastre sunt limbaje de nivel înalt.
Astfel, programatorii folosesc o sintaxă apropiată de limbajul natural pentru dezvoltarea aplicațiilor, ceea ce ușurează întregul proces (se folosesc cuvinte precum \textit{while}, \textit{if}, \textit{else}).
Atunci când scriem un program, totul pornește de la ideea clară a ceea ce vrem să obținem prin acel program.
Mai exact, trebuie să stabilim clar ce vrem să facă programul.
După aceea, următorul pas este să structurăm ideea și să o \textit{traducem} în limbajul de programare folosit.
Așa obținem programul final ce apoi poate fi rulat, proces sumarizat în \labelindexref{Figura}{fig:appdev:idea-to-program}.

\begin{figure}[htbp]
  \centering
  \def\svgwidth{\columnwidth}
  \includesvg{chapters/06-appdev/img/lang-idea.svg}
  \caption{De la idee la program}
  \label{fig:appdev:idea-to-program}
\end{figure}

Fișierele pe care programatorii le scriu se numesc \textbf{fișiere cod sursă}, ele conțin practic instrucțiuni scrise într-un anume limbaj de programare, de cele mai multe ori, limbaj de nivel înalt.
Codul sursă scris astfel nu poate fi executat pentru că procesorul nu îl înțelege.
Cum am menționat deja, procesorul înțelege doar codul mașină.
Drept urmare, pentru a executa programul, e necesar să transformăm codul sursă în cod mașină.

Procesul de transformare a codului sursă în cod mașină se numește \textbf{compilare}.

Bazat pe modul în care codul de nivel înalt este transformat și rulat pe procesor, putem clasifica limbajele de programare în limbaje compilate sau interpretate.
Diferența între cele două este că pentru a rula un limbaj compilat, acesta este întâi transformat în cod mașină, rezultând un fișier executabil, care apoi poate fi rulat oricând.
Pe de altă parte, un limbaj interpretat este executat direct, prin intermediul unui interpretor.
Interpretorul este cel care ia fiecare instrucțiune, o transformă în cod mașină, iar aceasta este executată imediat.

Fiecare dintre cele două tipuri de a rula cod sursă, compilare sau interpretare, are avantaje și dezavantaje de care este important să ținem cont când alegem limbajul de programare pentru dezvoltarea unei aplicații.

Principalul avantaj al unui limbaj compilat este viteza de execuție.
Odată generat fișierul executabil, acesta va fi rulat de către procesor de oricâte ori dorim, astfel compilarea este un proces separat de execuția programului și are loc o singură dată.
În schimb, în cazul unui limbaj interpretat, la fiecare rulare, codul sursă va fi interpretat linie cu linie de către interpretor, acesta fiind de fapt cel care execută acțiunile programului, proces care e mai lent.

Unele interpretoare mai eficiente vor încerca sa compileze parțial codul sursă și să îl execute direct pe procesor.
Practic, în acest caz, codul sursă se compilează la fiecare rulare.

Pe de altă parte, un limbaj interpretat are mai multă portabilitate.
Cum am precizat deja, codul mașină rezultat în urma compilării poate rula doar pe un anumit tip de procesor.
Deci pentru a rula aceeași aplicație pe arhitecturi (procesoare) diferite, avem nevoie de un fișier executabil pentru fiecare tip de arhitectură.
În aceea ce privește limbajele interpretate, putem folosi aceleași fișiere care sunt transformate de către interpretor în codul mașină specific procesorului.

Majoritatea limbajelor de programare apărute în ultima perioadă sunt considerate limbaje hibride.
Ele îmbină elemente ale ambelor procese, compilare și interpretare, pentru a obține atât portabilitate, cât și viteză în execuție.

Este important de reținut că indiferent de limbaj (compilat, interpretat sau hibrid), procesorul știe să ruleze doar cod mașină, așa că orice cod sursă va fi la un moment dat transformat în cod mașină, iar pentru programatori este important dacă limbajul folosit este unul compilat, interpretat sau hibrid.
Este important să cunoaștem tipul unui limbaj de programare, avantajele și dezavantajele sale pentru a putea alege în funcție de nevoile pe care le avem.

De exemplu, dacă ne dorim să dezvoltăm un joc tip curse de mașini, avem nevoie de rapiditate în execuție, față de o aplicație de organizare a lucrului în echipă, unde ne dorim să putem acoperi cât mai multe platforme.

\section{Limbaje compilate}
\label{sec:appdev:compiled-lang}

Limbajele compilate sunt cele unde codul sursă este transformat în fișiere executabile, ce sunt apoi rulate pe procesor, în urma unui proces care se numește compilare.
Astfel, instrucțiunile specifice limbajului de programare sunt transformate de compilator în instrucțiuni pe care procesorul le înțelege.

În procesul de compilare codul sursă este prelucrat și trece prin mai multe etape intermediare ca în \labelindexref{Figura}{fig:appdev:compile-phases}.

\begin{enumerate}
  \item compilare: codul sursă este transformat în cod în limbaj de asamblare
  \item asamblare: codul în limbaj de asamblare este transformat în cod mașină;
  \item link-editare (sau linking): se fac legăturile către fișiere externe care conțin simboluri sau funcții apelate din fișierul sursă;
dacă avem o funcție definită într-un fișier și folosită în altul, implementarea funcției trebuie legată de apelul ei.
\end{enumerate}

\begin{figure}[htbp]
  \centering
  \def\svgwidth{0.6\columnwidth}
  \includesvg{chapters/06-appdev/img/comp-steps.svg}
  \caption{Etapele compilării}
  \label{fig:appdev:compile-phases}
\end{figure}

În teorie, pentru fiecare limbaj de programare ar trebui să existe un compilator care să cunoască gramatica specifică acestuia, instrucțiunile specifice unui procesor și care să facă transformarea din cod sursă în cod mașină.

Pentru că există un număr foarte mare de limbaje și un număr mare de procesoare diferite, în practică, nu avem câte un compilator diferit.
În general, există framework-uri de compilator care au suport pentru mai multe limbaje de programare (de exemplu C sau C++ sau D), respectiv alte componente care au suport pentru mai multe arhitecturi de procesor (de exemplu x86 sau ARM sau MIPS).
Astfel, un compilator poate genera cod pentru diferite arhitecturi pornind de la diferite tipuri de limbaje.
În general, aceste compilatoare au un limbaj intern, numit limbaj intermediar, care facilitează aceste traduceri din limbaj în cod mașină de procesor, așa cum este indicat în \labelindexref{Figura}{fig:appdev:intermediary}.

\begin{figure}[htbp]
  \centering
  \def\svgwidth{0.6\columnwidth}
  \includesvg{chapters/06-appdev/img/comp-backend-frontend.svg}
  \caption{Traducerea în limbaj intermediar}
  \label{fig:appdev:intermediary}
\end{figure}

\subsection{Exemple de limbaje de programare compilate}
\label{sec:appdev:compiled-lang:ex}

Deși am prezentat avantajele și dezavantajele folosirii unui limbaj de programare compilat, dacă alegem că aceasta este opțiunea potrivită, avem un număr mare de limbaje compilate din care putem alege.
În continuare vom descrie câteva dintre cele mai folosite și particularitățile de care trebuie să ținem cont în luarea unei decizii.

\subsubsection{C}
\label{sec:appdev:compiled-lang:c}

C este un limbaj de programare care a fost dezvoltat în anii 1970 pentru a fi folosit la implementarea sistemului de operare Unix.
Până la apariția limbajului C, sistemele de operare erau scrise în limbaj de asamblare.
C aduce un nivel de abstractizare peste limbajul de asamblare, care ușurează scrierea programelor.

Pentru că limbajul permite accesarea ușoară a resurselor hardware, aplicațiile dezvoltate în C prezintă un risc mai mare de securitate, dacă acestea nu sunt implementate cu atenție.
Din aceleași considerente și probabilitatea de a genera erori în timpul implementării este mai mare.
În plus, C are un nivel de abstractizare foarte redus, adică nu avem la dispoziție funcții care să execute operații complexe, de aceea nu este un limbaj folosit pentru prototipare.
Este recomandat să folosim C dacă stăpânim bine limbajul și nu riscăm să generăm erori și dacă ne dorim o aplicație care să consume cât mai puține resurse.

Preponderent, limbajul C este folosit pentru a dezvolta sisteme de operare și drivere pentru acestea.

\subsubsection{C++}
\label{sec:appdev:compiled-lang:cpp}

C++ este un limbaj de programare de nivel înalt gândit pentru dezvoltarea de aplicații.
Este o adaptare a limbajului C pentru a-l face mai ușor de folosit.
În comparație cu alte limbaje mai noi, C++ este un limbaj mai complex și mai dificil de utilizat.
Totuși, un număr mare de aplicații existente axate pe performanță sunt implementate folosind C++.

\subsection{Pascal}
\label{sec:appdev:compiled-lang:pascal}

Pascal a fost dezvoltat în anii 1960, având ca scop principal dezvoltarea de programe pentru descrierea de algoritmi.
Limbajul a fost gândit pentru educație, pentru a fi utilizat în predarea algoritmilor.

Deși nu este un limbaj pe care să îl luam în calcul în dezvoltarea aplicațiilor, pentru că există alte limbaje mai moderne, poate fi în continuare luat în calcul în scopuri didactice.

\subsection{Alte limbaje}
\label{sec:appdev:compiled-lang:other}

Alte limbaje compilate sunt D, Go (folosit pentru scrierea sistemului de containere Docker) și Rust (folosit de Firefox pentru Servo, motorul de randare a paginilor web).

\section{Limbaje interpretate}
\label{sec:appdev-langs-int}

Limbajele interpretate sunt cele care sunt transformate în cod mașină în timpul execuției.

Pentru a putea rula o aplicație dezvoltată într-un limbaj de programare interpretat, avem nevoie de un program care ia fiecare linie din codul sursă și o transformă în cod mașină.
Acest program se numește interpretor.

Diferența între o aplicație compilată și una interpretată este că, spre deosebire de compilator, care generează un întreg fișier executabil care poate fi apoi rulat, interpretorul ia fiecare linie, generează codul mașină aferent ei, codul este rulat pe procesor, după care operația este reluată.
Deși acest proces crește timpul de execuție al aplicației, sunt avantaje pe care limbajele interpretate le au față de cele compilate.

În primul rând, programele scrise în astfel de limbaje pot fi direct rulate pe orice arhitectură, atât timp cât există un interpretor instalat.
Acest lucru înseamnă că putem copia codul sursă de pe un sistem care rulează Windows pe unul care rulează macOS.
În plus, pentru că toate instrucțiunile trec prin interpretor, limbajele interpretate nu oferă acces direct la resursele hardware, ceea ce le face mai sigure, dar și mai ineficiente.
De aceea astfel de limbaje sunt recomandate pentru dezvoltarea de aplicații care nu țin de sistemul de operare.

Pe de altă parte, limbajele interpretate presupun ca orice mașină care va rula aplicația dezvoltată, trebuie să conțină codul sursă ce va fi rulat.
Sunt cazuri în care nu se dorește publicarea codului sursă ar fi de preferat un limbaj compilat, care permite distribuirea unui executabil.

\subsection{Exemple de limbaje de programare interpretate}
\label{sec:appdev:interpreted-lang:ex}

În prezent utilizarea limbaje de programare pur interpretate este în scădere.
Acestea au ajuns să fie înlocuite de alte limbaje de programare, majoritatea hibride.
Totuși, dacă vom ajunge să lucrăm la aplicații complexe dezvoltate cu ani în urmă, există posibilitatea să ajungem să scriem cod sursă în limbaje precum PHP sau Perl.

PHP \abbrev{PHP}{PHP Hypertext Preprocessor} (\textit{PHP Hypertext Preprocessor}) este un limbaj folosit preponderent pentru dezvoltarea de servere web.
Mai nou este înlocuit de Node.js, dar încă mai putem găsi aplicații web dezvoltate în PHP.

Perl e un alt limbaj interpretat folosit pentru dezvoltare web.
Este considerat predecesorul PHP, deci este tot mai rar întâlnit.

\section{Limbaje hibride}
\label{sec:appdev:hybrid-lang}

Limbajele de programare hibride îmbină elemente de compilare cu elemente de interpretare, pentru a rezulta în aplicații portabile, rapide și sigure.

Pentru că principalul dezavantaj al programelor interpretate este timpul mare de execuție, una dintre îmbunătățirile aduse interpretoarelor este posibilitatea de a transforma codul sursă într-un cod intermediar, care se numește \textbf{bytecode}, care va fi apoi interpretat.
Bytecode-ul este practic codul mașină al interpretorului, deci transformarea acestuia în cod mașină este mult mai rapidă.
Practic, la rularea unui program, este generat un fișier intermediar care conține bytecode, iar interpretorul ia apoi instrucțiuni din bytecode și le transformă în cod mașină.
Putem deci considera că avem două etape în procesul de rulare: compilare către bytecode și apoi interpretare.

Pentru că în acest caz interpretorul are propriul limbaj mașină, acesta nu cunoaște un limbaj de programare anume, ci el transformă bytecode-ul specific în cod mașină, interpretorul se mai numește și \textbf{mașină virtuală} (ex. Java Virtual Machine).
Un avantaj pe care mașina virtuală îl aduce este posibilitatea de a genera bytecode-ul și de a distribui fișiere bytecode în locul codului sursă.
Astfel codul sursă al aplicației nu este făcut public odată cu distribuția aplicației.

Odată generat bytecode-ul, acesta poate fi transformat în cod mașină în două moduri diferite.
Fie se ia tot bytecode-ul și se transformă în cod mașină, după care codul mașină este executat (compilare \textit{ahead-of-time}), fie se iau bucăți din bytecode și se transformă în cod mașină în timpul rulării (compilare JIT\abbrev{JIT}{just-in-time} - \textit{just-in-time}).

Compilarea \textit{ahead-of-time} funcționează ca procesul normal de compilare, doar că se transformă bytecode în cod mașină.
Odată generat codul mașină, acesta este executat.
Avantajul acestei metode este timpul rapid de execuție.
În schimb, timpul de pornire a aplicației crește (timpul necesar pentru a transforma bytecode în cod mașină).

Compilarea \textit{just-in-time} (JIT) presupune ca bucăți de cod care urmează să fie executate să fie transformate în cod mașină în timpul rulării.
Interpretorul analizează blocul de cod care urmează să fie executat și îl transformă în cod mașină.

În plus, interpretorul are anumite mecanisme care îi permit să țină cont de factori precum frecvența de execuție a unui anumit bloc de cod și eficientizează procesul prin stocarea în memorie a codului mașină, pentru a nu transforma același cod de mai multe ori, deci făcând procesul și mai eficient.
Totuși, spre deosebire de compilarea ahead-of-time, JIT are un timp de execuție mai mare, dar timpul de pornire al aplicației scade.

\subsection{Exemple de limbaje de programare hibride}
\label{sec:appdev:hybrid-lang:ex}

În general, limbajele pe care le întâlnim cel mai frecvent astăzi sunt limbaje hibride, precum Java, Python sau JavaScript.
Cum multe din aplicațiile pe care le folosim zi de zi au devenit aplicații web, limbajele orientate spre așa ceva au crescut în popularitate.

\subsubsection{Java}
\label{sec:appdev:hybrid-lang:java}

Java este un limbaj de programare folosit pentru dezvoltarea de aplicații enterprise care a fost dezvoltat de Sun Microsystems.
Oracle a cumpărat Sun Microsystems, așa că astăzi Java este întreținut de către Oracle.
Paradigma pe care se bazează Java este programarea orientată pe obiecte.
Acesta facilitează scrierea unui cod cât mai modularizat și refolosibil.

Ideea de bază care a stat la baza limbajului Java a fost portabilitatea.
Se dorea scrierea unui program o singură dată și rularea lui pe multe platforme diferite fără a face vreo modificare și fără a fi necesară distribuirea sursei.
Inițial a fost folosit în programe de navigare sub forma de \textit{Java Applets}, mici aplicații scrise care se puteau include într-o pagină.

Fiind destinat dezvoltării de aplicații pentru companii, platforma Java este are două componente, compilatorul care generează bytecode-ul (Java Development Kit - JDK\abbrev{JDK}{Java Development Kit}) și mașina care rulează bytecode-ul (Java Runtime Environment - JRE\abbrev{JRE}{Java Runtime Environment}).
Astfel pentru distribuția aplicației nu este necesară distribuția codului sursă.

\subsubsection{C\#}
\label{sec:appdev:hybrid-lang:csharp}

C\# este un limbaj de programare similar cu Java, dezvoltat de Microsoft.
Este principalul limbaj de dezvoltare pentru platforma Windows, fiind integrat inclusiv în motoare de jocuri (de exemplu Unity).

\subsubsection{Python}
\label{sec:appdev:hybrid-lang:python}

Python este un limbaj dezvoltat inițial cu scopul de a fi folosit pentru a învăța programare, principalul avantaj fiind că oferă o lizibilitate mare codului.
A devenit foarte cunoscut datorită unor proiecte precum GNOME (inițial) și OpenStack, precum și datorită bibliotecilor de matematică avansate.

Python este folosit pentru automatizarea administrării sistemului, prin scripturi simple.
Este de asemenea foarte folosit în educație, codul sursă fiind ușor de urmărit.
În plus, Python impune reguli de indentare, ceea ce îl face potrivit ca prim limbaj de programare care obișnuiește elevii cu necesitatea de a indenta codul.
Limbajul mai este folosit în cercetare, find o variantă mai ,,ușoară'' față de alte limbaje științifice cum ar fi Matlab, R sau Julia.

O altă utilizare a limbajului Python este pentru dezvoltarea de servere web, cu ajutorul bibliotecilor Flask și Django.

\subsubsection{JavaScript}
\label{sec:appdev:hybrid-lang:js}

JavaScript este un limbaj folosit pentru dezvoltarea de aplicații web.
Are o sintaxa similară cu limbajele C și Java.
Deși se numește JavaScript, în afară de sintaxa similară, nu are nimic în comun cu Java.

La început JavaScript a fost folosit pentru partea de interfață a aplicațiilor web, interpretorul de JavaScript fiind integrat în browsere.
Odată cu apariția platformei Node.js, aplicațiile JavaScript pot fi rulate și în afara browserului.
Astfel, JavaScript este folosit pentru aplicații simple, pentru automatizarea operațiilor, dar mai ales pentru dezvoltarea de servere web, prin intermediul bibliotecii Express.

În principiu, dacă ne dorim să dezvoltăm o aplicație web, JavaScript, împreună cu Node.js sunt o opțiune importantă de luat în calcul.

\section{Dezvoltarea unui program}
\label{sec:appdev:dev}

Până acum am discutat despre limbajele de programare uzuale și cum acestea se diferențiază între ele și despre cum codul sursă pe care noi îl scriem rulează pe procesor.
Întrebarea pe care urmează să o punem este: \textit{ce presupune dezvoltarea unui program}?

\subsection{Scrierea codului sursă}
\label{sec:appdev:dev-stages:write}

În primul rând, dezvoltarea unui program constă în scrierea efectivă a codului sursă;
practic, acest pas se rezumă la a crea fișiere text și a le edita.
De aceea, aplicația principală de care avem nevoie în dezvoltarea unui program este un \textbf{editor de text}.

În ceea ce privește editoarele de text, avem un număr mare de editoare diferite din care putem alege, pornind de la unele foarte simple care au ca scop editarea unui fișier text, la aplicații complexe adaptate dezvoltării de programe.
În mod normal, orice editor poate fi folosit pentru scrierea de cod, chiar și Notepad din Microsoft Windows, totuși este de preferat să folosim unul adaptat scrierii de cod, pentru a ne ușura munca.

Editoarele adaptate dezvoltării de programe vin cu anumite caracteristici care ne ajută în procesul de dezvoltare cum ar fi indentarea automată a codului, evidențierea cuvintelor cheie, numerotarea liniilor, gruparea secvențelor de cod.

Ținând cont de numărul mare de limbaje de programare existente, fiecare cu particularitățile sale, este imposibil ca un editor să cuprindă toate aspectele fiecărui limbaj.
De aceea, majoritatea acestor editoare permit instalarea de extensii care aduc funcționalități în plus sau aduc funcționalități specifice pentru un anumit limbaj.

În continuare vom descrie cele mai cunoscute editoare folosite pentru scrierea programelor.

\subsection{Editoare în mod text}
\label{sec:appdev:dev-stages:editor}

\subsubsection{Vim}
\label{sec:appdev:dev-stages:editor:vim}

Vim este un editor care funcționează în linie de comandă, fiind versiunea îmbunătățită a \texttt{vi}, editor care se găsește pe toate sistemele Unix.
Este un editor foarte configurabil, dar nu foarte ușor de folosit pentru începători.
Acest lucru este cauzat de lipsa interfeței grafice, tot meniul fiind transpus prin comenzi.
Deși pentru utilizatorii experimentați acest lucru poate reprezenta un avantaj aducând rapiditate în folosire, pentru începători, lipsa cursorului și necesitatea de a memora comenzi îngreunează folosirea editorului.

Chiar dacă există multe variante de editoare mai intuitive, vă recomandam familiarizarea cu Vim, sau chiar cu varianta redusă, vi, tocmai pentru că este unul din puținele editoare disponibile pe orice sistem Unix.

Pentru familiarizarea cu Vim și comenzile de bază, recomandăm parcurgerea tutorialului \cmd{vimtutor}.
Este suficient să rulați comanda \cmd{vimtutor} în terminal și tutorialul va începe.

Deși putem găsi \texttt{vi} pe orice sistem Unix, doar unele distribuții vin cu editorul Vim instalat.
Pentru a instala Vim pe sistemele Debian/Ubuntu, rulăm comanda \cmd{sudo apt install vim}.

\subsubsection{Alte editoare}
\label{sec:appdev:dev:editor:other}

Alte editoare similare cu Vim, care funcționează tot în linie de comandă, sunt Nano, Pico și Emacs.
La fel ca Vim, ele se găsesc frecvent pe sistemele Unix, sunt folosite la editarea rapidă, cum ar fi editarea unor fișiere de configurare.

\subsection{Editoare în interfață grafică}
\label{sec:appdev:dev:editor-gui}

\subsubsection{Sublime Text}
\label{sec:appdev:dev:editor-gui:sublime}

Sublime Text este un editor de cod avansat.
Permite crearea sau editarea facilă a unor programe complexe, alcătuite din mai multe fișiere.
Pe lângă posibilitatea de a edita structura de fișiere și directoare a proiectului, editorul recunoaște limbajele de programare cele mai folosite și evidențiază cuvintele cheie specifice fiecărui limbaj.
În plus, recunoaște variabilele definite de utilizator și le sugerează acestuia în timpul scrierii codului.

Pe lângă caracteristicile de bază, cum am menționat și mai sus, Sublime Text oferă suport pentru extensii care aduc funcționalități cum ar fi integrarea cu sisteme de versionare a codului (ex. Git), recunoașterea unor limbaje de programare mai puțin cunoscute sau diverse teme.

Pentru a instala Sublime Text, trebuie să urmăm pașii descriși pe pagina web dedicată editorului\footnote{\url{https://www.sublimetext.com/3}}.
În momentul de față este recomandată folosirea versiunii 3, care a fost lansată drept versiune stabilă de curând.
Editorul nu este gratis, el poate fi folosit pentru o perioadă limitată, după care este necesară achiziționarea unei licențe de folosire.

\subsubsection{Atom}
\label{sec:appdev:dev:editor-gui:atom}

Atom este un editor asemănător cu Sublime Text, dezvoltat de GitHub.
Principalul avantaj pe care acesta îl are față de Sublime Text este că are sursa deschisă și nu se percepe nicio taxă pentru folosire, deci poate fi considerat o variantă mai accesibilă față de Sublime Text.

Informații despre instalarea editorului Atom se găsesc pe pagina acestuia\footnote{\url{https://atom.io/}}.

\subsubsection{Visual Studio Code}
\label{sec:appdev:dev:editor-gui:vscode}

Visual Studio Code (VSCode) este un editor de cod dezvoltat de Microsoft.
Similar cu Atom, editorul are sursa deschisă și poate fi folosit gratuit.

Precum cele două editoare descrise mai sus, VSCode poate fi folosit pe sisteme Windows, Linux și macOS.
Pentru a-l instala urmăm pașii descriși la în la \url{https://code.visualstudio.com/Download/}.

\subsection{Medii integrate de dezvoltare (IDE)}
\label{sec:appdev:dev:ide}

Am vorbit despre editoare și importanța lor în dezvoltarea programelor.
Pe lângă editoare avansate, avem posibilitatea de a folosi medii integrate de dezvoltare.
La o primă vedere, acestea sunt similare cu editoarele de text.
IDE-urile \abbrev{IDE}{Integrated Development Environment} (\textit{Integrated Development Environment}) au, în schimb, anumite funcționalități avansate, multe din ele fiind adaptate unui singur limbaj de programare, sau unui număr redus de limbaje.
În primul rând, orice IDE are integrat un compilator sau interpretor pentru limbajul sau limbajele de programare pe care le suportă și la o simplă apăsare de buton programul este rulat.

Având în vedere că editoare precum Sublime Text sau Visual Studio Code vin cu un număr mare de extensii care pot adăuga funcționalități specifice unui anumit limbaj de programare, diferența între un editor avansat și un IDE este de multe ori greu de sesizat.
Astfel, unele din funcționalitățile specifice unui IDE se regăsesc și în editoarele avansate, cum ar fi depistarea erorilor, căutarea avansată a unei variabile, afișarea definiției funcției utilizate, sau chiar rularea programului la o apăsare de buton.

Pe de altă parte, există multe funcționalități pe care un IDE le aduce în plus.
De exemplu, în ceea ce privește interpretarea codului sursă pe măsura ce acesta e scris, majoritatea editoarelor pot să identifice variabilele declarate, dar nu și tipul acestora.
De aceea un IDE va oferi sugestii mai relevante programatorului.
Un alt avantaj este posibilitatea de a refactoriza codul, adică putem alege să schimbăm numele unei variabile și schimbarea va avea loc în toate fișierele, peste tot unde variabila este folosită.

Multe dintre limbajele de programare existente sunt folosite pentru dezvoltarea de aplicații grafice, cum ar fi Objective-C pentru aplicații iOS.
Pentru aceste cazuri, IDE-urile oferă un mediu vizual pentru dezvoltarea interfeței grafice în care programatorul poate să construiască vizual interfața și codul specific este generat automat, ca în \labelindexref{Figura}{fig:appdev:gui-interface}.

\begin{figure}[!htbp]
  \centering
  \includegraphics[width=0.8\textwidth]{chapters/06-appdev/img/xcode-interface.png}
  \caption{Construirea vizuală a interfeței unei aplicații într-un IDE}
  \label{fig:appdev:gui-interface}
\end{figure}

Un alt avantaj important al mediilor de dezvoltare integrate sunt metodele avansate de a depana programele.
Folosind sistemul de depanare integrat în IDE, putem configura puncte în care să oprim execuția programului, după care putem inspecta valoarea curentă a variabilelor folosite sau valorile întoarse de funcții.
Mai mult, putem chiar modifica valorile unor variabile pentru a influența execuția programului.

În general, pentru fiecare limbaj de programare ales, avem un IDE adaptat acestuia.
Există totuși și destul de multe IDE-uri ce suportă mai multe limbaje.
Se recomandă folosirea unui IDE dedicat limbajului de programare folosit, dacă acesta există, pentru că ne ușurează mult munca.
Un IDE ne permite să ne concentrăm pe scrierea codului, fără a ne consuma timp cu a rula comenzi pentru compilarea, rularea sau depanarea acestuia.

Mai departe vom menționa câteva medii de dezvoltare importante pentru cele mai folosite limbaje de programare.

\subsubsection{Microsoft Visual Studio și MonoDevelop}
\label{sec:appdev:dev:ide:mono}

Microsoft Visual Studio este editorul dedicat dezvoltării în C/C++ și C\#.
 Pentru că este suportat doar pe sisteme Windows sau macOS, pentru sisteme Linux recomandăm folosirea IDE-ului MonoDevelop.
MonoDevelop este open source, dar este susținut de Microsoft, împreună cu Microsoft Visual Studio.

\subsubsection{Eclipse}
\label{sec:appdev:dev:ide:eclipse}

Eclipse este unul din IDE-urile cele mai folosite pentru dezvoltarea de aplicații Java.
Înainte de apariția Android Studio, era folosit chiar pentru dezvoltarea de aplicații Android, prin instalarea unei extensii.

\subsubsection{NetBeans}
\label{sec:appdev:dev:ide:netbeans}

NetBeans e un IDE cu sursă deschisă specializat pentru limbajul Java.
El oferă o modalitate ușoară de a crea interfețe grafice pentru aplicații Java, împreună cu editorul pentru dezvoltarea programelor.

\subsubsection{Xcode}
\label{sec:appdev:dev:ide:xcode}

Xcode este IDE-ul dezvoltat de Apple pentru crearea în principal de aplicații pentru iOS și macOS.
Este adaptat pentru C, Objective-C și Swift.
Objective-C e un limbaj de programare ce extinde limbajului C, folosit pentru dezvoltarea de aplicații iOS și macOS.
Swift este succesorul limbajului Objective-C.

O caracteristică importantă a Xcode este editorul de interfețe grafice integrat (storyboardul), care permite generarea vizuală a interfețelor grafice și conectarea mai multor ferestre fără necesitatea de a scrie cod.

În ceea ce privește editorul pentru scrierea codului sursă, oricare din editoarele descrise mai sus suportă un număr mare de limbaje de programare și este adaptat pentru acestea.
Recomandarea noastră este să încercați mai multe editoare și să alegeți pe cel pe care îl considerați cel mai comod în folosire, iar pentru limbajele de programare care au un IDE dedicat, se recomandă folosirea IDE-ului.

\subsection{Biblioteci și framework-uri}
\label{sec:appdev:dev:libs}

Un alt aspect de care trebuie să ținem cont în dezvoltarea programelor este folosirea bibliotecilor și a framework-urilor.

Bibliotecile sunt colecții de resurse pe care le integrăm în aplicațiile noastre pentru a ne ușura procesul de dezvoltare.
De exemplu, dacă dorim să dezvoltăm o aplicație Python care face cereri HTTP\abbrev{HTTP}{Hypertext Transfer Protocol}, vom folosi biblioteca \textit{requests} care expune funcția \textit{get()} pentru a face o cerere.
Fără a folosi biblioteca, ar trebui să scriem noi tot codul care se conectează la un server, generează pachetul pentru cerere, îl trimite, apoi preia și interpretează răspunsul.

O unealtă asemănătoare bibliotecilor sunt framework-urile.
Și acestea sunt unelte care ușurează procesul de dezvoltare, dar ele oferă un schelet pe care aplicația îl urmează pentru a beneficia de proprietățile framework-ului.
Practic putem să ne gândim că biblioteca e un modul care odată integrat în aplicație aduce funcții sau constante noi pe care le putem folosi când e cazul, în timp ce framework-ul impune anumite funcții sau elemente ce trebuie implementate în aplicație și aplicația trebuie să se muleze după acea structură.

Pentru a putea integra biblioteci și framework-uri în aplicațiile pe care le dezvoltăm, acestea trebuie să existe pe sistemul pe care lucrăm.
Vom vedea mai departe, că de multe ori fiecare platformă de dezvoltare specifică unui limbaj vine cu propriul sistem de instalare a acestor module.

\subsection{Rularea programului}
\label{sec:appdev:dev:run}

Odată ce am scris codul sursă și am instalat bibliotecile sau framework-urile folosite, pasul final este să rulăm programul.
Complexitatea acestui pas depinde de complexitatea programului obținut.
Pentru un program simplu, cu un fișier care conține codul sursă și care nu folosește nicio bibliotecă, în general, este suficient să specificăm fișierul sursă și compilatorul sau interpretorul se ocupă de restul procesului.
\labelindexref{Listing}{lst:appdev:simple-compile} conține comanda de compilare și de rulare a unui program simplu.

\begin{screen}[caption={Compilarea și rularea unui program simplu},label={lst:appdev:simple-compile}]
student@uso:~$ gcc main.c -o main
student@uso:~$ ./main
\end{screen}

În cazul unui program complex care conține mai multe fișiere, vom rula mai multe comenzi, ca în \labelindexref{Listing}{lst:appdev:multiple-compile}.
Linia 1 și linia 2 conțin comenzi de compilare, linia 3 este comanda de linking, iar linia 4 este comanda de rulare.

\begin{screen}[caption={Compilarea și rularea unui program din mai multe surse},label={lst:appdev:multiple-compile}]
student@uso:~$ gcc -c utils.c
student@uso:~$ gcc -c main.c
student@uso:~$ gcc utils.o main.o -o main
student@uso:~$ ./main
\end{screen}

\subsection{Automatizarea procesului de dezvoltare}
\label{sec:appdev:automation}

Cum am observat mai sus, pentru aplicații complexe, compilarea sau rularea acestora constă în rularea unei secvențe de comenzi.
Toate aceste comenzi trebuie rulate indiferent dacă se aduc modificări majore programului, sau dacă se schimbă doar o linie de cod.
Astfel, nu e greu să ne dăm seama că avem nevoie de o modalitate de a automatiza acțiunile necesare compilării și rulării codului, astfel încât să rulăm o comandă simplă care să \textit{se ocupe de tot}.
Practic, putem să ne gândim că fișierele care alcătuiesc programul pe care l-am dezvoltat sunt ingredientele necesare pentru o prăjitură, iar prăjitura este executabilul sau chiar rezultatul rulării programului.
Astfel, similar unei rețete care descrie pașii necesari pentru a obține prăjitura, vom scrie un fișier care conține pașii necesari obținerii rezultatului programului, iar utilitarul în cauză este bucătarul care urmează rețeta.

Unele dintre utilitarele cele mai folosite pentru automatizarea acestor taskuri sunt Make, Grunt, Gulp.

\section{Medii de dezvoltare}
\label{sec:appdev:dev-env}

Pentru fiecare limbaj de programare folosit avem nevoie de mediul de dezvoltare aferent (aplicații, biblioteci, compilator și/sau interpretor), pentru a putea rula aplicațiile pe care le scriem.
Bineînțeles că putem scrie codul sursă în orice editor, fără a folosi nici un alt program adiacent, dar am obține un fișier cu instrucțiuni inutil, pe care nu îl putem compila sau interpreta.
De aceea, în primul rând, în funcție de tipul limbajului, avem nevoie de un compilator sau un interpretor, care să ne permită acest lucru.

Pe lângă compilator sau interpretor, pentru folosirea unui limbaj de programare avem nevoie să instalăm bibliotecile folosite.

Mai departe vom analiza particularități în instalarea compilatorului sau a interpretorului pentru mai multe limbaje de programare.

\subsection{C / C++}
\label{sec:appdev:dev-env:c}

Pentru limbajul C, compilatorul cel mai cunoscut este GCC \abbrev{GCC}{GNU Compiler Collection} (\textit{GNU Compiler Collection}).
Conceput inițial pentru a compila limbajul C, în momentul de față GCC a fost extins pentru a suporta compilarea și a altor limbaje, cum ar fi C++ sau Java.

Pentru sistemele derivate din Debian, există un pachet special numit \textit{build-essential}, instalabil folosind comanda:

\begin{screen}
student@uso:~$ sudo apt install build-essential
\end{screen}

Pentru sistemele derivate din Redat, precum Fedora, CentOS etc.
trebuie instalate pachetele separat folosind comanda:

\begin{screen}
student@uso:~$ sudo dnf install make gcc g++
\end{screen}

Toate aceste pachete vor instala și limbajul C++.

GCC poate fi instalat și pe sisteme Windows, prin intermediul unei aplicații care care simulează un sistem Unix (Cygwin, MinGW etc.).

Pentru sisteme Windows, compilatorul de baza este integrat în aplicația Microsoft Visual C++.
Microsoft Visual C++ este un ecosistem de programare și constă într-un editor cu funcții avansate (sugestii de completare a codului, sugestii de posibile erori etc.) împreună cu compilatorul de C și C++ și se recomandă folosirea acestuia pentru dezvoltarea de programe C.

Pentru C++ se poate folosi aceeași colecție de compilatoare ca pentru limbajul C, adică GCC.
Cum am menționat mai sus, în momentul de față GCC suportă compilarea limbajului C++.

La folosirea \cmd{gcc}, sistemul analizează fișierele de intrare și hotărăște dacă limbajul compilat este C sau C++.
Pentru a eficientiza procesul, se poate folosi direct \cmd{g++}, adică compilatorul de C++ integrat în GCC, ca în \labelindexref{Listing}{lst:appdev:simple-cpp}.

\begin{screen}[caption={Exemplu compilare si rulare C++},label={lst:appdev:simple-cpp}]
student@uso:~$ g++ main.cpp -o main
student@uso:~$ ./main
\end{screen}

De asemenea, pentru sisteme Windows mediul de dezvoltare recomandat rămâne același.
Microsoft Visual C++ este recomandat și pentru dezvoltarea de aplicații C++.

\subsubsection{Dezvoltarea în C/C++}
\label{sec:appdev:dev-env:c-dev}

Cum am menționat anterior, compilatorul cel mai folosit pentru C și C++ este gcc/g++.
Pentru folosirea compilatorului este suficient să rulăm comanda \cmd{gcc} cu parametrii aferenți.

În procesul de compilare codul sursă este prelucrat și trece prin mai multe etape intermediare:

\begin{enumerate}
  \item preprocesare: se înlocuiesc macro-uri, se numerotează fiecare linie de cod;
    rezultatul e un fișier cu extensia \textbf{.i}
  \item compilare: codul sursă este transformat în cod în limbaj de asamblare;
    rezultatul e un fișier cu extensia \textbf{.s}
  \item asamblare: codul în limbaj de asamblare este transformat în cod mașină;
    rezultatul e un fișier cu extensia \textbf{.o}
  \item link-editare: se fac legăturile către fișiere externe care conțin simboluri sau funcții apelate din fișierul sursă;
    dacă avem o funcție definită într-un fișier și folosită în altul, implementarea funcției trebuie legată de apelul ei;
\end{enumerate}

La simpla rulare a comenzii \cmd{gcc}, se trece prin toate etapele menționate.
De exemplu, pentru un program C simplu (\texttt{main.c} fără nicio bibliotecă) vom rula comenzile indicate în \labelindexref{Listing}{lst:appdev:simple-compile}.

În schimb, pentru programe mai complexe, alcătuite din mai multe fișiere, e necesar să generăm fișierul obiect aferent fiecărui fișier sursă și apoi sa le link-edităm pentru a obține executabilul final, așa cum am indicat în \labelindexref{Listing}{lst:appdev:multiple-compile}.

\subsubsection{Automatizarea compilării în C/C++}
\label{sec:appdev:dev-env:c-dev}

Cum am menționat mai devreme, este obositor să rulăm toate comenzile de mai sus pentru fiecare modificare adusă în codul sursă.
De aceea există posibilitatea de a crea un fișier care conține toate aceste comenzi și care le poate rula pe toate printr-un singur apel.

Utilitarul de automatizare cel mai folosit pentru aplicațiile C/C++ este utilitarul Make.
Pentru a folosi Make e suficient să creăm un fișier cu numele \file{Makefile} în structura programului nostru.
La rularea comenzii \cmd{make}, utilitarul găsește acel fișier și execută instrucțiunile descrise.

Un fișier \file{Makefile} urmează formatul din \labelindexref{Listing}{lst:appdev:makefile-format}.

\begin{screen}[caption={Formatul unui fișier Maefile},label={lst:appdev:makefile-format}]
regula: dependente
<TAB> comanda
\end{screen}

Fișierul poate să conțină mai multe astfel de linii, precum fișierul \file{Makefile} din \labelindexref{Listing}{lst:appdev:makefile}.

\begin{screen}[caption={Exemplu Makefile},label={lst:appdev:makefile}]
build: utils.o hello.o help.o
       gcc utils.o help.o hello.o -o hello

all:
       gcc simple_hello.c -o simple

utils.o: utils.c
       gcc -c utils.c

hello.o: hello.c
       gcc -c hello.c

help.o: help.c
       gcc -c help.c

clean:
       rm -f *.o hello
\end{screen}

Regula reprezintă numele unei instrucțiuni.
La simpla rulare a comenzii \cmd{make} prima regulă din fișier este cea care va fi executată.
În schimb, dacă vom scrie \cmd{make clean}, se va executa comanda aferentă regulii \texttt{clean}.

Dependențele sunt fișiere și/sau reguli necesare pentru a putea rula o comandă.
Practic, la rularea unei comenzi se verifică dacă fișierul din dependență există.
Dacă acesta există, putem rula comanda, dacă nu, se caută o altă regulă cu acel nume și se va rula acea comandă (dacă dependențele ei sunt îndeplinite, dacă nu, se continuă lanțul de dependențe).
Bineînțeles, sunt cazuri în care dependențele pot să lipsească.

Acum să luăm cazul programului dezvoltat mai sus, care depinde de mai multe fișiere și să presupunem că am schimbat o linie într-un fișier.
La rularea comenzii \cmd{make}, toate fișierele ar fi refăcute, ceea ce nu este eficient.
De aceea, \cmd{make} ține cont de încă un aspect la verificarea dependențelor.
Dacă dependența (fișierul) există, \cmd{make} va urmări lanțul de dependențe până ajunge la fișierele de bază ale acesteia.
Dacă fișierele au fost modificate după ce dependența a fost creată, aceasta va fi regenerată, în caz contrar nu se va regenera dependența.

\subsection{Java}
\label{sec:appdev:dev-env:java}

Fiind un limbaj hibrid, pentru Java sistemul de execuție a codului sursă este mai complex.
Cum am menționat mai sus, în general limbajele hibride trec printr-o etapă de compilare în urma căreia rezultă un fișier bytecode, iar apoi acel cod este interpretat.

Pentru majoritatea limbajelor hibride (Python, JavaScript) există un sistem care conține atât compilatorul de bytecode cât și interpretorul, sau mașina virtuală.
Pentru Java, cele două elemente sunt separate clar, avem compilatorul care generează codul sursă în bytecode și mașina virtuală ce execută fișierele bytecode.
Cele două se instalează separat, ca două aplicații diferite și independente.
Astfel, putem avea un sistem care doar generează fișierele bytecode și un alt sistem care le execută.

În ceea ce privește platformele pentru a compila și a rula Java, putem alege între două platforme principale: OracleJDK sau OpenJDK.
Cum sugerează și numele, OracleJDK este dezvoltată de Oracle, în timp ce OpenJDK este o platformă open source și este special concepută pentru sisteme Unix.
Pe lângă JDK (Java Development Kit), putem instala OracleJRE sau OpenJRE (Java Runtime Environment).

În timp ce JDK este sistemul complet ce compilează și execută aplicații Java, JRE este doar mașina virtuală care rulează bytecode-ul generat de către compilator.
Ea este inclusă în JDK, dar poate fi instalată și separat.
Astfel, putem avea sisteme care doar rulează aplicații Java și care necesită bytecode-ul generat de JDK.

Pe sisteme Linux OpenJDK este într-un pachet cu numele de forma \texttt{openjdk-X-jdk}, unde \texttt{X} este numele versiunii.
De exemplu, pe Ubuntu 18.04 pachetul este \texttt{openjdk-8-jdk}.

Pentru sisteme care rulează Windows, se recomandă instalarea OracleJDK.
Deși se găsesc versiuni OpenJDK compatibile cu Windows, acestea nu sunt întreținute și prezintă riscuri de securitate.

Pentru a instala OracleJDK pe un sistem Windows, accesați site-ul Oracle\footnote{\url{https://www.oracle.com/technetwork/java/javase/overview/install-windows-142126.html}} și urmați instrucțiunile.

O particularitate a Java este aceea că bibliotecile sunt împachetate ca niște arhive (jar \abbrev{JAR}{Java Archive} - \textit{Java Archive}) care sunt incluse în aplicațiile pe care le dezvoltăm.
Fișierul \texttt{.jar} conține resurse diverse, de la definiții de clase la imagini sau alte fișiere multimedia.
Fiecare astfel de arhivă poate conține un fișier manifest ce oferă informații despre pachet și despre utilizarea sa.

\subsection{C\#}
\label{sec:appdev:dev-env:csharp}

Pentru C\# sistemul de execuție este .NET, dezvoltat de Microsoft.
Așadar, sistemele cu Windows au deja sistemul instalat.

Pentru sisteme Unix, se recomanda folosirea Mono, cu ajutorul pachetului \texttt{mono-devel}.
Mono este platforma open-source pentru C\#, susținută tot de Microsoft.

Pentru aplicațiile dezvoltate în C\# putem folosi fie Mono, fie .NET.
Indiferent de platforma folosită, pentru a putea instala biblioteci putem folosi utilitarul NuGet, dezvoltat de Microsoft.

La fel ca în Java, în C\# bibliotecile sunt similare unei arhive.

\subsection{Python}
\label{sec:appdev:dev-env:python}

La fel ca Java, Python e un limbaj de programare hibrid.
Diferența între cele două este că pentru Python nu se diferențiază compilatorul de mașina virtuală, cele două sunt integrate.

Pentru a folosi Python pe un sistem Unix este suficient să instalăm pachetul \texttt{python}.

Pentru sistemele Windows, site-ul \textit{python.org} ne oferă resursele necesare pentru instalare.

La instalarea interpretorului de Python este foarte importantă versiunea acestuia.
Putem alege între Python2 sau Python3, cele două fiind diferite.
Python 3.0 a fost lansat în anul 2008 și prezintă o sintaxă diferită de Python 2.
De aceea este important să instalăm un interpretor potrivit pentru versiunea Python pe care o folosim la dezvoltarea aplicațiilor.

Se recomandă folosirea Python3 deoarece Python2 nu mai este adus la zi.

Când vine vorba de dezvoltarea aplicațiilor folosind Python, abordarea e puțin diferită de C, pentru că Python e un limbaj interpretat.
Drept urmare, pentru a putea rula programul pe care îl scriem, va trebui să rulăm interpretorul și să îi spunem acestuia care e programul pe care dorim să îl interpreteze.

Pentru că în general vom lucra la proiecte complexe, care conțin mai multe fișiere, similar cu programele din C, în Python vom avea un fișier de bază care depinde de celelalte fișiere.
Astfel, pentru a rula aplicația e suficient să specificăm fișierul de pornire, ca în \labelindexref{Listing}{lst:appdev:python-example}.

\begin{screen}[caption={Exemplu rulare Python},label={lst:appdev:python-example}]
student@uso:~$ python main.py
\end{screen}

Cum am menționat mai sus, Python oferă un număr mare de biblioteci și framework-uri care pot fi integrate în programele pe care le dezvoltăm.
Pentru a putea folosi o bibliotecă, aceasta trebuie să existe pe sistemul care rulează programul.
Pentru instalarea unei biblioteci, folosim utilitarul \cmd{pip}, așa cum am precizat în \labelindexref{Secțiunea}{sec:package:specific}.

\subsection{Node.js}
\label{sec:appdev:dev-env:js}

Când menționăm Node.js, ne referim deja la mediul de dezvoltare al aplicațiilor JavaScript.

JavaScript este un limbaj folosit pentru aplicații web, iar interpretoarele de JavaScript sunt integrate în browsere.
De aceea a apărut Node.js, care e o platformă ce permite executarea codului JavaScript în afară browserului.
Unul din interpretoarele de JavaScript este V8, dezvoltat de Google și integrat în Google Chrome.
Node.js integrează acest interpretor.

Pentru a instala Node.js pe sisteme Linux, recomandăm folosirea instanței Snap ca în \labelindexref{Listing}{lst:appdev:install-nodejs}.
Platforma se poate instala și instalând pachetul \texttt{nodejs} în mod clasic, dar se va descărca o versiune foarte veche a platformei.

\begin{screen}[caption={Instalare Node.js},label={lst:appdev:install-nodejs}]
student@uso:~$ sudo snap install --classic node
node (12/stable) 12.18.4 from NodeSource, Inc.
(nodesource) installed

student@uso:~$ node
Welcome to Node.js v12.18.4.
Type ".help" for more information.
>
\end{screen}

Pentru instalarea pe sisteme ce rulează Windows, se recomandă descărcarea kitului de instalare de pe site-ul Node.js.

În general, dacă alegem să dezvoltăm o aplicație folosind Node.js, dorim să dezvoltăm o aplicație web la care atât partea de server, cât și cea de client să fie scrise folosind JavaScript.
Pentru partea de server vom folosi Node.js pentru a putea rula programul, în timp ce partea de client va fi interpretată în browser.

Când dezvoltăm o aplicație în Node.js, aspectele de care trebuie să ținem cont sunt similare cu cele care au legătură cu Python.
Similar cu Python, pentru rularea codului sursă scris avem nevoie de interpretorul necesar instalat.

Pentru a rula o aplicație Node.js folosim comanda:

\begin{screen}
student@uso:~$ node main.js
\end{screen}

Pentru manipularea modulelor externe se folosește utilitarul \cmd{npm}.
Pentru instalarea unei biblioteci comanda necesară este \cmd{npm install $<$biblioteca$>$}.

Cum am menționat mai sus, de cele mai multe ori, când folosim Node.js, vom crea aplicații complexe, de cele mai multe ori, aplicații web.
De aceea, după ce scriem codul sursă vom dori să îl prelucrăm.
De exemplu putem alege să minimizăm codul, adică să restrângem tot codul într-un singur fișier, greu inteligibil pentru dezvoltatori, pentru a ne asigura că nu poate fi ușor copiat.
Pe lângă asta, putem alege să trecem tot codul sursă printr-un utilitar care verifică dacă există greșeli, cum ar fi nedeclararea unei variabile.

Pentru că sunt multe astfel de operații pe care vrem să le aplicăm codului sursă înainte de a-l distribui, putem automatiza întregul proces.
Unul din utilitarele folosite la această automatizare este \cmd{grunt}.
Putem să ne gândim la \cmd{grunt} ca la un utilitar semănător cu Make, doar că operațiile pe care \cmd{grunt} le efectuează implică preprocesarea codului sursă.
Similar cu un fișier \file{Makefile}, pentru \cmd{grunt}, vom crea un fișier numit \file{Gruntfile} care conține toate regulile, ca în \labelindexref{Listing}{lst:appdev:gruntfile}.

\begin{screen}[caption={Exemplu cod Gruntfile},label={lst:appdev:gruntfile}]
module.exports = function(grunt) {

 grunt.initConfig({
   jshint: {
     files: ['Gruntfile.js', 'src/**/*.js', 'test/**/*.js'],
     options: {
       globals: {
         jQuery: true
       }
     }
   },
   watch: {
     files: ['<%= jshint.files %>'],
     tasks: ['jshint']
   }
 });

 grunt.loadNpmTasks('grunt-contrib-jshint');
 grunt.loadNpmTasks('grunt-contrib-watch');

 grunt.registerTask('default', ['jshint']);

};
\end{screen}

\section{Depanarea programelor}
\label{sec:appdev:debug}

Pe măsură ce scriem tot mai multe programe, ne dăm seama că de cele mai multe ori acestea nu funcționează cum ne dorim de la început.
De multe ori se va întâmpla ca programul rulat să arunce o eroare sau să nu obțină rezultatul dorit.
Deși am dori să putem spune că \textit{e vina calculatorului}, orice eroare sau comportament eronat apare dacă noi nu am scris ceva bine în codul sursă.
Erorile pot să varieze de la un simbol uitat, la accesări ilegale de memorie sau chiar erori în logica programului.
Astfel, o mare parte din timpul destinat dezvoltării aplicațiilor îl dedicăm depanării.
De aceea e de preferat să fim eficienți în descoperirea erorilor.

Am menționat mai sus că unul din aspectele importante ale unui IDE este posibilitatea de a seta puncte de oprire în execuția programului.
Astfel, odată ce o linie de cod este executată, întregul proces se oprește și noi putem monitoriza starea curentă.
Adică putem vizualiza valorile pe care le au variabilele sau adresele de memorie la care se găsesc, ca în \labelindexref{Figura}{fig:appdev:ide-debug}.

\begin{figure}[!htbp]
  \centering
  \includegraphics[width=\textwidth]{chapters/06-appdev/img/codeblocks-debug.png}
  \caption{Interfață de depanare într-un IDE (codeblocks)}
  \label{fig:appdev:ide-debug}
\end{figure}

Unul dintre cele mai cunoscute utilitare pentru depanare este GDB \abbrev{GDB}{GNU Debbuger} (\textit{GNU Debugger}).
GDB oferă suport pentru toate operațiile menționate mai sus și este compatibil cu toate sistemele de operare existente.

Cea mai simplă metodă de a depana un program este să afișăm mesaje pe parcursul execuției.
Astfel putem vizualiza valorile variabilelor sau blocurile de cod care se execută sau nu.
Deși pare un mod primitiv de a face depanare, este cel mai la îndemână.

De fiecare dată când ceva nu funcționează, se recomandă verificarea pas cu pas a liniilor de cod și printarea rezultatului.
De foarte multe ori eroarea se observă prin afișarea variabilelor.

În plus, nu este o practică eficientă să scriem zeci de linii de cod fără a testa.
E recomandat să scriem în incremente mici și să ne asigurăm ca fiecare bloc funcționează.
Altfel, ne vom trezi puși în fața a zeci de linii de cod care nu funcționează și va fi dificil să identificăm problema.

\section{Sisteme de management și de versionare a codului sursă}
\label{sec:appdev-versioning}

De cele mai multe ori, vom lucra la proiecte complexe, în cadrul unei echipe.
Acest lucru înseamnă că vom avea multe fișiere cu cod sursă pe care le vom modifica în paralel cu alte persoane.
Drept urmare, avem nevoie de o modalitate de a împărți proiectul într-un mod eficient.
Deși la început acest lucru nu pare neapărat dificil, pe măsură ce proiectul crește, este tot mai greu.
Pe lângă partajarea efectivă a întregii structuri a programului, avem nevoie de o modalitate de a urmări modificările aduse și cine ce modificări a făcut.
Să ne gândim la situația în care suntem o echipă de 5 oameni și cineva face o modificare ce rezultă într-un comportament defectuos al programului.
Având în vedere că suntem 5 oameni care au adus simultan modificări codului sursă, va fi dificil să identificăm eroarea.
În cazul unui proiect avansat, problema e și mai mare.

De aceea metode precum \textit{folosim un director online comun sau chiar trimitem sursele modificate prin email} nu sunt recomandate pentru a lucra în echipă la un proiect și de aceea există sisteme dedicate partajării de cod sursă.
Un astfel de sistem este Git.

Git este un sistem de management și versionare a codului sursă care permite partajarea unui proiect în cadrul unei echipe.
Proiectul este stocat într-un depozit (\textit{repository}).
Repository-ul conține fișierele efective ale proiectului și multe alte informații despre acesta.
Folosind Git, fiecare persoană lucrează la o versiune proprie a proiectului, pe care apoi o urcă online și este automat integrată în proiect.
Ceilalți colaboratori descarcă modificările aduse când doresc.
Sistemul urmărește modificările pe care fiecare persoană le aduce și la orice moment putem descărca o versiune anterioară a codului sursă.
Pentru a realiza toate acestea avem la dispoziție următoarele operații de bază:

\begin{itemize}
  \item \cmd{clone}: se clonează întreg repository-ul pe sistemul local;
    practic se va crea un director care conține toate fișierele aferente programului, plus niște fișiere care stochează informații despre proiect (de ex. cine a creat repository-ul, ce modificări au fost aduse etc.);
  \item \cmd{commit}: se salvează toate modificările aduse proiectului;
    starea actuală este salvată local;
    dacă modificările nu sunt publicate online, acestea nu sunt vizibile colaboratorilor;
  \item \cmd{push}: se publică modificările salvate prin commit;
    odată publicate aceste modificări, colaboratorii pot să vizualizeze cine le-a adus și pot să urmărească schimbările linie cu linie;
  \item \cmd{pull}: se descarcă local ultimele modificări publicate de ceilalți colaboratori și se integrează în versiunea locală a proiectului.
\end{itemize}

\section{Licențe pentru programe}
\label{sec:appdev:license}

Un aspect important când vine vorba de dezvoltarea programelor este licența acestuia.

Licența unui program oferă informații despre dreptul de folosire și distribuție a programului.

De multe ori vom dori să publicăm programele pe care le dezvoltăm pentru a permite altor persoane să le folosească sau chiar să le integreze în proiectele proprii.
Când facem acest lucru, este important să definim scopul aplicației pe care o dezvoltăm și să îi atribuim o licență corespunzătoare.
Astfel, cei care vor folosi programul nostru vor ști ce pot și ce nu pot să facă cu el.

Pe de altă parte, în procesul de dezvoltare a unui program vom folosi un număr semnificativ de biblioteci și framework-uri.
Când facem acest lucru este foarte important să verificăm licența fiecărui pachet extern pe care îl integrăm în aplicația noastră.
Licențele modulelor externe folosite pot influența folosirea programului pe care îl dezvoltăm.

Acum că știm că e important să stabilim o licență pentru programele pe care le dezvoltăm, întrebarea care urmează este, cum stabilim aceste reguli și cum alegem licența potrivită.
De foarte multe ori chiar licența unui modul extern folosit poate dicta licența pe care trebuie să o atribuim programului nostru.
Ca să înțelegem mai bine ce caracteristici poate să aibă o licență, vom menționa cele mai cunoscute și folosite licențe software.

\subsection{GNU GPL}
\label{sec:appdev-licensing-gpl}

GNU GPL \abbrev{GNU GPL}{GNU General Public License} este prescurtarea de la \textit{GNU General Public License} și de cele mai multe ori ne referim la această licență drept GPL.

Un program însoțit de o licență GPL poate fi distribuit doar dacă sursa acestuia este publică și programul poate fi integrat în orice alt program care este public și nu are nicio restricție de folosire.
Practic, orice program care folosește un modul GPL trebuie să aibă la rândul lui licență GPL.
În plus, biblioteca integrată poate fi modificată

\subsection{GNU LGPL}
\label{sec:appdev-licensing-lgpl}

GNU LGPL \abbrev{GNU LGPL}{GNU Lesser General Public License} (simplu LGPL) este o licență cu restricții mai reduse față de GNU GPL.
Numele vine de la \textit{GNU Lesser General Public License}.
În cazul acestei licențe programul în cauză poate fi integrat în alte programe care au orice altă licență.
Singura restricție este ca modulul LGPL folosit să fie făcut public în cadrul aplicației.
De asemenea, în cazul în care biblioteca LGPL este modificată și abia apoi integrată într-o aplicație, biblioteca modificată trebuie publicată sub licență LGPL.

\subsection{MIT}
\label{sec:appdev-licensing-mit}

MIT \abbrev{MIT}{Massachusetts Institute of Technology} (\textit{Massachusetts Institute of Technology}) este o licență mai permisivă față de GPL.
Orice bibliotecă care are licența MIT poate fi integrată în orice program și acesta poate fi distribuit, fără necesitatea de a face programul public.
Singura condiție este ca licența MIT să apară în programul principal.

\section{Sumar}
\label{sec:appdev-summary}

Pe parcursul acestui capitol am prezentat principalele tipuri de limbaje de programare: compilate, interpretate și hibride.
Am discutat despre avantajele și dezavantajele fiecăruia.
De asemenea, am trecut în revistă principalele limbaje de programare folosite din fiecare categorie, evidențiind unde se folosește fiecare.

Sunt importante de reținut etapele dezvoltării unei aplicații.
În primul rând, trebuie să hotărâm ce aplicație dorim să realizăm, cine o va folosi și care va fi mediul hardware și software în care va rula.
Odată stabilite acestea, trebuie să alegem limbajul potrivit, mediul de programare și să instalăm bibliotecile necesare.

Următorul pas este scrierea codului sursă, compilarea acestuia și testarea lui.
Acest proces este repetat de foarte multe ori până la definitivarea aplicației.

Odată ce aplicația funcționează corespunzător, trebuie să îi adăugăm o licență de distribuție și să facem un pachet de instalare.

Aplicația este gata de lansare.
